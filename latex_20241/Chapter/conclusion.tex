
% \section*{KẾT LUẬN}
% \phantomsection\addcontentsline{toc}{section}{\numberline{} KẾT LUẬN}
% \subsection*{Kết luận chung}
% \addcontentsline{toc}{section}{\numberline{} Kết luận chung}


% Trong quá trình thực hiện đồ án,
%  chúng em đã thực hiện các nghiên cứu và phát 
%  triển để hiện thực một hệ thống chất lượng và hữu ích
%   cho việc quản lý dữ liệu điện tâm đồ. Trải qua các phần
%    của đồ án, chúng em đã tiến hành nghiên cứu về cơ sở lý
%     thuyết về tín hiệu ECG, thiết kế cơ sở dữ liệu phù hợp,
%      xây dựng giao diện người dùng tương tác và triển khai
%       ứng dụng trên môi trường máy chủ thực tế.

%       Qua quá trình
%       thực hiện, chúng em đã đạt được những kết quả đáng kể,
%       bao gồm xây dựng giao diện người dùng trên web, tích hợp các
%        chức năng quản lý bệnh nhân, bác sĩ, tin tức và dữ liệu. Xây dựng thành công tính năng tự động triển khai ứng dụng khi có thay đổi mới.



% % Cung cấp khả năng phân tích dữ liệu và đưa ra các gợi ý chẩn đoán.
% Tuy nhiên, trong quá trình triển khai, chúng em cũng đã gặp một số thách thức và vấn đề chưa giải quyết hoàn toàn. Ví dụ, việc tích hợp thêm các thuật toán phân tích tiên tiến có thể là một hướng phát triển tiềm năng. Đồng thời, cần tối ưu hóa hệ thống để đảm bảo tính sẵn sàng và khả năng mở rộng trong tương lai.

% \subsection*{Hướng phát triển}
% \phantomsection\addcontentsline{toc}{section}{\numberline{} Hướng phát triển}


% Để nâng cao hiệu suất và tính năng của ứng dụng, chúng em đề xuất một số hướng phát triển tiềm năng sau này:

% \begin{adjustwidth}{1.5em}{}
%   \begin{itemize}
%       \item Phát triển thêm các thuật toán phân tích sâu hơn để cung cấp độ chính xác cao hơn trong việc chẩn đoán các vấn đề về tim.

  
%       \item Tích hợp khả năng dự báo và cảnh báo sớm về các tình trạng nguy hiểm hoặc không bình thường trong dữ liệu ECG.

  
%       \item Mở rộng ứng dụng để hỗ trợ nhiều loại thiết bị cảm biến và chuẩn giao tiếp khác nhau.

  
%       \item  Phát triển giao diện web thân thiện hơn để người dùng có thể truy cập và theo dõi dữ liệu ECG mọi lúc, mọi nơi.

%   \end{itemize}
%   \end{adjustwidth}


% \subsection*{Kiến nghị và đề xuất}
% \phantomsection\addcontentsline{toc}{section}{\numberline{} Kiến nghị và đề xuất}


% Dựa trên kinh nghiệm từ quá trình phát triển đồ án, chúng em có một vài đề xuất như sau:

% \begin{adjustwidth}{1.5em}{}
%   \begin{itemize}
%       \item Đảm bảo tích hợp các phản hồi từ người dùng, chuyên gia y tế và các nhóm nghiên cứu để liên tục cải thiện và hoàn thiện ứng dụng.

  
%       \item Xây dựng thêm các buổi hội thảo để giới thiệu về hệ thống, hướng dẫn cách dùng cho các chuyên gia y tế hoặc các thành viên trong nhóm nghiên cứu muốn tích.

  
%       \item Tìm kiếm cơ hội để đưa hệ thống vào thực tiễn trong môi trường bệnh viện hoặc phòng nghiên cứu.

%     \end{itemize}
%   \end{adjustwidth}



% \cleardoublepage