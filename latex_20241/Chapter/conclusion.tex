
\section*{KẾT LUẬN}
\phantomsection\addcontentsline{toc}{section}{\numberline{} KẾT LUẬN}
\subsection*{Kết luận chung}
\addcontentsline{toc}{section}{\numberline{} Kết luận chung}

Khi thực hiện đồ án, nhóm chúng em đã tiến hành nghiên cứu và phát triển một hệ thống nhằm hỗ trợ quản lý dữ liệu sức khỏe tim mạch và tương tác giữa bệnh nhân - bác sĩ một cách hiệu quả.
Chúng em đã tìm hiểu về lý thuyết về các tín hiệu thể hiện sức khỏe tim mạch, thiết kế cấu trúc dữ liệu phù hợp, xây dựng giao diện trực quan và triển khai hệ thống trên máy chủ thực tế.

Trong suốt quá trình phát triển, chúng em đã đạt được những kết quả quan trọng như phát triển nền tảng web thân thiện, tích hợp các chức năng quản lý thông tin bệnh nhân, bác sĩ cùng dữ liệu liên quan và cho phép quá trình tương tác giữa bệnh nhân - bác sĩ diễn ra mượt mà.

Dù đạt được nhiều kết quả, nhóm cũng gặp một số khó khăn và vấn đề cần giải quyết. Chẳng hạn, việc bổ sung các thuật toán phân tích nâng cao có thể giúp cải thiện độ chính xác trong hỗ trợ y tế. Đồng thời, việc tối ưu hệ thống sẽ góp phần nâng cao hiệu suất cũng như khả năng mở rộng trong tương lai.
\subsection*{Hướng phát triển}
\phantomsection\addcontentsline{toc}{section}{\numberline{} Hướng phát triển}


Để nâng cao hiệu suất và phát triển tính năng của hệ thống, nhóm đề xuất như sau:

\begin{adjustwidth}{1.5em}{}
	\begin{itemize}
		\item Phát triển các thuật toán có độ phân tích sâu hơn nhằm mục đích tính toán chính xác hơn các chẩn đoán vấn đề về tim mạch.

		\item Ứng dụng khả năng dự báo và khả năng cảnh báo sớm khi các tình trạng nguy hiểm hay các tình trạng không bình thường của bệnh nhận trong dữ liệu sức khỏe tim mạch xuất hiện.


		\item Phát triển hệ thống để tương thích với đa dạng thiết bị cảm biến và các giao thức kết nối khác.


		\item Cải thiện thiết kế giao diện web để nâng cao trải nghiệm người dùng, giúp truy cập, theo dõi dữ liệu và tương tác với bác sĩ ở bất kỳ đâu.

	\end{itemize}
\end{adjustwidth}


\subsection*{Đề xuất}
\phantomsection\addcontentsline{toc}{section}{\numberline{} Đề xuất}


Dựa trên kinh nghiệm thu được từ quá trình thực hiện đồ án, nhóm chúng em xin đưa ra một vài đề xuất như sau:

\begin{adjustwidth}{1.5em}{}
	\begin{itemize}
		\item Cần đảm bảo tiếp thu phản hồi từ người dùng, các chuyên gia y tế và nhóm nghiên cứu để hệ thống có thể được cải tiến và nâng cao chất lượng liên tục.

		\item Tổ chức thêm các buổi hội thảo để giới thiệu hệ thống, đồng thời cung cấp hướng dẫn sử dụng cho các chuyên gia y tế và các thành viên nhóm nghiên cứu muốn tích lũy thêm kiến thức.

		\item Tìm kiếm cơ hội để đưa hệ thống vào thực tiễn trong môi trường bệnh viện hoặc phòng nghiên cứu.

		\item Tìm kiếm cơ hội áp dụng hệ thống vào môi trường phòng khám, bệnh viện hoặc phòng nghiên cứu, nhằm tối ưu hóa hiệu quả thực tiễn.

	\end{itemize}
\end{adjustwidth}



\cleardoublepage