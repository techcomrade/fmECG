\section*{TÓM TẮT ĐỒ ÁN}
\phantomsection\addcontentsline{toc}{section}{\numberline{}TÓM TẮT ĐỒ ÁN}

Đồ án "Phát triển ứng dụng di động tích hợp trong hệ thống quản lý dữ liệu tim
mạch, hỗ trợ kết nối giữa bệnh nhân - bác sĩ" phát triển một hệ thống tích hợp gồm Web/App/Server, hỗ trợ quản lý dữ liệu sức khỏe tim mạch, đặt lịch hẹn và cung cấp kênh giao tiếp hiệu quả giữa bệnh nhân và bác sĩ.
Hệ thống cho phép lưu trữ thông tin người dùng, dữ liệu lịch hẹn, dữ liệu các phiên đo và nội dung trao đổi giữa bệnh nhân và bác sĩ trên server, giúp người dùng dễ dàng tra cứu khi cần.

Một trong những điểm nổi bật của hệ thống là khả năng lưu trữ và hiển thị dữ liệu đo sức khỏe tim mạch dưới dạng biểu đồ trực quan. Tính năng này giúp bệnh nhân theo dõi sức khỏe lâu dài và cung cấp dữ liệu giá trị hỗ trợ bác sĩ trong việc chẩn đoán và điều trị.
Ngoài ra, hệ thống hỗ trợ đặt lịch hẹn trực tuyến, giúp bệnh nhân lựa chọn bác sĩ và thời gian thuận tiện. Bác sĩ có thể phê duyệt hoặc từ chối yêu cầu, với thông báo tự động gửi đến bệnh nhân. Sau khi lịch hẹn được xác nhận, bệnh nhân và bác sĩ có thể sử dụng tính năng nhắn tin để trao đổi thông tin sức khỏe.

Đồ án được thực hiện dựa trên quy trình phát triển phần mềm tiêu chuẩn, bao gồm các bước xác định yêu cầu, phân tích, thiết kế, triển khai và kiểm thử hệ thống. Phương pháp phân tích và thiết kế hướng đối tượng được áp dụng, với các luồng xử lý và hành động được biểu diễn thông qua sơ đồ UML.
Ứng dụng di động sử dụng Flutter framework, server được xây dựng trên nền NodeJS với framework NestJS, và cơ sở dữ liệu được triển khai bằng MySQL cùng MongoDB.

Nội dung quyển đồ án được trình bày theo quy trình phát triển phần mềm, với các chương lần lượt gồm: phân tích hệ thống, thiết kế hệ thống, triển khai và kiểm thử, kết thúc bằng phần kết luận. Các nội dung được diễn giải chi tiết kèm theo sơ đồ minh họa cụ thể.

% \newpage
% \section*{ABSTRACT}
% % \phantomsection\addcontentsline{toc}{section}{\numberline{}ABSTRACT}
% The project "Development of a Mobile Application Integrated into a Cardiovascular Data Management System, Supporting Patient-Doctor Connectivity" is an integrated system comprising Web/App/Server, leveraging Bluetooth Low Energy (BLE) technology to facilitate effective and convenient monitoring of electrocardiogram (ECG) data through a mobile application. The data collected after each measurement session is stored on the server, enabling users to review it at any time and supporting future research purposes. Furthermore, the system provides features that enhance direct connectivity between patients and doctors, including appointment scheduling, messaging for information exchange, and health data monitoring, delivering convenience and professionalism in healthcare management.

% This project focuses on completing the system following a software development lifecycle (SDLC), starting from identifying system requirements to analyzing, designing, and implementing the solution. The development process applies object-oriented analysis and design (OOAD) methodologies, utilizing Unified Modeling Language (UML) to represent action flows. From a technical perspective, the mobile application is developed for both Android and iOS platforms using the Flutter framework. The web application is built with ReactJS and AdminJS, while the server utilizes NodeJS and the NestJS framework, deployed on Amazon EC2. The system employs MySQL in combination with NoSQL (MongoDB) databases to ensure both performance and flexible data storage capabilities.

% The project documentation is structured according to the software development lifecycle, with detailed content presented in individual chapters, including system analysis, system design, implementation, testing, and finally, conclusions. All content is illustrated with diagrams and comprehensive explanations, ensuring clarity and scientific rigor.

\cleardoublepage



