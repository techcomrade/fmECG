\section*{TÓM TẮT ĐỒ ÁN}
\phantomsection\addcontentsline{toc}{section}{\numberline{}TÓM TẮT ĐỒ ÁN}

Đồ án "Hệ thống quản lý dữ liệu điện tim và tương tác giữa bệnh nhân - bác sĩ" là hệ thống bao gồm Web/App/Server 
giúp việc đặt lịch và tương tác giữa các bác sĩ, bệnh nhân trên website và ứng dụng di động một cách dễ dàng và hiệu quả. 
Tại đây thông tin người dùng, dữ liệu đặt lịch, dữ liệu đo hay các đoạn hội thoại sẽ được lưu trữ trên server để người dùng có thể xem lại bất cứ lúc nào.
Bên cạnh đó hệ thống cung cấp tính năng đặt lịch hẹn giúp bệnh nhân thoải mái chọn lịch phù hợp với thời gian của mình.
Bác sĩ sẽ đồng ý hoặc từ chối lịch, sau đó thông báo sẽ được gửi đến bệnh nhân. Nếu bác sĩ quên, sẽ có mail thông báo đến bác sĩ.
Sau khi bác sĩ đồng ý lịch hẹn, bệnh nhân và bác sĩ có thể nhắn tin để trao đổi thông tin.

Đồ án tập trung hoàn thiện theo quy trình phát triển phần mềm, 
sau khi xác định được các yêu cầu của hệ thống sẽ tiến hành phân tích, thiết kế và triển khai hệ thống. 
Quá trình này áp dụng phương pháp phân tích và thiết kế hướng đối tượng, đồng thời sử dụng ngôn ngữ UML để biểu diễn các luồng thực hiện hành động. 
Ứng dụng Web sử dụng ReactJS, và server là NodeJS, framework NestJS. Cơ sở dữ liệu sử dụng là MySQL và MongoDb.

Quyển đồ án được trình bày theo quy trình phát triển phần mềm, nội dung mỗi chương được triển khai và trình bày thông qua sơ đồ và diễn giải chi tiết. 
Các chương theo thứ tự lần lượt là phân tích hệ thống, thiết kế hệ thống, triển khai và kiểm thử, cuối cùng là kết luận.

% \cleardoublepage

\newpage
\section*{ABSTRACT}
% \phantomsection\addcontentsline{toc}{section}{\numberline{}ABSTRACT}

The project "Electrocardiogram Data Management System and Patient-Doctor Interaction" is a system comprising Web/App/Server components that facilitate scheduling 
and interaction between doctors and patients through websites and mobile applications in an easy and efficient manner.

Here, user information, appointment schedules, measurement data, and conversations are stored on the server, allowing users to review them at any time. 
Additionally, the system offers a scheduling feature that enables patients to conveniently select a suitable appointment time. Doctors can approve or decline the schedule, 
and a notification will then be sent to the patient. If the doctor forgets, a reminder email will be sent to the doctor. 
Once the doctor approves the appointment, patients and doctors can chat to exchange information.

The project focuses on completing the software development process. 
After identifying the system requirements, the analysis, design, and implementation phases are carried out. 
This process applies object-oriented analysis and design methods, while utilizing UML language to represent action execution flows. 
The web application uses ReactJS, the server is built with NodeJS and the NestJS framework, and the databases used are MySQL and MongoDB.

The project report is presented according to the software development process, with each chapter developed and detailed through diagrams and comprehensive explanations. 
The chapters are ordered as follows: system analysis, system design, implementation and testing, and finally, conclusion.

\cleardoublepage



