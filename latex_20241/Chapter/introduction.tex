
\section*{PHẦN MỞ ĐẦU}
\phantomsection\addcontentsline{toc}{section}{\numberline{} PHẦN MỞ ĐẦU}
\subsection*{Đặt vấn đề} 

Sau đại dịch Covid-19, sức khỏe đã trở thành mối quan tâm hàng đầu của người dân Việt Nam. Trong bối cảnh môi trường sống ngày càng ô nhiễm, thời gian tiếp xúc với các thiết bị điện tử tăng cao, và thói quen ít vận động thể thao trở nên phổ biến, ngày càng nhiều người bắt đầu chú ý hơn đến các tín hiệu quan trọng từ cơ thể như nhịp tim, huyết áp và nhiệt độ. Điều này càng nhấn mạnh tầm quan trọng của việc chăm sóc sức khỏe tại nhà.

Với những người muốn chủ động theo dõi tình trạng sức khỏe, các thiết bị đo lường nhỏ gọn kết hợp cùng ứng dụng di động hỗ trợ đã trở thành lựa chọn không thể thiếu. Tuy nhiên, đối với những người đang gặp vấn đề sức khỏe, việc phải đối mặt với cảnh đông đúc, chờ đợi lâu tại các bệnh viện lại là một thử thách lớn, gây ra không ít phiền toái và căng thẳng.

Vậy, liệu có một giải pháp hiệu quả nào cho phép người dùng theo dõi tình trạng sức khỏe ngay tại nhà mà vẫn nhận được sự tư vấn và hỗ trợ từ đội ngũ chuyên gia y tế một cách kịp thời và tiện lợi? Đây chính là vấn đề đặt ra, đòi hỏi một phương án tối ưu để cân bằng giữa nhu cầu chăm sóc sức khỏe cá nhân và sự hỗ trợ chuyên môn từ y tế.

\subsection*{Đề xuất hệ thống}
% \phantomsection\addcontentsline{toc}{section}{\numberline{} Đề xuất hệ thống}


Nhận thấy nhu cầu chăm sóc sức khỏe tại nhà không chỉ giới hạn ở những người gặp vấn đề về sức khỏe mà còn mở rộng đến cả những người bình thường mong muốn theo dõi tình trạng cơ thể, đồ án của chúng em đề xuất một hệ thống IoT tiên tiến để quản lý và theo dõi dữ liệu điện tim. Hệ thống này bao gồm các thành phần chính: Website, Server, Ứng dụng di động, và Thiết bị đo điện tim.
Trong phạm vi đồ án, chúng em tập trung vào phát triển Server và Website.

Khi làm việc cùng nhóm phần cứng, chúng em đã được tiếp cận thiết bị đo điện tim không tiếp xúc, từ đó định hướng xây dựng một hệ thống quản lý các thiết bị đo, thu thập và lưu trữ dữ liệu theo thời gian thực, đồng thời đáp ứng nhu cầu phân tích chuyên sâu. Hệ thống cũng tích hợp các tính năng như lập lịch hẹn và nhắn tin giữa bệnh nhân và bác sĩ, giúp nâng cao hiệu quả chăm sóc sức khỏe và tối ưu hóa trải nghiệm người dùng.

Cụ thể, hệ thống sẽ bao gồm:

\begin{adjustwidth}{1.5em}{}
  \begin{itemize}
     
      \item Một website dành cho bệnh nhân, cung cấp khả năng xem thông tin cá nhân, theo dõi lịch sử đo, quản lý và đặt lịch hẹn, đồng thời trao đổi tin nhắn trực tiếp với bác sĩ.
      \item Một website dành cho bác sĩ, hỗ trợ quản lý lịch hẹn, phản hồi yêu cầu, theo dõi kết quả đo của các bệnh nhân đã đặt lịch thành công, và trao đổi tin nhắn trực tiếp với bệnh nhân.
      \item Một website dành cho admin, cho phép quản lý toàn bộ hệ thống, bao gồm người dùng, thiết bị đo và các dữ liệu đo tương ứng.
      \item Một server để lưu cơ sở dữ liệu liên quan đến người dùng và dữ liệu đo của bệnh nhân, có thể phục vụ cho công tác nghiên cứu và phân tich dữ liệu sau này

  \end{itemize}
  \end{adjustwidth}


\subsection*{Mục tiêu của đề tài}
Sau khi đã trình bày đề xuất về một hệ thống theo dõi và quản lý dữ liệu điện tim, mục tiêu đặt ra khi thực hiện
đề tài này đó là:

\begin{adjustwidth}{1.5em}{}
  \begin{itemize}
      \item Nắm được cơ sở lý thuyết và phương pháp thiết kế một hệ thống phần mềm.
      \item Hoàn thiện Website và Server được đề ra trong mục Đề xuất hệ thống với các chức năng hoạt động ổn định.
      \item Đảm bảo hệ thống hoạt động đồng bộ với các thiết bị phần cứng đang được nghiên cứu và phát triển.
      \item Cung cấp tài liệu tham khảo đầy đủ và đảm bảo tính chính xác, trung thực.

  \end{itemize}
  \end{adjustwidth}

\subsection*{Phương pháp nghiên cứu}
Trong đồ án này, chúng em đã áp dụng kết hợp nhiều phương pháp nghiên cứu. Đầu tiên, chúng em tham khảo thông tin từ các bài báo khoa học và các sản phẩm liên quan đến thiết bị đo điện tim được phát triển trong các phòng nghiên cứu tại trường.
Đồng thời, chúng em tìm hiểu cách các hệ thống phần cứng và phần mềm kết nối và hoạt động cùng nhau. 

Sau khi nắm vững cơ sở lý thuyết, chúng em tiến hành thực nghiệm, thu thập và lưu trữ dữ liệu từ các bản ghi đo đạc, đồng thời xây dựng các biểu đồ để trực quan hóa các chỉ số sức khỏe. Để đảm bảo dữ liệu được truyền tải và xử lý chính xác, nhóm đã phối hợp với đội firmware kiểm tra tính toàn vẹn của hệ thống.
Ngoài ra, chúng em cũng tham vấn các chuyên gia y tế, đặc biệt trong lĩnh vực tim mạch, để xác thực độ chính xác của đồ thị và các kết quả thu được, đảm bảo hệ thống đáp ứng yêu cầu chuyên môn.
\subsection*{Kết quả đạt được}

Trong quá trình thực hiện đồ án, chúng em, Cồ Huy Dũng và Nguyễn Đức Dương, đã có cơ hội nghiên cứu chuyên sâu về phần cứng, hệ thống IoT và cách các thành phần trong hệ thống được kết nối và hoạt động cùng nhau.
Những kết quả đạt được tính đến thời điểm hoàn thiện quyển đồ án bao gồm:

\begin{adjustwidth}{1.5em}{}
  \begin{itemize}
      \item Hoàn thiện quyển đồ án với nội dung chi tiết về các bước xây dựng và phát triển hệ thống.
      \item Hoàn thiện các sản phẩm ứng dụng theo kế hoạch đã đề ra trong phần Đề xuất hệ thống, với khả năng kết nối các thành phần một cách hiệu quả, tích hợp theo dõi, lưu trữ dữ liệu điện tim trên server, đáp ứng tốt nhu cầu phân tích, nghiên cứu và quản lý dữ liệu trong tương lai.
      Đồng thời tích hợp các tính năng tương tác giữa bệnh nhân và bác sĩ như đặt lịch hẹn, nhắn tin trực tiếp.
      \item Nâng cao kỹ năng làm việc nhóm, viết báo cáo đồ án, phối hợp chặt chẽ với nhóm firmware và các chuyên gia y tế, góp phần hoàn thiện và nâng cao chất lượng sản phẩm.
    \end{itemize}
  \end{adjustwidth}
\subsection*{Cấu trúc đồ án}

% \phantomsection\addcontentsline{toc}{section}{\numberline{} Cấu trúc đồ án}
\begin{adjustwidth}{1.5em}{}
\begin{itemize}
  \item Giới thiệu mục đích của đồ án, đề xuất hệ thống, phân tích tính khả thi và trình bày bố cục nội dung.
  \item Chương 1: Thu thập và xác định yêu cầu hệ thống, bao gồm kỹ thuật thu thập thông tin, xác định yêu cầu chi tiết và thiết kế các sơ đồ các trường hợp sử dụng của hệ thống.
  \item Chương 2: Phân tích hệ thống, tập trung vào mô tả thẻ CRC, thiết kế sơ đồ lớp và sơ đồ tuần tự.
  \item Chương 3: Thiết kế hệ thống, bao gồm sơ đồ kiến trúc tổng thể, sơ đồ khối phần mềm, thiết kế cơ sở dữ liệu, giao diện người dùng, sơ đồ lớp và các chức năng hệ thống.
  \item Chương 4: Triển khai và kiểm thử, trình bày quá trình cài đặt hệ thống và kiểm tra tính năng.
  \item Phần kết luận: Kết luận và đề xuất hướng phát triển trong tương lai.
\end{itemize}
\end{adjustwidth}

\cleardoublepage

% \pagenumbering{arabic}