
\section*{CHƯƠNG 4. TRIỂN KHAI VÀ KIỂM THỬ}
\setcounter{section}{4}
\setcounter{subsection}{0} %LƯU Ý MỖI LẦN THÊM CHƯƠNG MỚI CẦN THÊM CÂU NÀY ĐỂ RESET THỨ TỰ CỦA SUBSECTON VỀ 1
\setcounter{table}{0} % LƯU Ý SAU MỖI LẦN GỌI BẢNG HAY HÌNH ẢNH PHẢI THÊM CÂU NÀY ĐỂ RESET THỨ TỰ
\setcounter{figure}{0} %% LƯU Ý SAU MỖI LẦN GỌI BẢNG HAY HÌNH ẢNH PHẢI THÊM CÂU NÀY ĐỂ RESET THỨ TỰ
\addcontentsline{toc}{section}{\numberline{}CHƯƠNG 4. Xây Dựng VÀ KIỂM THỬ}

\subsection{Công nghệ sử dụng}
\subsubsection{Thiết kế giao diện website}
\paragraph{ReactJS}
\mbox{}

ReactJS (thường được gọi tắt là React) là một thư viện JavaScript mã nguồn mở được phát triển bởi Facebook (nay là Meta) \cite{reactjs}. Chúng em ưu tiên sử dụng React để xây dựng giao diện người dùng (UI) bởi sự hiệu quả và linh hoạt của nó. React tập trung vào việc chia nhỏ giao diện thành các thành phần (components) độc lập, giúp quản lý và tái sử dụng code dễ dàng hơn.

Một số đặc điểm nổi bật của React chúng em đã sử dụng trong đồ án:
\begin{itemize}
	\item Với các Components có tính lặp lại, chúng em tách chúng thành các thành phần độc lập để tái sử dụng, giúp code gọn gàng và hiệu quả hơn.
	\item Hiệu suất cao, tối ưu việc cập nhật giao diện giúp phát triển giao diện nhanh chóng
	\item Dễ dàng debug: Facebook đã phát hành một Chrome extension hỗ trợ debug trong quá trình phát triển ứng dụng React.
	\item Tương thích với nhiều trình duyệt như Chrome, Edge, FireFox, Cốc Cốc,...
\end{itemize}

\begin{figure}[H]
	\centering
	\includegraphics[width=15cm,height=5.4cm]{Images/Technology/react_logo.png}
	\caption[Logo ReactJs]{\bfseries \fontsize{12pt}{0pt}
		\selectfont Logo ReactJs}
	\label{reactjs_cover} %đặt tên cho ảnh
\end{figure}

\subsubsection{Server}
\paragraph{TypeScript}
\mbox{}

TypeScript là một ngôn ngữ lập trình mã nguồn mở được phát triển và duy trì bởi Microsoft. TypeScript là một công cụ mạnh mẽ giúp nâng cao chất lượng và hiệu suất của việc phát triển ứng dụng JavaScript.

\begin{figure}[H]
	\centering
	\includegraphics[width=7.5cm,height=4cm]{Images/Technology/typescript.png}
	\caption[Logo TypeScript]{\bfseries \fontsize{12pt}{0pt}
		\selectfont Logo TypeScript}
	\label{typescript} %đặt tên cho ảnh
\end{figure}

TypeScript mang lại nhiều lợi ích cho việc phát triển server của chúng em, bao gồm:
\begin{itemize}
	\item Kiểu tĩnh giúp phát hiện lỗi ngay trong quá trình viết code, giúp tiết kiệm thời gian kiểm thử và sửa lỗi.
	\item Code dễ đọc và dễ bảo trì hơn với kiểu dữ liệu rõ ràng, giúp tăng hiệu suất và chất lượng code.
	\item Công cụ và thư viện hỗ trợ mạnh mẽ như NestJS \cite{nestjs}, giúp phát triển ứng dụng nhanh chóng và hiệu quả.
	\item Hỗ trợ OOP giúp chúng em xây dựng ứng dụng theo mô hình hướng đối tượng, dễ dàng mở rộng và bảo trì.
\end{itemize}

\paragraph{NodeJS}
\mbox{}

NodeJS (hay Node.js) là một môi trường chạy (runtime environment) cho JavaScript \cite{nodejs}, cho phép bạn thực thi mã JavaScript bên ngoài trình duyệt web. Điều này giúp chúng em có thể sử dụng JavaScript để viết các ứng dụng phía máy chủ (back-end), các công cụ dòng lệnh, và nhiều thứ khác, thay vì chỉ giới hạn trong việc viết mã chạy trên trình duyệt (front-end).

\begin{figure}[H]
	\centering
	\includegraphics[width=7.5cm,height=4cm]{Images/Technology/nodejs.png}
	\caption[Logo NodeJS]{\bfseries \fontsize{12pt}{0pt}
		\selectfont Logo NodeJS}
	\label{nodejs} %đặt tên cho ảnh
\end{figure}

Một số đặc điểm chính của NodeJS:
\begin{itemize}
	\item Dựa trên V8 JavaScript Engine: NodeJS được xây dựng dựa trên V8, engine JavaScript cực kỳ mạnh mẽ của Google Chrome. Điều này giúp NodeJS thực thi mã JavaScript rất nhanh.
	\item Mã nguồn mở và đa nền tảng: NodeJS là dự án mã nguồn mở, được phát triển và duy trì bởi một cộng đồng lớn. Nó cũng chạy được trên nhiều hệ điều hành khác nhau như Windows, macOS và Linux.
	\item Mô hình Non-blocking I/O: NodeJS sử dụng mô hình vào/ra không đồng bộ (non-blocking I/O), cho phép nó xử lý nhiều yêu cầu đồng thời một cách hiệu quả mà không bị tắc nghẽn.
	\item npm (Node Package Manager): NodeJS đi kèm với npm, một hệ thống quản lý gói mạnh mẽ, cho phép chúng em dễ dàng cài đặt, quản lý và chia sẻ các thư viện và module JavaScript hay TypeScript.
\end{itemize}

\paragraph{MySQL}
\mbox{}

MySQL là một hệ quản trị cơ sở dữ liệu quan hệ (RDBMS) với mã nguồn mở phổ biến \cite{mysql}, được phát triển và duy trì bởi Oracle Corporation. Nó sử dụng Ngôn ngữ truy vấn cấu trúc (SQL) để quản lý và thao tác dữ liệu. MySQL lưu trữ dữ liệu trong các bảng, được tổ chức thành các cơ sở dữ liệu. Các bảng có thể liên kết với nhau thông qua các khóa, tạo thành mối quan hệ giữa các dữ liệu.

MySQL nổi tiếng với tốc độ, độ tin cậy và tính dễ sử dụng. Nó được sử dụng rộng rãi trong các ứng dụng web, ứng dụng doanh nghiệp và các hệ thống nhúng. MySQL tương thích với nhiều hệ điều hành và nhiều ngôn ngữ lập trình, như PHP, Java, Python và C++.

MySQL là một phần mềm mã nguồn mở, cho phép người dùng sử dụng, phân phối và sửa đổi mã nguồn một cách tự do theo các điều khoản của Giấy phép Công cộng GNU. Điều này đã góp phần vào sự phổ biến và phát triển của MySQL trong cộng đồng các nhà phát triển. Nó cung cấp nhiều tính năng mạnh mẽ, bao gồm bảo mật dữ liệu, sao lưu và phục hồi, và khả năng mở rộng.

\begin{figure}[H]
	\centering
	\includegraphics[width=7.5cm,height=5cm]{Images/Technology/mysql.png}
	\caption[Logo MySQL]{\bfseries \fontsize{12pt}{0pt}
		\selectfont Logo MySQL}
	\label{mysql} %đặt tên cho ảnh
\end{figure}

\paragraph{MongoDB}
\mbox{}

MongoDB là một hệ quản trị cơ sở dữ liệu NoSQL mã nguồn mở, được thiết kế để xử lý dữ liệu lớn, dữ liệu phi cấu trúc và bán cấu trúc một cách hiệu quả. Thay vì sử dụng các bảng và hàng như trong cơ sở dữ liệu quan hệ (SQL), MongoDB sử dụng các tài liệu (documents) theo định dạng JSON (JavaScript Object Notation) để lưu trữ dữ liệu. Điều này mang lại sự linh hoạt cao trong việc xử lý các loại dữ liệu khác nhau. MongoDB được phát triển và duy trì bởi MongoDB Inc.

Đặc điểm nổi bật của MongoDB là khả năng mở rộng linh hoạt theo chiều ngang (horizontal scaling) thông qua sharding, cho phép phân tán dữ liệu trên nhiều máy chủ, giúp xử lý lượng dữ liệu lớn và tăng hiệu suất. MongoDB thường được sử dụng trong các ứng dụng web hiện đại, ứng dụng di động, phân tích dữ liệu lớn và các ứng dụng yêu cầu tốc độ và khả năng mở rộng cao \cite{mongodb}.

\begin{figure}[H]
	\centering
	\includegraphics[width=8cm,height=4cm]{Images/Technology/mongo.png}
	\caption[Logo MongoDB]{\bfseries \fontsize{12pt}{0pt}
		\selectfont Logo MongoDB}
	\label{mongodb} %đặt tên cho ảnh
\end{figure}

\paragraph{Postman}
\mbox{}

Postman là một nền tảng mạnh mẽ giúp đơn giản hóa quá trình phát triển và kiểm thử API. Thay vì phải viết code phức tạp để tương tác với API, Postman cung cấp một giao diện trực quan, cho phép người dùng dễ dàng gửi các yêu cầu HTTP (GET, POST, PUT, DELETE,...) với đầy đủ các tùy chỉnh về headers, body, parameters. Không chỉ dừng lại ở việc gửi yêu cầu, Postman còn hỗ trợ tổ chức API thành các bộ sưu tập (Collections), tự động hóa kiểm thử bằng JavaScript, giả lập server (Mock Server) và theo dõi hiệu suất API. Với những tính năng này, Postman trở thành một "trợ thủ đắc lực" cho cả lập trình viên và kiểm thử viên \cite{postman}.

Trong quá trình phát triển phần mềm hiện đại, API đóng vai trò trung tâm trong việc kết nối các ứng dụng. Việc kiểm thử API một cách hiệu quả là vô cùng quan trọng để đảm bảo chất lượng và tính ổn định của hệ thống. Postman ra đời để giải quyết bài toán này. Nó giúp người dùng dễ dàng tạo và kiểm thử các API, tiết kiệm thời gian và công sức so với việc viết code thủ công. Khả năng tự động hóa kiểm thử của Postman giúp phát hiện sớm các lỗi tiềm ẩn, trong khi tính năng mock server cho phép tiếp tục phát triển ngay cả khi API chưa hoàn thiện.

\begin{figure}[H]
	\centering
	\includegraphics[width=8cm,height=4cm]{Images/Technology/postman.jpg}
	\caption[Logo Postman]{\bfseries \fontsize{12pt}{0pt}
		\selectfont Logo Postman}
	\label{postman} %đặt tên cho ảnh
\end{figure}


% \paragraph{Docker}
% \mbox{}

% Trước Docker, việc triển khai ứng dụng thường gặp nhiều khó khăn do sự khác biệt về môi trường giữa các máy tính. Ví dụ: một ứng dụng chạy tốt trên máy tính của nhà phát triển có thể gặp lỗi khi triển khai lên máy chủ do thiếu thư viện, phiên bản phần mềm khác nhau, hoặc cấu hình không tương thích. Docker giải quyết vấn đề này bằng cách đóng gói tất cả các thành phần cần thiết cho ứng dụng vào một container, đảm bảo ứng dụng luôn chạy như nhau ở mọi nơi. Các từ khóa chính cầm quan tâm như: Image, container, DockerFile, Docker Compose, ... \cite{docker}

% Những lợi ích của Docker có thể kể đến như:
% \begin{itemize}
% 	\item Tính nhất quán: đảm bảo ứng dụng chạy trên mọi môi trường
% 	\item Tính di động: dễ dàng di chuyển ứng dụng giữa các máy tính, máy chủ, và đám mây.
% 	\item Tiết kiệm tài nguyên: Container nhẹ hơn nhiều so với máy ảo (Virtual Machine) vì chúng chia sẻ kernel của hệ điều hành host.
% 	\item Tăng tốc độ phát triển và triển khai: Docker giúp tự động hóa quá trình xây dựng, kiểm thử, và triển khai ứng dụng.
% 	\item Dễ dàng quản lý phiên bản: Docker giúp quản lý các phiên bản khác nhau của ứng dụng một cách dễ dàng.
% 	\item Cách ly ứng dụng: Container cung cấp môi trường cách ly, giúp ngăn chặn xung đột giữa các ứng dụng.
% \end{itemize}

% \begin{figure}[H]
% 	\centering
% 	\includegraphics[width=8cm,height=4cm]{Images/Technology/docker.jpg}
% 	\caption[Logo của Docker]{\bfseries \fontsize{12pt}{0pt}
% 		\selectfont Logo của Docker}
% 	\label{docker} %đặt tên cho ảnh
% \end{figure}


\subsection{Xây dựng ứng dụng}
 Về quản lý cơ sở dữ liệu, chúng em sử dụng MySQL Server và MongoDB Compass, đảm bảo hiệu suất và tính linh hoạt. Ngôn ngữ sử dụng TypeScript, hỗ trợ bởi framework và thư viện như ReactJS, NestJS. Công cụ kiểm thử API bao gồm Postman các trình duyệt như Cốc Cốc, Edge, Chrome. Bên cạnh đó toàn bộ mã nguồn dự án được quản lý trên GitHub, giúp theo dõi và kiểm soát phiên bản hiệu quả.
% \subsubsection{Quy trình CI/CD}
% CI/CD là một tập hợp các phương pháp và thực hành giúp tự động hóa quy trình phát triển, kiểm thử và triển khai phần mềm \cite{cicd}. CI (Continuous Integration - Tích hợp liên tục) tập trung vào việc thường xuyên tích hợp mã nguồn từ nhiều nhà phát triển vào một kho lưu trữ chung, sau đó tự động xây dựng và kiểm tra phần mềm. Việc này giúp phát hiện sớm các lỗi tích hợp và giảm thiểu xung đột giữa các thay đổi.

% CD (Continuous Delivery/Continuous Deployment - Chuyển giao/Triển khai liên tục) mở rộng CI bằng cách tự động hóa quá trình chuyển giao phần mềm đã được kiểm tra đến môi trường thử nghiệm hoặc sản xuất. Continuous Delivery cho phép triển khai thủ công sau khi kiểm tra, trong khi Continuous Deployment tự động triển khai lên môi trường sản xuất. CI/CD giúp tăng tốc độ phát hành phần mềm, cải thiện chất lượng và giảm thiểu rủi ro.
% \begin{figure}[H]
% 	\centering
% 	\includegraphics[width=15cm,height=8cm]{Images/Technology/cicd.png}
% 	\caption[Quy trình CI/CD]{\bfseries \fontsize{12pt}{0pt}
% 		\selectfont Quy trình CI/CD}
% 	\label{cicd} %đặt tên cho ảnh
% \end{figure}

\subsubsection{Kiến trúc Microservices}
Kiến trúc Microservices là một cách tiếp cận trong quy trình phát triển phần mềm, nơi một ứng dụng lớn được tách thành nhiều dịch vụ nhỏ hoạt động độc lập với nhau \cite{microservice}.
Mỗi dịch vụ thực hiện một chức năng cụ thể và có thể được phát triển, triển khai, mở rộng và bảo trì một cách độc lập với các microservice khác. Chúng thường giao tiếp với nhau thông qua mạng, sử dụng các giao thức như HTTP/REST hoặc message queues.

\begin{figure}[H]
	\centering
	\includegraphics[width=15cm,height=8cm]{Images/Technology/microservice.png}
	\caption[Kiến trúc microservices]{\bfseries \fontsize{12pt}{0pt}
		\selectfont Kiến trúc microservices}
	\label{microservice} %đặt tên cho ảnh
\end{figure}

Một số lợi ích của microservice đã hỗ trợ chúng em:
\begin{itemize}
	\item Tính độc lập: Mỗi microservice là một đơn vị triển khai độc lập. Việc thay đổi một microservice không ảnh hưởng đến các microservice khác, giúp tăng tính linh hoạt và tốc độ phát triển.
	\item Tính chuyên biệt: Mỗi microservice tập trung vào một chức năng nghiệp vụ cụ thể, giúp đơn giản hóa việc phát triển và bảo trì.
	\item Khả năng mở rộng: Có thể mở rộng từng microservice một cách độc lập dựa trên nhu cầu, giúp tối ưu hóa việc sử dụng tài nguyên.
	\item Khả năng chịu lỗi: Nếu một microservice gặp sự cố, các microservice khác vẫn có thể hoạt động bình thường, giúp tăng tính ổn định của hệ thống.
\end{itemize}

% \subsubsection{Triển khai Server và ứng dụng web trên máy chủ VPS}
\subsection{Kiểm thử}

\subsubsection{Đánh giá và kiểm tra hoạt động các API}


% Môi trường: 

% \begin{adjustwidth}{1.5em}{}
% \begin{itemize}
%   \item Base URL: http://103.200.20.59/ hoặc http://localhost:3000/
% \end{itemize}
% \end{adjustwidth}

Công cụ: Postman - Dùng để thiết lập và kiểm thử các yêu cầu API.

\paragraph{API liên quan đến việc xác thực người dùng}
\mbox{}

% Tham khảo bảng \ref{table_api_auth} để xem thông tin của các api liên quan


\begin{enumerate}[a)]
	\item URL: POST auth/register
	      \begin{xltabular}{\textwidth}{
		      | >{\raggedright\arraybackslash}p{1cm}
		      | >{\raggedright\arraybackslash}p{2.5cm}
		      | >{\raggedright\arraybackslash}X
		      | >{\raggedright\arraybackslash}X
		      | >{\raggedright\arraybackslash}p{1cm}|
		      }
		      \caption{\bfseries \fontsize{12pt}{0pt}\selectfont Bảng kiểm thử API đăng ký tài khoản}
		      % \label{table_api_news}
		      \\
		      \hline
		      \bfseries Test case    &\bfseries Điều kiện   &\bfseries Đầu vào
		      &\bfseries Mong muốn đầu ra &\bfseries Kết quả\\ \hline


		      TC-1
		      & Người dùng mới
		      & Dữ liệu đăng ký

		      \{

		      "username": "Co Huy Dung",

		      "password": "123456a@",

		      "email": "codung2909@gmail.com",

		      "birth": "29/09/2003",

		      "gender": 1,

		      "phone\_number": "077433306x",

		      "role": 0

		      \}

		      &

		      Status code: 200 OK

		      Response message:

		      \{

		      "status": "success",

		      "message": "register successfully"

		      \}

		      & OK

		      \\ \hline

		      TC-2
		      & Tài khoản email đã tồn tại
		      & Dữ liệu đăng ký

		      \{

		      "username": "Nguyễn Đức B",

		      "password": "xYnF1452",

		      "email": "codung2909@gmail.com",

		      "birth": "15/07/1995",

		      "gender": 1,

		      "phone\_number": "091601736x",

		      "role": 0

		      \}

		      &

		      Status code: 400 Bad Request

		      Response message:

		      \{

		      "status": "error",

		      "message": "email is existed"

		      \}

		      & OK

		      \\ \hline


	      \end{xltabular}


	\item URL: POST auth/login
	      \begin{xltabular}{\textwidth}{
		      | >{\raggedright\arraybackslash}p{1cm}
		      | >{\raggedright\arraybackslash}p{2.5cm}
		      | >{\raggedright\arraybackslash}X
		      | >{\raggedright\arraybackslash}X
		      | >{\raggedright\arraybackslash}p{1cm}|
		      }
		      \caption{\bfseries \fontsize{12pt}{0pt}\selectfont Bảng kiểm thử API người dùng đăng nhập}
		      % \label{table_api_news}
		      \\
		      \hline
		      \bfseries Test case    &\bfseries Điều kiện   &\bfseries Đầu vào
		      &\bfseries Mong muốn đầu ra &\bfseries Kết quả\\ \hline


		      TC-1
		      & Thông tin tài khoản và mật khẩu hợp lệ
		      & Thông tin đăng nhập

		      \{

		      "email": email đã đăng ký,
		      "password": mật khẩu tương ứng

		      \}

		      &

		      Status code: 200 OK

		      Response message:

		      \{

		      "status": "success",

		      data: Thông tin user sau khi đăng ký thành công

		      \}

		      & OK

		      \\ \hline

		      TC-2
		      & Thông tin tài khoản và mật khẩu không hợp lệ
		      & Thông tin đăng nhập

		      \{

		      "email": email người dùng,
		      "password": mật khẩu người dùng

		      \}

		      &

		      Status code: 401 Unauthorized

		      Response message:

		      \{

		      "status": "401 Unauthorized",

		      "message": "Invalid email or password"

		      \}

		      & OK

		      \\ \hline


	      \end{xltabular}



	\item URL: POST auth/logout


	      \begin{xltabular}{\textwidth}{
		      | >{\raggedright\arraybackslash}p{1cm}
		      | >{\raggedright\arraybackslash}p{2.5cm}
		      | >{\raggedright\arraybackslash}X
		      | >{\raggedright\arraybackslash}X
		      | >{\raggedright\arraybackslash}p{1cm}|
		      }
		      \caption{\bfseries \fontsize{12pt}{0pt}\selectfont Bảng kiểm thử API đăng xuất người dùng}
		      % \label{table_api_news}
		      \\
		      \hline
		      \bfseries Test case    &\bfseries Điều kiện   &\bfseries Đầu vào
		      &\bfseries Mong muốn đầu ra &\bfseries Kết quả\\ \hline


		      TC-1
		      & Người dùng đã đăng nhập vào hệ thống
		      & JWT Token tồn tại

		      &

		      Status code: 200 OK

		      Response message:

		      \{

		      "status": "success",

		      "message": "Logged out successfully"

		      \}

		      & OK

		      \\ \hline

		      TC-2
		      & Người dùng chưa đăng nhập vào hệ thống
		      & JWT Token không tồn tại

		      &

		      Status code: 401 Unauthorized

		      Response message:

		      \{

		      "status": "error",

		      "message": "No token found"

		      \}

		      & OK

		      \\ \hline


	      \end{xltabular}



\end{enumerate}



\paragraph{API liên quan đến thông tin người dùng}
\mbox{}

% Tham khảo bảng \ref{table_api_user} để xem thông tin của các api liên quan

\begin{enumerate}[a)]
	\item URL: GET users

	      \begin{xltabular}{\textwidth}{
		      | >{\raggedright\arraybackslash}p{1cm}
		      | >{\raggedright\arraybackslash}p{2.5cm}
		      | >{\raggedright\arraybackslash}X
		      | >{\raggedright\arraybackslash}X
		      | >{\raggedright\arraybackslash}p{1cm}|
		      }
		      \caption{\bfseries \fontsize{12pt}{0pt}\selectfont Bảng kiểm thử API lấy danh sách người dùng}
		      % \label{table_api_news}
		      \\
		      \hline
		      \bfseries Test case    &\bfseries Điều kiện   &\bfseries Đầu vào
		      &\bfseries Mong muốn đầu ra &\bfseries Kết quả\\ \hline


		      TC-1
		      & Quản trị viên của hệ thống
		      & Access token tương ứng

		      &

		      Status code: 200

		      Response message:

		      \{

		      "data": Danh sách người dùng

		      \}

		      & OK

		      \\ \hline

		      TC-2
		      & Không phải quản trị viên
		      & Access token tương ứng
		      &

		      Status code: 403

		      Response message:

		      \{

		      "status": "error",

		      "message": "403 Forbidden"

		      \}

		      & OK

		      \\ \hline


		      TC-3
		      & Không có token
		      & NULL

		      &

		      Status code: 401

		      Response message:

		      \{

		      "status": "error",

		      "message": "401 Unauthorized"

		      \}

		      & OK

		      \\ \hline


	      \end{xltabular}

	\item URL: GET users/doctors

	      \begin{xltabular}{\textwidth}{
		      | >{\raggedright\arraybackslash}p{1cm}
		      | >{\raggedright\arraybackslash}p{2.5cm}
		      | >{\raggedright\arraybackslash}X
		      | >{\raggedright\arraybackslash}X
		      | >{\raggedright\arraybackslash}p{1cm}|
		      }
		      \caption{\bfseries \fontsize{12pt}{0pt}\selectfont Bảng kiểm thử API lấy danh sách bác sĩ}
		      % \label{table_api_news}
		      \\
		      \hline
		      \bfseries Test case    &\bfseries Điều kiện   &\bfseries Đầu vào
		      &\bfseries Mong muốn đầu ra &\bfseries Kết quả\\ \hline


		      TC-1
		      & Quản trị viên, bác sĩ của hệ thống
		      & Access token tương ứng

		      &

		      Status code: 200

		      Response message:

		      \{

		      "data": Danh sách bác sĩ

		      \}

		      & OK

		      \\ \hline

		      TC-2
		      & Không phải quản trị viên, bác sĩ
		      & Access token tương ứng
		      &

		      Status code: 403

		      Response message:

		      \{

		      "status": "error",

		      "message": "403 Forbidden"

		      \}

		      & OK

		      \\ \hline


		      TC-3
		      & Không có token
		      & NULL

		      &

		      Status code: 401

		      Response message:

		      \{

		      "status": "error",

		      "message": "401 Unauthorized"

		      \}

		      & OK

		      \\ \hline


	      \end{xltabular}

	\item URL: GET users/{:id}
	      \begin{xltabular}{\textwidth}{
		      | >{\raggedright\arraybackslash}p{1cm}
		      | >{\raggedright\arraybackslash}p{2.5cm}
		      | >{\raggedright\arraybackslash}X
		      | >{\raggedright\arraybackslash}X
		      | >{\raggedright\arraybackslash}p{1cm}|
		      }
		      \caption{\bfseries \fontsize{12pt}{0pt}\selectfont Bảng kiểm thử API lấy dữ liệu người dùng bằng id}
		      \\
		      \hline
		      \bfseries Test case    &\bfseries Điều kiện   &\bfseries Đầu vào
		      &\bfseries Mong muốn đầu ra &\bfseries Kết quả\\ \hline


		      TC-1
		      & Người dùng tồn tại trong hệ thống và có ID tương ứng
		      & ID người dùng

		      &

		      Status code: 200 OK

		      Response message:

		      \{

		      "data": Thông tin người dùng

		      \}

		      & OK

		      \\ \hline

		      TC-2
		      & Người dùng không tồn tại trong hệ thống với ID tương ứng
		      & ID người dùng

		      &

		      Status code: 404 Not Found

		      Response message:

		      \{

		      "status": "error",

		      "message": "No user found, please try again"

		      \}

		      & OK

		      \\ \hline


	      \end{xltabular}

	\item URL: GET users/data/patient-data
		  \clearpage
	      \begin{xltabular}{\textwidth}{
		      | >{\raggedright\arraybackslash}p{1cm}
		      | >{\raggedright\arraybackslash}p{2.5cm}
		      | >{\raggedright\arraybackslash}X
		      | >{\raggedright\arraybackslash}X
		      | >{\raggedright\arraybackslash}p{1cm}|
		      }
		      \caption{\bfseries \fontsize{12pt}{0pt}\selectfont Bảng kiểm thử API lấy danh sách bệnh nhân theo ID của bác sĩ}
		      % \label{table_api_news}
		      \\
		      \hline
		      \bfseries Test case    &\bfseries Điều kiện   &\bfseries Đầu vào
		      &\bfseries Mong muốn đầu ra &\bfseries Kết quả\\ \hline


		      TC-1
		      & Quản trị viên, bác sĩ tương ứng với id
		      & ID của bác sĩ

		      &

		      Status code: 200

		      Response message:

		      \{

		      "data": Danh sách bệnh nhân

		      \}

		      & OK

		      \\ \hline

		      TC-2
		      & Không phải quản trị viên, bác sĩ
		      & ID của bác sĩ
		      &

		      Status code: 403

		      Response message:

		      \{

		      "status": "error",

		      "message": "403 Forbidden"

		      \}

		      & OK

		      \\ \hline


		      TC-3
		      & Không có token
		      &

		      &

		      Status code: 401

		      Response message:

		      \{

		      "status": "error",

		      "message": "401 Unauthorized"

		      \}

		      & OK

		      \\ \hline


	      \end{xltabular}

	\item URL: GET users/data/doctor-id

	      \begin{xltabular}{\textwidth}{
		      | >{\raggedright\arraybackslash}p{1cm}
		      | >{\raggedright\arraybackslash}p{2.5cm}
		      | >{\raggedright\arraybackslash}X
		      | >{\raggedright\arraybackslash}X
		      | >{\raggedright\arraybackslash}p{1cm}|
		      }
		      \caption{\bfseries \fontsize{12pt}{0pt}\selectfont Bảng kiểm thử API lấy danh sách bác sĩ theo ID của bệnh nhân}
		      \\
		      \hline
		      \bfseries Test case    &\bfseries Điều kiện   &\bfseries Đầu vào
		      &\bfseries Mong muốn đầu ra &\bfseries Kết quả\\ \hline


		      TC-1
		      & Quản trị viên, bác sĩ hoặc bệnh nhân có ID tương ứng
		      & ID của bệnh nhân

		      &

		      Status code: 200

		      Response message:

		      \{

		      "data": Danh sách bác sĩ của bệnh nhân

		      \}

		      & OK

		      \\ \hline

		      TC-2
		      & Bệnh nhân chưa có bác sĩ
		      & ID của bệnh nhân

		      &

		      Status code: 200

		      Response message:

		      \{

		      "status": "error",

		      "message": "No user found, please try again"

		      \}

		      & OK

		      \\ \hline


		      TC-3
		      & Không phải quản trị viên, bác sĩ hoặc bệnh nhân
		      & ID của bệnh nhân
		      &

		      Status code: 403

		      Response message:

		      \{

		      "status": "error",

		      "message": "403 Forbidden"

		      \}

		      & OK

		      \\ \hline

		      TC-4
		      & Không có token
		      &

		      &

		      Status code: 401

		      Response message:

		      \{

		      "status": "error",

		      "message": "401 Unauthorized"

		      \}

		      & OK

		      \\ \hline


	      \end{xltabular}


	\item URL: PUT users

	      \begin{xltabular}{\textwidth}{
		      | >{\raggedright\arraybackslash}p{1cm}
		      | >{\raggedright\arraybackslash}p{2.5cm}
		      | >{\raggedright\arraybackslash}X
		      | >{\raggedright\arraybackslash}X
		      | >{\raggedright\arraybackslash}p{1cm}|
		      }
		      \caption{\bfseries \fontsize{12pt}{0pt}\selectfont Bảng kiểm thử API cập nhật thông tin người dùng}
		      \\
		      \hline
		      \bfseries Test case    &\bfseries Điều kiện   &\bfseries Đầu vào
		      &\bfseries Mong muốn đầu ra &\bfseries Kết quả\\ \hline


		      TC-1
		      & Người dùng đã tồn tại trong hệ thống với ID cho trước
		      & Dữ liệu cập nhật của người dùng
		      &

		      Status code: 200 OK

		      Response message:

		      \{

		      "data": Dữ liệu người dùng đã được cập nhật

		      \}

		      & OK

		      \\ \hline

		      TC-2
		      & Người dùng không tồn tại trong hệ thống với ID cho trước
		      &
		      &

		      Status code: 404 Not found

		      Response message:

		      \{

		      "message": "No user found to update, please try again"

		      \}

		      & OK

		      \\ \hline

		      TC-3
		      & Người dùng không có quyền sửa thông tin
		      &
		      &

		      Status code: 403 Forbidden

		      Response message:

		      \{

		      "message": "403 Forbidden"

		      \}

		      & OK

		      \\ \hline

	      \end{xltabular}


	      % TODO: Note new page
	\item URL: DELETE users/{:userId}

	      \begin{xltabular}{\textwidth}{
		      | >{\raggedright\arraybackslash}p{1cm}
		      | >{\raggedright\arraybackslash}p{2.5cm}
		      | >{\raggedright\arraybackslash}X
		      | >{\raggedright\arraybackslash}X
		      | >{\raggedright\arraybackslash}p{1cm}|
		      }
		      \caption{\bfseries \fontsize{12pt}{0pt}\selectfont Bảng kiểm thử API xóa thông tin người dùng}
		      \\
		      \hline
		      \bfseries Test case    &\bfseries Điều kiện   &\bfseries Đầu vào
		      &\bfseries Mong muốn đầu ra &\bfseries Kết quả\\ \hline


		      TC-1
		      & Người dùng đã tồn tại trong hệ thống với ID cho trước
		      & ID của người dùng

		      &

		      Status code: 200 OK

		      Response message:

		      \{

		      "message": "User has been deleted successfully"

		      \}

		      & OK

		      \\ \hline

		      TC-2
		      & Người dùng không tồn tại trong hệ thống với ID cho trước
		      & ID của người dùng

		      &

		      Status code: 404 Not found

		      Response message:

		      \{

		      "message": "No user found to delete, please try again"

		      \}

		      & OK
		      \\ \hline

	      \end{xltabular}

	\item URL: GET statistic
	      \begin{xltabular}{\textwidth}{
		      | >{\raggedright\arraybackslash}p{1cm}
		      | >{\raggedright\arraybackslash}p{2.5cm}
		      | >{\raggedright\arraybackslash}X
		      | >{\raggedright\arraybackslash}X
		      | >{\raggedright\arraybackslash}p{1cm}|
		      }
		      \caption{\bfseries \fontsize{12pt}{0pt}\selectfont Bảng kiểm thử API lấy dữ liệu thống kê}
		      \\
		      \hline
		      \bfseries Test case    &\bfseries Điều kiện   &\bfseries Đầu vào
		      &\bfseries Mong muốn đầu ra &\bfseries Kết quả\\ \hline


		      TC-1
		      & Quản trị viên của hệ thống
		      & Access token tương ứng

		      &

		      Status code: 200 OK

		      Response message:

		      \{

		      data: dữ liệu thống kê số lượng người dùng, thiết bị, dữ liệu phiên đo mỗi tháng

		      \}

		      & OK

		      \\ \hline

		      TC-2
		      & Không phải quản trị viên của hệ thống
		      & Access token tương ứng

		      &

		      Status code: 403 Forbidden

		      Response message:

		      \{

		      "message": "403 Forbidden"

		      \}

		      & OK
		      \\ \hline

	      \end{xltabular}


\end{enumerate}


\paragraph{API liên quan đến thiết bị}
\mbox{}

% Tham khảo bảng \ref{table_api_device} để xem thông tin của các api liên quan

\begin{enumerate}[a)]
	\item URL: GET device
	      \begin{xltabular}{\textwidth}{
		      | >{\raggedright\arraybackslash}p{1cm}
		      | >{\raggedright\arraybackslash}p{2.5cm}
		      | >{\raggedright\arraybackslash}X
		      | >{\raggedright\arraybackslash}X
		      | >{\raggedright\arraybackslash}p{1cm}|
		      }
		      \caption{\bfseries \fontsize{12pt}{0pt}\selectfont Bảng kiểm thử API lấy danh sách thiết bị}
		      % \label{table_api_news}
		      \\
		      \hline
		      \bfseries Test case    &\bfseries Điều kiện   &\bfseries Đầu vào
		      &\bfseries Mong muốn đầu ra &\bfseries Kết quả\\ \hline


		      TC-1
		      & Quản trị viên của hệ thống
		      & Access token tương ứng
		      &

		      Status code: 200 OK

		      Response message:

		      \{

		      "data": Danh sách thiết bị

		      \}
		      & OK

		      \\ \hline

		      TC-2
		      & Không phải quản trị viên của hệ thống
		      & Access token tương ứng

		      &

		      Status code: 403 Forbidden

		      Response message:

		      \{

		      "message": "403 Forbidden"

		      \}
		      & OK
		      \\ \hline


	      \end{xltabular}


	\item URL: GET device/{:id}
	      \begin{xltabular}{\textwidth}{
		      | >{\raggedright\arraybackslash}p{1cm}
		      | >{\raggedright\arraybackslash}p{2.5cm}
		      | >{\raggedright\arraybackslash}X
		      | >{\raggedright\arraybackslash}X
		      | >{\raggedright\arraybackslash}p{1cm}|
		      }
		      \caption{\bfseries \fontsize{12pt}{0pt}\selectfont Bảng kiểm thử API lấy dữ liệu thiết bị theo id}
		      % \label{table_api_news}
		      \\
		      \hline
		      \bfseries Test case    &\bfseries Điều kiện   &\bfseries Đầu vào
		      &\bfseries Mong muốn đầu ra &\bfseries Kết quả\\ \hline


		      TC-1
		      & Thiết bị tồn tại với ID cho trước
		      & ID thiết bị
		      &

		      Status code: 200 OK

		      Response message:

		      \{

		      data: Thông tin của thiết bị

		      \}
		      & OK

		      \\ \hline

		      TC-2
		      & Thiết bị không tồn tại với ID cho trước
		      & ID thiết bị
		      &

		      Status code: 404 Not Found

		      Response message:

		      \{

		      "message": "No device found, please try again"

		      \}
		      & OK

		      \\ \hline


	      \end{xltabular}

	      % //TODO: new page
	\item URL: POST device/add
		  \clearpage
	      \begin{xltabular}{\textwidth}{
		      | >{\raggedright\arraybackslash}p{1cm}
		      | >{\raggedright\arraybackslash}p{2.5cm}
		      | >{\raggedright\arraybackslash}X
		      | >{\raggedright\arraybackslash}X
		      | >{\raggedright\arraybackslash}p{1cm}|
		      }
		      \caption{\bfseries \fontsize{12pt}{0pt}\selectfont Bảng kiểm thử API thêm thiết bị}
		      % \label{table_api_news}
		      \\
		      \hline
		      \bfseries Test case    &\bfseries Điều kiện   &\bfseries Đầu vào
		      &\bfseries Mong muốn đầu ra &\bfseries Kết quả\\ \hline


		      TC-1
		      & Bệnh nhân và bác sĩ tồn tại với ID tương ứng
		      & Thông tin thiết bị

		      \{

		      "user\_id": ID bệnh nhân,

		      "device\_name": Tên thiết bị,

		      "infomation": Thông tin thiết bị,

		      "device\_type\_id": ID loại thiết bị,

		      "status\_id": ID trạng thái thiết bị,

		      "start\_time": Giờ bắt đầu sử dụng

		      "end\_time": Giờ kết thúc sử dụng

		      \}
		      &

		      Status code: 200 OK

		      Response message:

		      \{
		      "message": "Device created successfully",

		      data: Thông tin của thiết bị

		      \}

		      & OK

		      \\ \hline

		      TC-2
		      & Người dùng không tồn tại với ID đã cho
		      & Thông tin thiết bị

		      \{

		      "user\_id": ID bệnh nhân,

		      "device\_name": Tên thiết bị,

		      "infomation": Thông tin thiết bị,

		      "device\_type\_id": ID loại thiết bị,

		      "status\_id": ID trạng thái thiết bị,

		      "start\_time": Giờ bắt đầu sử dụng

		      "end\_time": Giờ kết thúc sử dụng

		      \}
		      &

		      Status code: 404 Not found

		      Response message:

		      \{

		      "message": "No user found, please try again"

		      \}

		      & OK

		      \\ \hline


	      \end{xltabular}

	\item URL: POST device\_detail
	      \begin{xltabular}{\textwidth}{
		      | >{\raggedright\arraybackslash}p{1cm}
		      | >{\raggedright\arraybackslash}p{2.5cm}
		      | >{\raggedright\arraybackslash}X
		      | >{\raggedright\arraybackslash}X
		      | >{\raggedright\arraybackslash}p{1cm}|
		      }
		      \caption{\bfseries \fontsize{12pt}{0pt}\selectfont Bảng kiểm thử API thêm thông số kỹ thuật thiết bị}
		      % \label{table_api_news}
		      \\
		      \hline
		      \bfseries Test case    &\bfseries Điều kiện   &\bfseries Đầu vào
		      &\bfseries Mong muốn đầu ra &\bfseries Kết quả\\ \hline


		      TC-1
		      & Thiết bị tồn tại với ID cho trước
		      & thông số kỹ thuật thiết bị

		      \{

		      "device\_id": ID thiết bị,

		      "detail\_name": Tên chi tiết,

		      "detail\_type": loại chi tiết,

		      "value": giá trị,

		      "infomation": Thông tin ghi chú chi tiết,

		      \}
		      &

		      Status code: 200 OK

		      Response message:

		      \{

		      "message": "Device detail created successfully",

		      data: thông số kỹ thuật của thiết bị

		      \}

		      & OK

		      \\ \hline

		      TC-2
		      & Thiết bị không tồn tại với ID cho trước
		      & thông số kỹ thuật thiết bị

		      \{

		      "device\_id": ID thiết bị,

		      "detail\_name": Tên chi tiết,

		      "detail\_type": loại chi tiết,

		      "value": giá trị,

		      "infomation": Thông tin ghi chú chi tiết,

		      \}
		      &

		      Status code: 404 Not found

		      Response message:

		      \{

		      "message": "Device not found"

		      \}

		      & OK

		      \\ \hline

		      TC-3
		      & Người dùng không tồn tại với ID tương ứng
		      & thông số kỹ thuật thiết bị

		      \{

		      "device\_id": ID thiết bị,

		      "detail\_name": Tên chi tiết,

		      "detail\_type": loại chi tiết,

		      "value": giá trị,

		      "infomation": Thông tin ghi chú chi tiết,

		      \}
		      &

		      Status code: 404 Not found

		      Response message:

		      \{

		      "message": "No user found, please try again"

		      \}

		      & OK

		      \\ \hline

	      \end{xltabular}

	\item URL: DELETE device/{:device\_id}

	      \begin{xltabular}{\textwidth}{
		      | >{\raggedright\arraybackslash}p{1cm}
		      | >{\raggedright\arraybackslash}p{2.5cm}
		      | >{\raggedright\arraybackslash}X
		      | >{\raggedright\arraybackslash}X
		      | >{\raggedright\arraybackslash}p{1cm}|
		      }
		      \caption{\bfseries \fontsize{12pt}{0pt}\selectfont Bảng kiểm thử API xóa thiết bị theo id}
		      % \label{table_api_news}
		      \\
		      \hline
		      \bfseries Test case    &\bfseries Điều kiện   &\bfseries Đầu vào
		      &\bfseries Mong muốn đầu ra &\bfseries Kết quả\\ \hline


		      TC-1
		      & Thiết bị tồn tại với ID cho trước
		      & ID thiết bị

		      &

		      Status code: 200 OK

		      Response message:

		      \{

		      "message": "Delete device successful"

		      \}

		      & OK

		      \\ \hline

		      TC-2
		      & Không tồn tại thiết bị với ID cho trước
		      & ID thiết bị

		      &

		      Status code: 404 Not found

		      Response message:

		      \{

		      "message": "Device not found"

		      \}

		      & OK

		      \\ \hline


	      \end{xltabular}

	\item URL: PUT device/{:device\_id}
		  \clearpage
	      \begin{xltabular}{\textwidth}{
		      | >{\raggedright\arraybackslash}p{1cm}
		      | >{\raggedright\arraybackslash}p{2.5cm}
		      | >{\raggedright\arraybackslash}X
		      | >{\raggedright\arraybackslash}X
		      | >{\raggedright\arraybackslash}p{1cm}|
		      }
		      \caption{\bfseries \fontsize{12pt}{0pt}\selectfont Bảng kiểm thử API cập nhật thông tin thiết bị}
		      % \label{table_api_news}
		      \\
		      \hline
		      \bfseries Test case    &\bfseries Điều kiện   &\bfseries Đầu vào
		      &\bfseries Đầu ra mong muốn &\bfseries Kết quả\\ \hline


		      TC-1
		      & Thiết bị tồn tại với ID cho trước
		      & Thông tin thiết bị

		      \{

		      "user\_id": ID bệnh nhân,

		      "device\_name": Tên thiết bị,

		      "infomation": Thông tin thiết bị,

		      "device\_type\_id": ID loại thiết bị,

		      "status\_id": ID trạng thái thiết bị,

		      "start\_time": Giờ bắt đầu sử dụng

		      "end\_time": Giờ kết thúc sử dụng

		      \}
		      &

		      Status code: 200 OK

		      Response content:

		      \{

		      data: Thông tin sau khi cập nhật của thiết bị

		      \}

		      & OK

		      \\ \hline

		      TC-2
		      & Không tồn tại thiết bị với ID cho trước
		      & Thông tin thiết bị

		      \{

		      "user\_id": ID bệnh nhân,

		      "device\_name": Tên thiết bị,

		      "infomation": Thông tin thiết bị,

		      "device\_type\_id": ID loại thiết bị,

		      "status\_id": ID trạng thái thiết bị,

		      "start\_time": Giờ bắt đầu sử dụng

		      "end\_time": Giờ kết thúc sử dụng

		      \}
		      &

		      Status code: 404 Not found

		      Response content:

		      \{

		      "message": "No device found to update, please try again"

		      \}

		      & OK

		      \\ \hline

		      TC-3
		      & Người dùng không tồn tại với ID cho trước
		      & Thông tin thiết bị

		      \{

		      "user\_id": ID bệnh nhân,

		      "device\_name": Tên thiết bị,

		      "infomation": Thông tin thiết bị,

		      "device\_type\_id": ID loại thiết bị,

		      "status\_id": ID trạng thái thiết bị,

		      "start\_time": Giờ bắt đầu sử dụng

		      "end\_time": Giờ kết thúc sử dụng

		      \}
		      &

		      Status code: 404 Not found

		      Response content:

		      \{

		      "message": "No user found, please try again"

		      \}

		      & OK

		      \\ \hline
	      \end{xltabular}

	\item URL: PUT device\_detail/
	      \begin{xltabular}{\textwidth}{
		      | >{\raggedright\arraybackslash}p{1cm}
		      | >{\raggedright\arraybackslash}p{2.5cm}
		      | >{\raggedright\arraybackslash}X
		      | >{\raggedright\arraybackslash}X
		      | >{\raggedright\arraybackslash}p{1cm}|
		      }
		      \caption{\bfseries \fontsize{12pt}{0pt}\selectfont Bảng kiểm thử API cập nhật thông số kỹ thuật thiết bị}
		      % \label{table_api_news}
		      \\
		      \hline
		      \bfseries Test case    &\bfseries Điều kiện   &\bfseries Đầu vào
		      &\bfseries Mong muốn đầu ra &\bfseries Kết quả\\ \hline


		      TC-1
		      & Thông tin chi tiết mới phù hợp
		      & thông số kỹ thuật thiết bị

		      \{

		      "device\_id": ID thiết bị,

		      "detail\_name": Tên chi tiết,

		      "detail\_type": loại chi tiết,

		      "value": giá trị,

		      "infomation": Thông tin ghi chú chi tiết,

		      \}
		      &

		      Status code: 200 OK

		      Response message:

		      \{

		      "message": "Device detail updated successfully",

		      data: thông số kỹ thuật của thiết bị sau khi cập nhật

		      \}

		      & OK

		      \\ \hline

		      TC-2
		      & Thông tin chi tiết mới không phù hợp
		      & thông số kỹ thuật thiết bị

		      \{

		      "device\_id": ID thiết bị,

		      "detail\_name": Tên chi tiết,

		      "detail\_type": loại chi tiết,

		      "value": giá trị,

		      "infomation": Thông tin ghi chú chi tiết,

		      \}
		      &

		      Status code: 400 Bad Request

		      Response message:

		      \{

		      "message": "Error when update device"

		      \}

		      & OK

		      \\ \hline
	      \end{xltabular}

	\item URL: DELETE device\_detail/{:detail\_id}
	      \begin{xltabular}{\textwidth}{
		      | >{\raggedright\arraybackslash}p{1cm}
		      | >{\raggedright\arraybackslash}p{2.5cm}
		      | >{\raggedright\arraybackslash}X
		      | >{\raggedright\arraybackslash}X
		      | >{\raggedright\arraybackslash}p{1cm}|
		      }
		      \caption{\bfseries \fontsize{12pt}{0pt}\selectfont Bảng kiểm thử API xóa thông số kỹ thuật thiết bị}
		      % \label{table_api_news}
		      \\
		      \hline
		      \bfseries Test case    &\bfseries Điều kiện   &\bfseries Đầu vào
		      &\bfseries Mong muốn đầu ra &\bfseries Kết quả\\ \hline


		      TC-1
		      & Thông tin chi tiết mới phù hợp
		      & ID thông số kỹ thuật thiết bị
		      &

		      Status code: 200 OK

		      Response message:

		      \{

		      "message": "Device detail deleted successfully",

		      data: thông số kỹ thuật của thiết bị sau khi cập nhật

		      \}

		      & OK

		      \\ \hline
	      \end{xltabular}
\end{enumerate}



\paragraph{API liên quan đến dữ liệu phiên đo}
\mbox{}

\begin{enumerate}[a)]
	\clearpage
	\item URL: GET records
	      \begin{xltabular}{\textwidth}{
		      | >{\raggedright\arraybackslash}p{1cm}
		      | >{\raggedright\arraybackslash}p{2.5cm}
		      | >{\raggedright\arraybackslash}X
		      | >{\raggedright\arraybackslash}X
		      | >{\raggedright\arraybackslash}p{1cm}|
		      }
		      \caption{\bfseries \fontsize{12pt}{0pt}\selectfont Bảng kiểm thử API lấy tất cả dữ liệu phiên đo}
		      % \label{table_api_news}
		      \\
		      \hline
		      \bfseries Test case    &\bfseries Điều kiện   &\bfseries Đầu vào
		      &\bfseries Mong muốn đầu ra &\bfseries Kết quả\\ \hline


		      TC-1
		      & Quản trị viên của hệ thống
		      & Access Token tương ứng

		      &

		      Status code: 200 OK

		      Response message:

		      \{

		      data: Danh sách dữ liệu phiên đo

		      \}

		      & OK

		      \\ \hline

		      TC-2
		      & Không phải quản trị viên của hệ thống
		      & Access Token tương ứng

		      &

		      Status code: 403 Forbidden

		      Response message:

		      \{

		      "message": "Forbidden"

		      \}

		      & OK
		      \\ \hline


	      \end{xltabular}

	\item URL: GET records/{:id}
	      \begin{xltabular}{\textwidth}{
		      | >{\raggedright\arraybackslash}p{1cm}
		      | >{\raggedright\arraybackslash}p{2.5cm}
		      | >{\raggedright\arraybackslash}X
		      | >{\raggedright\arraybackslash}X
		      | >{\raggedright\arraybackslash}p{1cm}|
		      }
		      \caption{\bfseries \fontsize{12pt}{0pt}\selectfont Bảng kiểm thử API lấy dữ liệu phiên đo theo id}
		      % \label{table_api_news}
		      \\
		      \hline
		      \bfseries Test case    &\bfseries Điều kiện   &\bfseries Đầu vào
		      &\bfseries Mong muốn đầu ra &\bfseries Kết quả\\ \hline


		      TC-1
		      & Là quản trị viên, bác sĩ của hệ thống, dữ liệu phiên đo tồn tại
		      & ID của dữ liệu phiên đo

		      &

		      Status code: 200 OK

		      Response message:

		      \{

		      "data": Danh sách tất cả dữ liệu phiên đo trong hệ thống

		      \}
		      & OK

		      \\ \hline

		      TC-2
		      & Không phải quản trị viên, bác sĩ
		      & ID của dữ liệu phiên đo

		      &

		      Status code: 403 Forbidden

		      Response message:

		      \{

		      "message": "Forbidden"

		      \}
		      & OK

		      \\ \hline

		      TC-3
		      & Là quản trị viên, bác sĩ của hệ thống, dữ liệu phiên đo không tồn tại
		      & ID của dữ liệu phiên đo

		      &

		      Status code: 404 Not Found

		      Response message:

		      \{

		      "message": "No record found, please try again"

		      \}
		      & OK

		      \\ \hline


	      \end{xltabular}


	\item URL: GET records/doctor/{:doctorId}
	      \begin{xltabular}{\textwidth}{
		      | >{\raggedright\arraybackslash}p{1cm}
		      | >{\raggedright\arraybackslash}p{2.5cm}
		      | >{\raggedright\arraybackslash}X
		      | >{\raggedright\arraybackslash}X
		      | >{\raggedright\arraybackslash}p{1cm}|
		      }
		      \caption{\bfseries \fontsize{12pt}{0pt}\selectfont Bảng kiểm thử API lấy các dữ liệu phiên đo theo ID của bác sĩ}
		      % \label{table_api_news}
		      \\
		      \hline
		      \bfseries Test case    &\bfseries Điều kiện   &\bfseries Đầu vào
		      &\bfseries Mong muốn đầu ra &\bfseries Kết quả\\ \hline


		      TC-1
		      & Là quản trị viên hoặc bác sĩ có ID cho trước
		      & ID bác sĩ

		      &

		      Status code: 200 OK

		      Response message:

		      \{
			"status": "success"

		      "data": Danh sách các dữ liệu phiên đo của bác sĩ

		      \}
		      & OK
		      \\ \hline

		      TC-2
		      & Không phải là quản trị viên hoặc bác sĩ tương ứng
		      & ID bác sĩ

		      &

		      Status code: 403 Forbidden

		      Response message:

		      \{
				"status": "error"

		      "message": "Forbidden"

		      \}
		      & OK
		      \\ \hline


	      \end{xltabular}

	\item URL: POST records
	\clearpage
	      \begin{xltabular}{\textwidth}{
		      | >{\raggedright\arraybackslash}p{1cm}
		      | >{\raggedright\arraybackslash}p{2.5cm}
		      | >{\raggedright\arraybackslash}X
		      | >{\raggedright\arraybackslash}X
		      | >{\raggedright\arraybackslash}p{1cm}|
		      }
		      \caption{\bfseries \fontsize{12pt}{0pt}\selectfont Bảng kiểm thử API thêm dữ liệu phiên đo}
		      % \label{table_api_news}
		      \\
		      \hline
		      \bfseries Test case    &\bfseries Điều kiện   &\bfseries Đầu vào
		      &\bfseries Mong muốn đầu ra &\bfseries Kết quả\\ \hline


		      TC-1
		      & Là quản trị viên hoặc bác sĩ, file hợp lệ,
		      & Thông tin dữ liệu phiên đo

		      \{

		      "patient\_id": ID bệnh nhân,

		      "device\_id": ID thiết bị,

		      "record\_type": Loại dữ liệu phiên đo,

		      "start\_time": Thời gian bắt đầu,

		      "end\_time": Thời gian kết thúc,

		      "data\_rec\_url": Đường dẫn file dữ liệu phiên đo

		      \}
		      &

		      Status code: 200 OK

		      Response message:

		      \{

		      "message": "Record created successfully"

		      \}
		      & OK

		      \\ \hline

		      TC-2
		      & Là quản trị viên hoặc bác sĩ, file không hợp lệ,
		      & Thông tin dữ liệu phiên đo

		      \{

		      "patient\_id": ID bệnh nhân,

		      "device\_id": ID thiết bị,

		      "record\_type": Loại dữ liệu phiên đo,

		      "start\_time": Thời gian bắt đầu,

		      "end\_time": Thời gian kết thúc,

		      "data\_rec\_url": Đường dẫn file dữ liệu phiên đo

		      \}
		      &

		      Status code: 400 Bad Request

		      Response message:

		      \{

		      "message": "Invalid file type"

		      \}
		      & OK

		      \\ \hline

		      TC-3
		      & Thông tin dữ liệu phiên đo không hợp lệ
		      & Thông tin dữ liệu phiên đo

		      \{

		      "patient\_id": ID bệnh nhân,

		      "device\_id": ID thiết bị,

		      "record\_type": Loại dữ liệu phiên đo,

		      "start\_time": Thời gian bắt đầu,

		      "end\_time": Thời gian kết thúc,

		      "data\_rec\_url": Đường dẫn file dữ liệu phiên đo

		      \}
		      &

		      Status code: 400 Bad Request

		      Response message:

		      \{

		      "message": "Failed to create record"

		      \}
		      & OK

		      \\ \hline

	      \end{xltabular}

	\item URL: PUT records
	      \begin{xltabular}{\textwidth}{
		      | >{\raggedright\arraybackslash}p{1cm}
		      | >{\raggedright\arraybackslash}p{2.5cm}
		      | >{\raggedright\arraybackslash}X
		      | >{\raggedright\arraybackslash}X
		      | >{\raggedright\arraybackslash}p{1cm}|
		      }
		      \caption{\bfseries \fontsize{12pt}{0pt}\selectfont Bảng kiểm thử API cập nhật dữ liệu dữ liệu phiên đo}
		      % \label{table_api_news}
		      \\
		      \hline
		      \bfseries Test case    &\bfseries Điều kiện   &\bfseries Đầu vào
		      &\bfseries Mong muốn đầu ra &\bfseries Kết quả\\ \hline


		      TC-1
		      & Dữ liệu cập nhật phù hợp
		      & Thông tin cập nhật dữ liệu phiên đo
		      \{

		      "patient\_id": ID bệnh nhân,

		      "device\_id": ID thiết bị,

		      "record\_type": Loại dữ liệu phiên đo,

		      "start\_time": Thời gian bắt đầu,

		      "end\_time": Thời gian kết thúc,

		      "data\_rec\_url": Đường dẫn file dữ liệu phiên đo

		      \}
		      &

		      Status code: 200 OK

		      Response message:

		      \{

		      "message": "Record updated successfully"

		      \}

		      & OK

		      \\ \hline

		      TC-2
		      & Dữ liệu cập nhật không phù hợp
		      & Thông tin cập nhật dữ liệu phiên đo
		      \{

		      "patient\_id": ID bệnh nhân,

		      "device\_id": ID thiết bị,

		      "record\_type": Loại dữ liệu phiên đo,

		      "start\_time": Thời gian bắt đầu,

		      "end\_time": Thời gian kết thúc,

		      "data\_rec\_url": Đường dẫn file dữ liệu phiên đo

		      \}

		      &

		      Status code: 404 Not Found

		      Response message:

		      \{

		      "message": "Error when update record"

		      \}

		      & OK

		      \\ \hline
	      \end{xltabular}

	\item URL: DELETE records/{:record\_id}
	      \begin{xltabular}{\textwidth}{
		      | >{\raggedright\arraybackslash}p{1cm}
		      | >{\raggedright\arraybackslash}p{2.5cm}
		      | >{\raggedright\arraybackslash}X
		      | >{\raggedright\arraybackslash}X
		      | >{\raggedright\arraybackslash}p{1cm}|
		      }
		      \caption{\bfseries \fontsize{12pt}{0pt}\selectfont Bảng kiểm thử API xóa dữ liệu dữ liệu phiên đo theo id}
		      % \label{table_api_news}
		      \\
		      \hline
		      \bfseries Test case    &\bfseries Điều kiện   &\bfseries Đầu vào
		      &\bfseries Mong muốn đầu ra &\bfseries Kết quả\\ \hline


		      TC-1
		      & Tồn tại dữ liệu phiên đo với ID cho trước
		      & ID dữ liệu phiên đo

		      &

		      Status code: 200 OK

		      Response message:

		      \{

		      "message": "Record deleted successfully"

		      \}
		      & OK

		      \\ \hline

		      TC-2
		      & Không tồn tại dữ liệu phiên đo với ID cho trước
		      & ID dữ liệu phiên đo

		      &

		      Status code: 404 Not Found

		      Response message:

		      \{

		      "message": "No record found to delete, please try again"

		      \}
		      & OK

		      \\ \hline


	      \end{xltabular}


\end{enumerate}


\paragraph{API liên quan liên quan đến việc đặt lịch bác sĩ - bệnh nhân}
\mbox{}

\begin{enumerate}[a)]
	\item URL: GET schedules
	      \begin{xltabular}{\textwidth}{
		      | >{\raggedright\arraybackslash}p{1cm}
		      | >{\raggedright\arraybackslash}p{2.5cm}
		      | >{\raggedright\arraybackslash}X
		      | >{\raggedright\arraybackslash}X
		      | >{\raggedright\arraybackslash}p{1cm}|
		      }
		      \caption{\bfseries \fontsize{12pt}{0pt}\selectfont Bảng kiểm thử API lấy tất cả lịch khám của các bác sĩ - các bệnh nhân}
		      % \label{table_api_news}
		      \\
		      \hline
		      \bfseries Test case    &\bfseries Điều kiện   &\bfseries Đầu vào
		      &\bfseries Mong muốn đầu ra &\bfseries Kết quả\\ \hline


		      TC-1
		      & Quản trị viên hệ thống
		      & Access token tương ứng
		      &

		      Status code: 200 OK

		      Response message:

		      \{

		      data: Danh sách tất cả lịch khám

		      \}
		      & OK

		      \\ \hline

		      TC-2
		      & Không phải quản trị viên hệ thống
		      & NULL

		      &

		      Status code: 403 Forbidden

		      Response message:

		      \{

		      "message": "Forbidden"

		      \}
		      & OK
		      \\ \hline
	      \end{xltabular}

	\item URL: GET schedules/doctor-id
	      \begin{xltabular}{\textwidth}{
		      | >{\raggedright\arraybackslash}p{1cm}
		      | >{\raggedright\arraybackslash}p{2.5cm}
		      | >{\raggedright\arraybackslash}X
		      | >{\raggedright\arraybackslash}X
		      | >{\raggedright\arraybackslash}p{1cm}|
		      }
		      \caption{\bfseries \fontsize{12pt}{0pt}\selectfont Bảng kiểm thử API lấy danh sách lịch khám theo ID bác sĩ}
		      % \label{table_api_news}
		      \\
		      \hline
		      \bfseries Test case    &\bfseries Điều kiện   &\bfseries Đầu vào
		      &\bfseries Mong muốn đầu ra &\bfseries Kết quả\\ \hline


		      TC-1
		      & Bác sĩ tồn tại trong hệ thống
		      & ID bác sĩ
		      &

		      Status code: 200 OK

		      Response message:

		      \{

		      data: Danh sách lịch khám của bác sĩ

		      \}

		      & OK

		      \\ \hline

		      TC-2
		      & Bác sĩ không tồn tại trong hệ thống
		      & ID bác sĩ
		      &

		      Status code: 404 Not Found

		      Response message:

		      \{

		      "status": "error",

		      "message": "No user found, please try again."

		      \}

		      & OK

		      \\ \hline


	      \end{xltabular}

	\item URL: GET schedules/patient-id
	      \begin{xltabular}{\textwidth}{
		      | >{\raggedright\arraybackslash}p{1cm}
		      | >{\raggedright\arraybackslash}p{2.5cm}
		      | >{\raggedright\arraybackslash}X
		      | >{\raggedright\arraybackslash}X
		      | >{\raggedright\arraybackslash}p{1cm}|
		      }
		      \caption{\bfseries \fontsize{12pt}{0pt}\selectfont Bảng kiểm thử API lấy danh sách lịch khám theo ID bệnh nhân}
		      % \label{table_api_news}
		      \\
		      \hline
		      \bfseries Test case    &\bfseries Điều kiện   &\bfseries Đầu vào
		      &\bfseries Mong muốn đầu ra &\bfseries Kết quả\\ \hline


		      TC-1
		      & Bệnh nhân tồn tại trong hệ thống
		      & ID bác sĩ
		      &

		      Status code: 200 OK

		      Response message:

		      \{

		      data: Danh sách lịch khám của bệnh nhân

		      \}

		      & OK

		      \\ \hline

		      TC-2
		      & Bệnh nhân không tồn tại trong hệ thống
		      & ID bác sĩ
		      &

		      Status code: 404 Not Found

		      Response message:

		      \{
		      "message": "No user found, please try again."

		      \}

		      & OK

		      \\ \hline


	      \end{xltabular}

	\item URL: POST schedules/create-by-doctor
	      \begin{xltabular}{\textwidth}{
		      | >{\raggedright\arraybackslash}p{1cm}
		      | >{\raggedright\arraybackslash}p{2.5cm}
		      | >{\raggedright\arraybackslash}X
		      | >{\raggedright\arraybackslash}X
		      | >{\raggedright\arraybackslash}p{1cm}|
		      }
		      \caption{\bfseries \fontsize{12pt}{0pt}\selectfont Bảng kiểm thử API tạo lịch khám bởi bác sĩ}
		      % \label{table_api_news}
		      \\
		      \hline
		      \bfseries Test case    &\bfseries Điều kiện   &\bfseries Đầu vào
		      &\bfseries Mong muốn đầu ra &\bfseries Kết quả\\ \hline


		      TC-1
		      & Các trường thông tin lịch khám hợp lệ
		      & Thông tin lịch khám
		      \{

		      "doctor\_id": ID bệnh nhân,

		      "patient\_id": ID bác sĩ,

		      "schedule\_start\_time": Thời gian bắt đầu,

		      "schedule\_end\_time": Thời gian kết thúc,

		      "status\_id": ID trạng thái lịch khám

		      "schedule\_result": Kết quả lịch khám

		      \}
		      &

		      Status code: 200 OK

		      Response message:

		      \{

		      "message": "Schedule created successfully"

		      \}
		      & OK

		      \\ \hline
		      TC-2
		      & Các trường thông tin lịch khám không hợp lệ
		      & Thông tin lịch khám
		      \{

		      "doctor\_id": ID bệnh nhân,

		      "patient\_id": ID bác sĩ,

		      "schedule\_start\_time": Thời gian bắt đầu,

		      "schedule\_end\_time": Thời gian kết thúc,

		      "status\_id": ID trạng thái lịch khám

		      "schedule\_result": Kết quả lịch khám

		      \}
		      &

		      Status code: 400 Bad Request

		      Response message:

		      \{

		      "message": "Failed to create schedule"

		      \}
		      & OK

		      \\ \hline
	      \end{xltabular}

	\item URL: POST schedules/create-by-patient
	      \begin{xltabular}{\textwidth}{
		      | >{\raggedright\arraybackslash}p{1cm}
		      | >{\raggedright\arraybackslash}p{2.5cm}
		      | >{\raggedright\arraybackslash}X
		      | >{\raggedright\arraybackslash}X
		      | >{\raggedright\arraybackslash}p{1cm}|
		      }
		      \caption{\bfseries \fontsize{12pt}{0pt}\selectfont Bảng kiểm thử API tạo lịch khám bởi bệnh nhân}
		      % \label{table_api_news}
		      \\
		      \hline
		      \bfseries Test case    &\bfseries Điều kiện   &\bfseries Đầu vào
		      &\bfseries Mong muốn đầu ra &\bfseries Kết quả\\ \hline


		      TC-1
		      & Các trường thông tin lịch khám hợp lệ
		      & Thông tin lịch khám
		      \{

		      "doctor\_id": ID bệnh nhân,

		      "patient\_id": ID bác sĩ,

		      "schedule\_start\_time": Thời gian bắt đầu,

		      "schedule\_end\_time": Thời gian kết thúc,

		      "status\_id": ID trạng thái lịch khám

		      \}
		      &

		      Status code: 200 OK

		      Response message:

		      \{

		      "message": "Schedule created successfully"

		      \}
		      & OK

		      \\ \hline

		      TC-2
		      & Lịch khám đang chờ phê duyệt của bệnh nhân vượt quá 5
		      & Thông tin lịch khám
		      \{

		      "doctor\_id": ID bệnh nhân,

		      "patient\_id": ID bác sĩ,

		      "schedule\_start\_time": Thời gian bắt đầu,

		      "schedule\_end\_time": Thời gian kết thúc,

		      "status\_id": ID trạng thái lịch khám

		      \}
		      &

		      Status code: 400 Bad Request

		      Response message:

		      \{

		      "message": "Quá giới hạn lịch được đặt, vui lòng đợi các bác sĩ phê duyệt lịch đã đặt trước khi tiếp tục."

		      \}
		      & OK

		      \\ \hline

		      TC-3
		      & Đặt lịch trùng thời gian với lịch khám đang chờ phê duyệt
		      & Thông tin lịch khám
		      \{

		      "doctor\_id": ID bệnh nhân,

		      "patient\_id": ID bác sĩ,

		      "schedule\_start\_time": Thời gian bắt đầu,

		      "schedule\_end\_time": Thời gian kết thúc,

		      "status\_id": ID trạng thái lịch khám

		      \}
		      &

		      Status code: 400 Bad Request

		      Response message:

		      \{

		      "message": "Bạn đã đặt lịch vào thời điểm này trước đó, vui lòng đợi bác sĩ phê duyệt"

		      \}
		      & OK

		      \\ \hline

		      TC-4
		      & Đặt lịch trùng thời gian với lịch khám đã được chấp nhận
		      & Thông tin lịch khám
		      \{

		      "doctor\_id": ID bệnh nhân,

		      "patient\_id": ID bác sĩ,

		      "schedule\_start\_time": Thời gian bắt đầu,

		      "schedule\_end\_time": Thời gian kết thúc,

		      "status\_id": ID trạng thái lịch khám

		      \}
		      &

		      Status code: 400 Bad Request

		      Response message:

		      \{

		      "message": "Bạn đã có lịch vào thời điểm này, vui lòng kiểm tra lại."

		      \}
		      & OK

		      \\ \hline
	      \end{xltabular}

	\item URL: GET schedules/time/available-doctor/:schedule\_time
	      \begin{xltabular}{\textwidth}{
		      | >{\raggedright\arraybackslash}p{1cm}
		      | >{\raggedright\arraybackslash}p{2.5cm}
		      | >{\raggedright\arraybackslash}X
		      | >{\raggedright\arraybackslash}X
		      | >{\raggedright\arraybackslash}p{1cm}|
		      }
		      \caption{\bfseries \fontsize{12pt}{0pt}\selectfont Bảng kiểm thử API lấy danh sách bác sĩ có thể đặt lịch khám theo thời gian cụ thể}
		      % \label{table_api_news}
		      \\
		      \hline
		      \bfseries Test case    &\bfseries Điều kiện   &\bfseries Đầu vào
		      &\bfseries Mong muốn đầu ra &\bfseries Kết quả\\ \hline


		      TC-1
		      & Thời gian hợp lệ
		      & Thời gian đã chọn trước
		      &

		      Status code: 200 OK

		      Response message:

		      \{

		      data: Danh sách các bác sĩ có thể đặt lịch khám

		      \}

		      & OK

		      \\ \hline

	      \end{xltabular}

	\item URL: GET schedules/available-schedule/:id
	      \begin{xltabular}{\textwidth}{
		      | >{\raggedright\arraybackslash}p{1cm}
		      | >{\raggedright\arraybackslash}p{2.5cm}
		      | >{\raggedright\arraybackslash}X
		      | >{\raggedright\arraybackslash}X
		      | >{\raggedright\arraybackslash}p{1cm}|
		      }
		      \caption{\bfseries \fontsize{12pt}{0pt}\selectfont Bảng kiểm thử API lấy danh sách thời gian có thể đặt lịch khám của bác sĩ cụ thể}
		      % \label{table_api_news}
		      \\
		      \hline
		      \bfseries Test case    &\bfseries Điều kiện   &\bfseries Đầu vào
		      &\bfseries Mong muốn đầu ra &\bfseries Kết quả\\ \hline


		      TC-1
		      & Bác sĩ tồn tại trong hệ thống
		      & ID của bác sĩ đã chọn
		      &

		      Status code: 200 OK

		      Response message:

		      \{

		      data: Danh sách các lịch khám có thể đặt với bác sĩ này

		      \}

		      & OK

		      \\ \hline

	      \end{xltabular}

	\item URL: PUT schedules/accept-schedule
	\clearpage
	      \begin{xltabular}{\textwidth}{
		      | >{\raggedright\arraybackslash}p{1cm}
		      | >{\raggedright\arraybackslash}p{2.5cm}
		      | >{\raggedright\arraybackslash}X
		      | >{\raggedright\arraybackslash}X
		      | >{\raggedright\arraybackslash}p{1cm}|
		      }
		      \caption{\bfseries \fontsize{12pt}{0pt}\selectfont Bảng kiểm thử API bác sĩ chấp nhận lịch khám từ bệnh nhân}
		      % \label{table_api_news}
		      \\
		      \hline
		      \bfseries Test case    &\bfseries Điều kiện   &\bfseries Đầu vào
		      &\bfseries Mong muốn đầu ra &\bfseries Kết quả\\ \hline


		      TC-1
		      & Lịch khám của bệnh nhân đang ở trạng thái chờ phê duyệt
		      & Thông tin lịch khám đang chờ phê duyệt
		      \{

		      "doctor\_id": ID bệnh nhân,

		      "patient\_id": ID bác sĩ,

		      "schedule\_start\_time": Thời gian bắt đầu,

		      "schedule\_end\_time": Thời gian kết thúc,

		      "status\_id": ID trạng thái lịch khám

		      "schedule\_result": Kết quả lịch khám

		      \}
		      &

		      Status code: 200 OK

		      Response message:

		      \{

		      "message": "Schedule accepted successfully"

		      \}

		      & OK

		      \\ \hline

	      \end{xltabular}

	\item URL: DELETE schedules/reject-schedule/{:id}
	      \begin{xltabular}{\textwidth}{
		      | >{\raggedright\arraybackslash}p{1cm}
		      | >{\raggedright\arraybackslash}p{2.5cm}
		      | >{\raggedright\arraybackslash}X
		      | >{\raggedright\arraybackslash}X
		      | >{\raggedright\arraybackslash}p{1cm}|
		      }
		      \caption{\bfseries \fontsize{12pt}{0pt}\selectfont Bảng kiểm thử API bác sĩ từ chối lịch khám từ bệnh nhân}
		      % \label{table_api_news}
		      \\
		      \hline
		      \bfseries Test case    &\bfseries Điều kiện   &\bfseries Đầu vào
		      &\bfseries Mong muốn đầu ra &\bfseries Kết quả\\ \hline


		      TC-1
		      & Lịch khám của bệnh nhân đang ở trạng thái chờ phê duyệt
		      & Thông tin lịch khám đang chờ phê duyệt
		      \{

		      "doctor\_id": ID bệnh nhân,

		      "patient\_id": ID bác sĩ,

		      "schedule\_start\_time": Thời gian bắt đầu,

		      "schedule\_end\_time": Thời gian kết thúc,

		      "status\_id": ID trạng thái lịch khám

		      "schedule\_result": Kết quả lịch khám

		      \}
		      &

		      Status code: 200 OK

		      Response message:

		      \{

		      "message": "Schedule has been rejected successfully"

		      \}

		      & OK

		      \\ \hline

	      \end{xltabular}

\end{enumerate}

\paragraph{API liên quan liên quan đến chẩn đoán}
\mbox{}
\begin{enumerate}
	\item URL: POST diagnosis/
	      \begin{xltabular}{\textwidth}{
		      | >{\raggedright\arraybackslash}p{1cm}
		      | >{\raggedright\arraybackslash}p{2.5cm}
		      | >{\raggedright\arraybackslash}X
		      | >{\raggedright\arraybackslash}X
		      | >{\raggedright\arraybackslash}p{1cm}|
		      }
		      \caption{\bfseries \fontsize{12pt}{0pt}\selectfont Bảng kiểm thử API tạo chẩn đoán mới}
		      % \label{table_api_news}
		      \\
		      \hline
		      \bfseries Test case    &\bfseries Điều kiện   &\bfseries Đầu vào
		      &\bfseries Mong muốn đầu ra &\bfseries Kết quả\\ \hline


		      TC-1
		      & Ca khám đã có chẩn đoán từ bác sĩ
		      & Thông tin chẩn đoán của ca khám
		      \{

		      "schedule\_id": ID lịch khám,

		      "infomation": Thông tin chẩn đoán,

		      \}
		      &

		      Status code: 200 OK

		      Response message:

		      \{

		      "message": "Diagnosis created successfully"

		      \}

		      & OK

		      \\ \hline

		      TC-2
		      & Ca khám chưa có chẩn đoán từ bác sĩ
		      & Thông tin chẩn đoán của buổi khám
		      \{

		      "schedule\_id": ID lịch khám,

		      "infomation": Thông tin chẩn đoán,

		      \}
		      &

		      Status code: 400 Bad Request

		      Response message:

		      \{

		      "message": "Failed to create diagnosis"

		      \}

		      & OK

		      \\ \hline

	      \end{xltabular}

	\item URL: GET diagnosis/schedule/{:schedule\_id}
	      \begin{xltabular}{\textwidth}{
		      | >{\raggedright\arraybackslash}p{1cm}
		      | >{\raggedright\arraybackslash}p{2.5cm}
		      | >{\raggedright\arraybackslash}X
		      | >{\raggedright\arraybackslash}X
		      | >{\raggedright\arraybackslash}p{1cm}|
		      }
		      \caption{\bfseries \fontsize{12pt}{0pt}\selectfont Bảng kiểm thử API lấy thông tin chẩn đoán theo ID lịch khám}
		      % \label{table_api_news}
		      \\
		      \hline
		      \bfseries Test case    &\bfseries Điều kiện   &\bfseries Đầu vào
		      &\bfseries Mong muốn đầu ra &\bfseries Kết quả\\ \hline


		      TC-1
		      & Lịch khám của bệnh nhân đã được chấp nhận
		      & ID lịch khám
		      &

		      Status code: 200 OK

		      Response message:

		      \{

		      data: dữ liệu chẩn đoán của buổi khám

		      \}

		      & OK

		      \\ \hline

		      TC-2
		      & Lịch khám của bệnh nhân chưa được chấp nhận
		      & ID lịch khám
		      &

		      Status code: 400 Bad Request

		      Response message:

		      \{

		      "message": "No diagnosis found, please try again"

		      \}

		      & OK

		      \\ \hline

	      \end{xltabular}

	\item URL: POST diagnosis/update
	      \begin{xltabular}{\textwidth}{
		      | >{\raggedright\arraybackslash}p{1cm}
		      | >{\raggedright\arraybackslash}p{2.5cm}
		      | >{\raggedright\arraybackslash}X
		      | >{\raggedright\arraybackslash}X
		      | >{\raggedright\arraybackslash}p{1cm}|
		      }
		      \caption{\bfseries \fontsize{12pt}{0pt}\selectfont Bảng kiểm thử API cập nhật chẩn đoán theo id}
		      % \label{table_api_news}
		      \\
		      \hline
		      \bfseries Test case    &\bfseries Điều kiện   &\bfseries Đầu vào
		      &\bfseries Mong muốn đầu ra &\bfseries Kết quả\\ \hline


		      TC-1
		      & Thông tin cập nhật chẩn đoán hợp lệ
		      & ID chẩn đoán, thông tin chẩn đoán của buổi khám
		      \{

		      "schedule\_id": ID lịch khám,

		      "infomation": Thông tin chẩn đoán,

		      \}
		      &

		      Status code: 200 OK

		      Response message:

		      \{

		      "message": "Diagnosis updated successfully"

		      \}

		      & OK

		      \\ \hline

		      TC-2
		      & Thông tin cập nhật chẩn đoán không hợp lệ
		      & ID chẩn đoán, thông tin chẩn đoán của buổi khám
		      \{

		      "schedule\_id": ID lịch khám,

		      "infomation": Thông tin chẩn đoán,

		      \}
		      &

		      Status code: 400 Bad Request

		      Response message:

		      \{

		      "message": "Error when update diagnosis by schedule id"

		      \}

		      & OK

		      \\ \hline

	      \end{xltabular}
\end{enumerate}

\paragraph{API liên quan liên quan đến thông báo}
\mbox{}
\begin{enumerate}
	\item URL: GET notification/get
	      \begin{xltabular}{\textwidth}{
		      | >{\raggedright\arraybackslash}p{1cm}
		      | >{\raggedright\arraybackslash}p{2.5cm}
		      | >{\raggedright\arraybackslash}X
		      | >{\raggedright\arraybackslash}X
		      | >{\raggedright\arraybackslash}p{1cm}|
		      }
		      \caption{\bfseries \fontsize{12pt}{0pt}\selectfont Bảng kiểm thử API lấy thông báo theo ID của người dùng}
		      % \label{table_api_news}
		      \\
		      \hline
		      \bfseries Test case    &\bfseries Điều kiện   &\bfseries Đầu vào
		      &\bfseries Mong muốn đầu ra &\bfseries Kết quả\\ \hline


		      TC-1
		      & Là bệnh nhân
		      & ID người dùng tương ứng
		      &

		      Status code: 200 OK

		      Response message:

		      \{

		      data: Danh sách thông báo của người dùng này

		      \}

		      & OK

		      \\ \hline

		      TC-2
		      & Là bác sĩ của hệ thống
		      & ID người dùng tương ứng
		      &

		      Status code: 200 OK

		      Response message:

		      \{

		      data: Danh sách thông báo của người dùng này

		      \}

		      & OK

		      \\ \hline

	      \end{xltabular}

	\item URL: POST notification
	      \begin{xltabular}{\textwidth}{
		      | >{\raggedright\arraybackslash}p{1cm}
		      | >{\raggedright\arraybackslash}p{2.5cm}
		      | >{\raggedright\arraybackslash}X
		      | >{\raggedright\arraybackslash}X
		      | >{\raggedright\arraybackslash}p{1cm}|
		      }
		      \caption{\bfseries \fontsize{12pt}{0pt}\selectfont Bảng kiểm thử API tạo thông báo}
		      % \label{table_api_news}
		      \\
		      \hline
		      \bfseries Test case    &\bfseries Điều kiện   &\bfseries Đầu vào
		      &\bfseries Mong muốn đầu ra &\bfseries Kết quả\\ \hline


		      TC-1
		      & Là người dùng của hệ thống và lịch khám liên quan chưa được phê duyệt
		      & Thông tin thông báo
		      \{

		      "doctor\_id": ID bác sĩ,

			  "patient\_id": ID bệnh nhân,

		      "schedule\_start\_time": Thời điểm bắt đầu lịch khám,

			  "type": Loại thông báo,

			  "status": Trạng thái thông báo

		      \}
		      &

		      Status code: 200 OK

		      Response message:

		      \{

		      "message": "Notification added successfully",

		      \}

		      & OK

		      \\ \hline

		      TC-2
		      & Là người dùng của hệ thống và lịch khám đã được phê duyệt
		      & Thông tin thông báo
		      \{

		      "doctor\_id": ID bác sĩ,

			  "patient\_id": ID bệnh nhân,

		      "schedule\_start\_time": Thời điểm bắt đầu lịch khám,

			  "type": Loại thông báo,

			  "status": Trạng thái thông báo,

			  "reject\_reason": Lý do từ chối lịch khám (nếu có)

		      \}
		      &

		      Status code: 200 OK

		      Response message:

		      \{

		      "message": "Notification added successfully",

		      \}

		      & OK

		      \\ \hline

	      \end{xltabular}

	\item URL: POST notification/update-seen
	      \begin{xltabular}{\textwidth}{
		      | >{\raggedright\arraybackslash}p{1cm}
		      | >{\raggedright\arraybackslash}p{2.5cm}
		      | >{\raggedright\arraybackslash}X
		      | >{\raggedright\arraybackslash}X
		      | >{\raggedright\arraybackslash}p{1cm}|
		      }
		      \caption{\bfseries \fontsize{12pt}{0pt}\selectfont Bảng kiểm thử API cập nhật trạng thái thông báo đã được xem}
		      % \label{table_api_news}
		      \\
		      \hline
		      \bfseries Test case    &\bfseries Điều kiện   &\bfseries Đầu vào
		      &\bfseries Mong muốn đầu ra &\bfseries Kết quả\\ \hline


		      TC-1
		      & Thông báo chưa được xem
		      & ID của thông báo
		      &

		      Status code: 200 OK

		      Response message:

		      \{

		      "message": "Seen status updated successfully",

		      \}

		      & OK

		      \\ \hline

		      TC-2
		      & Thông báo đã được xem
		      & ID của thông báo
		      &

		      Status code: 400 Bad Request

		      Response message:

		      \{

		      "message": "400 Bad Request",

		      \}

		      & OK

		      \\ \hline

	      \end{xltabular}

	\item URL: DELETE notification/{:id}
		  \clearpage
	      \begin{xltabular}{\textwidth}{
		      | >{\raggedright\arraybackslash}p{1cm}
		      | >{\raggedright\arraybackslash}p{2.5cm}
		      | >{\raggedright\arraybackslash}X
		      | >{\raggedright\arraybackslash}X
		      | >{\raggedright\arraybackslash}p{1cm}|
		      }
		      \caption{\bfseries \fontsize{12pt}{0pt}\selectfont Bảng kiểm thử API xóa thông báo theo id}
		      % \label{table_api_news}
		      \\
		      \hline
		      \bfseries Test case    &\bfseries Điều kiện   &\bfseries Đầu vào
		      &\bfseries Mong muốn đầu ra &\bfseries Kết quả\\ \hline


		      TC-1
		      & Thông báo tồn tại trong hệ thống
		      & ID của thông báo
		      &

		      Status code: 200 OK

		      Response message:

		      \{

		      "message": "Seen status updated successfully",

		      \}

		      & OK

		      \\ \hline

		      TC-2
		      & Thông báo không tồn tại trong hệ thống
		      & ID của thông báo
		      &

		      Status code: 400 Bad Request

		      Response message:

		      \{

		      "message": "400 Bad Request",

		      \}

		      & OK

		      \\ \hline

	      \end{xltabular}
\end{enumerate}

\paragraph{API liên quan liên quan đến tin nhắn}
\mbox{}
\begin{enumerate}
	\item URL: POST groupChat
	      \begin{xltabular}{\textwidth}{
		      | >{\raggedright\arraybackslash}p{1cm}
		      | >{\raggedright\arraybackslash}p{2.5cm}
		      | >{\raggedright\arraybackslash}X
		      | >{\raggedright\arraybackslash}X
		      | >{\raggedright\arraybackslash}p{1cm}|
		      }
		      \caption{\bfseries \fontsize{12pt}{0pt}\selectfont Bảng kiểm thử API tạo nhóm trò chuyện}
		      % \label{table_api_news}
		      \\
		      \hline
		      \bfseries Test case    &\bfseries Điều kiện   &\bfseries Đầu vào
		      &\bfseries Mong muốn đầu ra &\bfseries Kết quả\\ \hline


		      TC-1
		      & Quản trị viên hoặc bác sĩ của hệ thống
		      & Thông tin nhóm trò chuyện
		      \{
		      "title": Tên nhóm trò chuyện,
		      "hostId: ID của người tạo nhóm,
		      "member": Danh sách ID của các thành viên trong nhóm trò chuyện,
		      "sendEvent": Sự kiện gửi của nhóm trò chuyện
		      "receiveEvent": Sự kiện nhận của nhóm trò chuyện
		      \}
		      &

		      Status code: 200 OK

		      Response message:

		      \{

		      data: thông tin của nhóm trò chuyện đã tạo,

		      \}

		      & OK

		      \\ \hline

		      TC-2
		      & Là bệnh nhân của hệ thống (bệnh nhân chỉ có thể tạo đoạn chat 2 người với bác sĩ khám cho mình)
		      & Thông tin nhóm trò chuyện
		      \{
		      "title": Tên nhóm trò chuyện,
		      "hostId: ID của người tạo nhóm,
		      "member": Danh sách ID của các thành viên trong nhóm,
		      "sendEvent": Sự kiện gửi của nhóm trò chuyện
		      "receiveEvent": Sự kiện nhận của nhóm trò chuyện
		      \}
		      &

		      Status code: 200 OK

		      Response message:

		      \{

		      data: thông tin của đoạn chat đã tạo,

		      \}

		      & OK

		      \\ \hline

	      \end{xltabular}

	\item URL: GET groupChat
	      \begin{xltabular}{\textwidth}{
		      | >{\raggedright\arraybackslash}p{1cm}
		      | >{\raggedright\arraybackslash}p{2.5cm}
		      | >{\raggedright\arraybackslash}X
		      | >{\raggedright\arraybackslash}X
		      | >{\raggedright\arraybackslash}p{1cm}|
		      }
		      \caption{\bfseries \fontsize{12pt}{0pt}\selectfont Bảng kiểm thử API tìm danh sách nhóm trò chuyện của người dùng}
		      % \label{table_api_news}
		      \\
		      \hline
		      \bfseries Test case    &\bfseries Điều kiện   &\bfseries Đầu vào
		      &\bfseries Mong muốn đầu ra &\bfseries Kết quả\\ \hline


		      TC-1
		      & Người dùng của hệ thống
		      & ID người dùng tương ứng
		      &

		      Status code: 200 OK

		      Response message:

		      \{

		      data: danh sách nhóm trò chuyện của người dùng này

		      \}

		      & OK

		      \\ \hline

		      TC-2
		      & Không phải người dùng hệ thống
		      & ID của người dùng
		      &

		      Status code: 401 Unauthorized

		      Response message:

		      \{

		      "message": "401 Unauthorized",

		      \}

		      & OK

		      \\ \hline

	      \end{xltabular}

	\item URL: GET chat/messages/:groupChatId
		  \clearpage
	      \begin{xltabular}{\textwidth}{
		      | >{\raggedright\arraybackslash}p{1cm}
		      | >{\raggedright\arraybackslash}p{2.5cm}
		      | >{\raggedright\arraybackslash}X
		      | >{\raggedright\arraybackslash}X
		      | >{\raggedright\arraybackslash}p{1cm}|
		      }
		      \caption{\bfseries \fontsize{12pt}{0pt}\selectfont Bảng kiểm thử API lấy tin nhắn theo ID nhóm trò chuyện}
		      % \label{table_api_news}
		      \\
		      \hline
		      \bfseries Test case    &\bfseries Điều kiện   &\bfseries Đầu vào
		      &\bfseries Mong muốn đầu ra &\bfseries Kết quả\\ \hline


		      TC
		      & Người dùng của hệ thống
		      & ID nhóm trò chuyện của người dùng
		      &

		      Status code: 200 OK

		      Response message:

		      \{

		      Tin nhắn của người dùng trong nhóm trò chuyện này được lấy đúng,

		      Tin nhắn của các người dùng khác trong nhóm trò chuyện này được lấy đúng

		      \}

		      & OK

		      \\ \hline

	      \end{xltabular}

	\item URL: POST chat/send
	      \begin{xltabular}{\textwidth}{
		      | >{\raggedright\arraybackslash}p{1cm}
		      | >{\raggedright\arraybackslash}p{2.5cm}
		      | >{\raggedright\arraybackslash}X
		      | >{\raggedright\arraybackslash}X
		      | >{\raggedright\arraybackslash}p{1cm}|
		      }
		      \caption{\bfseries \fontsize{12pt}{0pt}\selectfont Bảng kiểm thử API gửi tin nhắn trong nhóm trò chuyện}
		      % \label{table_api_news}
		      \\
		      \hline
		      \bfseries Test case    &\bfseries Điều kiện   &\bfseries Đầu vào
		      &\bfseries Mong muốn đầu ra &\bfseries Kết quả\\ \hline


		      TC
		      & Người dùng của hệ thống
		      & Thông tin tin nhắn
		      \{
		      "senderId" : ID người gửi,
		      "groupChatId" : ID nhóm trò chuyện,
		      "message" : nội dung tin nhắn,
		      "time": thời gian gửi
		      \}
		      &

		      Status code: 200 OK

		      Tin nhắn được gửi đúng nhóm,

		      Thời gian gửi được lưu vào cơ sở dữ liệu theo thời gian thực,

		      Người nhận nhận được tin nhắn theo thời gian thực, đúng nhóm trò chuyện, đúng người gửi

		      & OK

		      \\ \hline

	      \end{xltabular}
\end{enumerate}

\subsubsection{Kiểm thử ứng dụng web}
\clearpage
\begin{xltabular}{\textwidth}{
	| >{\raggedright\arraybackslash}p{2cm}
	| >{\raggedright\arraybackslash}X
	| >{\raggedright\arraybackslash}X
	| >{\raggedright\arraybackslash}p{1cm}|
	}
	\caption{\bfseries \fontsize{12pt}{0pt}\selectfont Bảng kiểm thử chức năng của website}
	% \label{table_api_news}
	\\
	\hline
	\bfseries Test case    &\bfseries Tái hiện
	&\bfseries Kết quả mong muốn &\bfseries Đánh giá\\ \hline


	Kiểm tra chức năng đăng nhập
	& 1. Truy cập giao diện đăng nhập tài khoản $\rightarrow$ Nhập thông tin đăng nhập không hợp lệ
	$\rightarrow$ Nhấn nút đăng nhập

	% \\ &

	2. Truy cập giao diện đăng nhập tài khoản $\rightarrow$ Nhập thông tin đăng nhập hợp lệ
	$\rightarrow$ Nhấn nút đăng nhập
	&

	1. Thông báo lỗi: Email đăng nhập hoặc mật khẩu truy cập không đúng


	2. Thông báo đăng nhập thành công vào hệ thống
	& OK

	\\ \hline


	Kiểm tra chức năng đăng ký
	&

	1. Truy cập giao diện tạo tài khoản người dùng mới $\rightarrow$ Điền thông tin tương ứng với tài khoản đã tồn tại
	$\rightarrow$ Nhấn đăng ký.

	% \\ &

	2. Truy cập giao diện tạo tài khoản người dùng mới $\rightarrow$ Điền hợp lệ các thông tin tài khoản
	$\rightarrow$ Nhấn đăng ký.

	&


	1. Hiển thị thông báo email đã tồn tại hoặc sai thông tin đăng ký.

	2. Hiển thị thông báo tài khoản đã được tạo mới thành công.

	& OK

	\\ \hline
	Kiểm tra chức năng quản lý người dùng hệ thống
	&

	1. Truy cập giao diện quản lý các người dùng hiện có.

	2. Tra cứu thông tin chi tiết của người dùng cụ thể.

	3. Sửa thông tin chi tiết của một người dùng  $\rightarrow$ Lưu thay đổi.

	4. Xóa người dùng ra khỏi hệ thống.

	&

	1. Hiển thị danh sách toàn bộ người dùng hiện có.

	2. Hiển thị thông tin chi tiết của người dùng đã chọn.

	3. Chỉnh sửa dữ liệu của người dùng thành công.

	4. Xóa người dùng ra khỏi hệ thống thành công.

	& OK

	\\ \hline

	Kiểm tra chức năng xem thông tin các bác sĩ
	&

	1. Truy cập giao diện thông tin các bác sĩ.

	2. Tra cứu thông tin chi tiết của bác sĩ cụ thể.

	&

	1. Hiển thị danh sách các bác sĩ trong hệ thống.

	2. Hiển thị thông tin chi tiết của bác sĩ đã chọn.


	& OK

	\\ \hline

	Kiểm tra chức năng xem danh sách bệnh nhân
	&

	1. Truy cập giao diện quản lý bệnh nhân.

	2. Tra cứu thông tin chi tiết của bệnh nhân cụ thể.

	&

	1. Hiển thị danh sách các bệnh nhân trong hệ thống.

	2. Hiển thị thông tin chi tiết của bệnh nhân đã chọn.

	& OK

	\\ \hline


	Kiểm tra chức năng quản lý thiết bị y tế
	&

	1. Truy cập giao diện quản lý thiết bị y tế.

	2. Tra cứu thông tin chi tiết thiết bị y tế cụ thể.

	3. Thêm thiết bị mới cùng các thông tin tương ứng $\rightarrow$ Lưu thông tin.

	4. Cập nhật thông tin cụ thể của thiết bị $\rightarrow$ Lưu thay đổi.

	5. Xóa thiết bị (Đối với Quản trị viên hệ thống).

	&

	1. Hiển thị danh sách thiết bị y tế trong hệ thống.

	2. Hiển thị thông tin chi tiết thiết bị y tế đã chọn.

	3. Thêm thiết bị mới thành công.

	4. Cập nhật thông tin cụ thể của thiết bị thành công.

	5. Thành công xóa thiết bị ra khỏi hệ thống.

	& OK

	\\ \hline

	Kiểm tra chức năng quản lý lịch khám cá nhân
	&

	1. Truy cập giao diện quản lý lịch khám cá nhân.

	2. Tra cứu thông tin chi tiết lịch khám cụ thể.

	3. Đặt lịch khám theo yêu cầu (chọn bác sĩ trước hoặc thời gian rảnh của bản thân trước).

	4. Lưu.

	&

	1. Hiển thị giao diện các lịch khám cá nhân, bao gồm những ngày trống và những ngày có lịch khám.

	2. Hiển thị thông tin chi tiết lịch khám đã chọn.

	3. Hiển thị giao diện đặt lịch khám theo yêu cầu.

	4. Lịch khám được đặt thành công, gửi thông báo đến cả bác sĩ và bệnh nhân.

	& OK

	\\ \hline

	Kiểm tra chức năng phê duyệt lịch khám của bác sĩ.
	&

	1. Vào giao diện quản lý lịch khám hoặc xem thông báo.

	2. Chọn lịch khám cần phê duyệt.

	3. Phê duyệt lịch khám.

	&

	1. Hiển thị giao diện các lịch khám, bao gồm những ngày trống và những ngày có lịch khám.

	2. Hiển thị thông tin chi tiết lịch khám đã chọn.

	3. Lịch khám được chấp nhận hoặc từ chối kèm theo lý do (nếu có), gửi thông báo đến cả bác sĩ và bệnh nhân.

	& OK

	\\ \hline

	Kiểm tra chức năng quản lý chẩn đoán và đặt lịch tái khám
	&

	1. Chọn 1 lịch khám đã hoặc đang diễn ra.

	2. Chọn chỉnh sửa chẩn đoán.

	3. Chọn lịch tái khám (nếu cần thiết).

	4. Điền những thông tin cần thiết và ấn lưu.

	&

	1. Hiển thị giao diện thông tin chi tiết lịch khám đã chọn.

	2. Hiển thị giao diện cho phép điền hoặc chỉnh sửa thông tin chẩn đoán (nếu đã có chẩn đoán).

	3. Hiển thị thông tin các ca khám rảnh, cho phép bác sĩ chọn lịch phù hợp.

	4. Thông tin chẩn đoán được lưu và lịch tái khám được tạo thành công (nếu có).

	& OK

	\\ \hline

	Kiểm tra chức năng quản lý thông tin cá nhân
	&
	1. Truy cập giao diện quản lý thông tin cá nhân.

	2. Sửa thông tin cá nhân có sẵn $\rightarrow$ Lưu thông tin.

	&

	1. Hiển thị các thông tin cụ thể của người dùng.

	2. Cập nhật thông tin tài khoản thành công.


	& OK


	\\ \hline
	Kiểm tra chức năng quản lý dịch vụ tin nhắn
	&
	1. Truy cập giao diện quản lý dịch vụ tin nhắn.

	2. Chọn người dùng hoặc nhóm muốn nhắn.

	3. Tiến hành nhắn tin và ấn nút gửi.

	&

	1. Hiển thị danh sách những đoạn hội thoại.

	2. Hiển thị đầy đủ lịch sử trò chuyện của đoạn hội thoại được chọn.

	3. Tin nhắn được gửi thành công.

	& OK

	\\ \hline
	Kiểm tra chức năng tạo nhóm trò chuyện
	&
	1. Truy cập giao diện quản lý dịch vụ tin nhắn.

	2. Chọn tạo nhóm trò chuyện.

	3. Nhập các thông tin cần thiết và ấn lưu.

	&

	1. Hiển thị danh sách các đoạn hội thoại và nút tạo nhóm.

	2. Hiển thị giao diện tạo nhóm trò chuyện.

	3. Nhóm trò chuyện được tạo thành công.

	& OK


	\\ \hline

\end{xltabular}


\subsection{Kết luận chương}
Chương 4 cung cấp cái nhìn chi tiết về quá trình triển khai hệ thống, bao gồm các công nghệ và nền tảng được áp dụng. Ngoài ra, chương này cũng đề cập đến quá trình kiểm thử, bao gồm kiểm thử API và ứng dụng web,
nhằm đánh giá hiệu suất và độ chính xác của hệ thống. Việc kiểm thử kỹ lưỡng trước khi đưa vào hoạt động giúp đảm bảo hệ thống vận hành ổn định và đáp ứng các yêu cầu chất lượng đặt ra.
\newpage
