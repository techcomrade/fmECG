
\section*{CHƯƠNG 1. THU THẬP YÊU CẦU}
\setcounter{section}{1}
\setcounter{subsection}{0} %LƯU Ý MỖI LẦN THÊM CHƯƠNG MỚI CẦN THÊM CÂU NÀY ĐỂ RESET THỨ TỰ CỦA SUBSECTON VỀ 1
\setcounter{table}{0} % LƯU Ý SAU MỖI LẦN GỌI BẢNG HAY HÌNH ẢNH PHẢI THÊM CÂU NÀY ĐỂ RESET THỨ TỰ
\setcounter{figure}{0} %% LƯU Ý SAU MỖI LẦN GỌI BẢNG HAY HÌNH ẢNH PHẢI THÊM CÂU NÀY ĐỂ RESET THỨ TỰ
\addcontentsline{toc}{section}{\numberline{}CHƯƠNG 1. THU THẬP YÊU CẦU}
Chương này sẽ tiến hành thu thập yêu cầu cho dự án đề tài "Hệ thống quản lý dữ liệu điện tim và tương tác giữa bênh nhân - bác sĩ"
dựa trên các mục tiêu đã nêu ra trong Mục Đề xuất hệ thống ở Phần mở đầu.

\subsection{Yêu cầu hệ thống}
\subsubsection{Yêu cầu về người dùng hệ thống}
Hệ thống được thiết kế để phục vụ các đối tượng sau:
\begin{adjustwidth}{1.5em}{}
\begin{itemize}
    \item Bệnh nhân: Hệ thống giúp bệnh nhân theo dõi dữ liệu điện tim (ECG), quản lý lịch khám, và tìm kiếm bác sĩ phù hợp để thăm khám hoặc tư vấn y tế.
    Tính năng nhắn tin trực tiếp với bác sĩ mang đến sự thuận tiện trong giao tiếp. Nhờ đó, bệnh nhân dễ dàng quản lý sức khỏe một cách hiệu quả, an toàn và toàn diện.
    
    \item Bác sĩ: Hệ thống hỗ trợ bác sĩ quản lý danh sách bệnh nhân, truy cập dữ liệu đo điện tim, và cập nhật thông tin sức khỏe.
    Tính năng đặt lịch tái khám, trao đổi trực tiếp với bệnh nhân, và tham gia nhóm thảo luận giúp bác sĩ làm việc hiệu quả hơn, đồng thời nâng cao chất lượng tư vấn và phối hợp điều trị.
    
    \item Quản trị viên: Quản trị viên chịu trách nhiệm quản lý thông tin tài khoản của bệnh nhân, bác sĩ và các thiết bị y tế, đồng thời giám sát lịch khám và dữ liệu đo lường.
    Họ đảm bảo hệ thống vận hành chính xác, hiệu quả và đáp ứng nhu cầu của mọi người dùng.
    
\end{itemize}
\end{adjustwidth}

\subsubsection{Yêu cầu chức năng}
Các chức năng chính của hệ thống bao gồm:
\begin{adjustwidth}{1.5em}{}
  \begin{itemize}
      \item Ghi dữ liệu đo từ thiết bị: Ứng dụng trên điện thoại thu thập dữ liệu đo từ thiết bị điện tim thông qua kết nối Bluetooth. Các thông số đo được ghi lại dưới dạng bảng biểu chi tiết và lưu trữ trên máy chủ của hệ thống để đảm bảo tính an toàn, sẵn sàng cho các nhu cầu sử dụng tiếp theo.
      \item Hiển thị dữ liệu: Các số liệu được lưu trữ trên hệ thống máy chủ được xử lý và tính toán thông qua các thuật toán chuyên biệt. Kết quả được hiển thị trực quan trên ứng dụng hoặc website dưới dạng biểu đồ đường, giúp bác sĩ dễ dàng theo dõi và phân tích.
      \item Lưu trữ: Hệ thống hỗ trợ lưu trữ dữ liệu đo lường từ thiết bị không chỉ trên ứng dụng mà còn trên máy chủ. Quá trình đồng bộ hóa giúp truyền tải thông tin từ ứng dụng lên máy chủ, đảm bảo dữ liệu điện tim được lưu trữ an toàn, bảo mật và có thể truy cập mọi lúc. Điều này mang lại độ tin cậy cao và đảm bảo khả năng sử dụng lâu dài cho các mục đích phân tích hoặc theo dõi sức khỏe trong tương lai.
      \item Trao đổi và chia sẻ thông tin về dữ liệu y tế: Hệ thống cung cấp nền tảng để người dùng dễ dàng trao đổi trực tiếp với bác sĩ, chia sẻ kết quả đo điện tim, đặt câu hỏi về sức khỏe hoặc thảo luận về các vấn đề liên quan đến thiết bị. Những tính năng này không chỉ mang lại sự tiện lợi mà còn đóng vai trò quan trọng trong việc hỗ trợ người dùng hiểu rõ hơn về tình trạng sức khỏe và đưa ra các quyết định phù hợp.

  \end{itemize}
\end{adjustwidth}
% Hệ thống hỗ trợ các chức năng cơ bản sau đối với người dùng:
\textbf{Đối với bệnh nhân:}
\begin{adjustwidth}{1.5em}{}
\begin{itemize}
    \item Đăng nhập và đăng ký tài khoản bằng thông tin cá nhân, bao gồm tên, địa chỉ email, ngày sinh, số điện thoại và mật khẩu.
    \item Cập nhật các thông tin cá nhân
    \item Được theo dõi điện tim trực tiếp khi kết nối ứng dụng di động với thiết bị đo điện tim thông qua Bluetooth.
    \item Xem kết quả các phiên đo của mình, bao gồm biểu đồ và các thông số liên quan.
    \item Nhận thông báo và có thể trao đổi trực tiếp với bác sĩ về tình hình sức khoẻ và các kết quả đo được từ thiết bị.
\end{itemize}
\end{adjustwidth}
\textbf{Đối với bác sĩ:}
\begin{adjustwidth}{1.5em}{}
\begin{itemize}
    \item Đăng nhập và đăng ký tài khoản bằng thông tin cá nhân, bao gồm tên, địa chỉ email, ngày sinh, số điện thoại và mật khẩu.
    \item Cập nhật thông tin cá nhân.
    \item Quản lý danh sách bệnh nhân.
    \item Quản lý danh sách các bản ghi dữ liệu đo của các bệnh nhân mà bác sĩ đó đã chấp nhận yêu cầu đặt lịch hẹn.
    \item Nhận thông báo và có thể trao đổi trực tiếp với bệnh nhân về tình hình sức khoẻ và các kết quả đo được từ thiết bị.
\end{itemize}
\end{adjustwidth}
\textbf{Đối với quản trị viên:}
\begin{adjustwidth}{1.5em}{}
\begin{itemize}
    \item Đăng nhập và đăng ký tài khoản bằng thông tin cá nhân, bao gồm tên, địa chỉ email, số điện thoại và mật khẩu.
    \item Cập nhật thông tin cá nhân.
    \item Quản lý danh sách người dùng trong hệ thống, bao gồm bệnh nhân và bác sĩ.
    \item Quản lý danh sách thiết bị, chịu trách nhiệm phân công thiết bị cho người dùng.
    \item Quản lí các bản ghi đã đo được.
    \item Quản lý lịch hẹn của toàn bộ người dùng.
\end{itemize}
\end{adjustwidth}

\subsubsection{Yêu cầu phi chức năng}
\begin{itemize}
    \item Hệ thống có thể tương thích với hầu hết các trình duyệt phổ biến hiện nay.
    \item Hệ thống đảm bảo tính bảo mật và quyền riêng tư thông tin của người dùng.
    \item Hệ thống phải có giao diện người dùng thân thiện, dễ sử dụng để có thể tương tác mà không gặp quá nhiều khó khăn.
    \item Thời gian phản hồi của hệ thống phải nhanh chóng và ổn định.
    \item Hệ thống cần được tối ưu hóa để hoạt động hiệu quả ngay cả khi có lưu lượng truy cập cao.
\end{itemize}

Quá trình phân tích yêu cầu hệ thống giúp xác định rõ ràng các chức năng, yêu cầu phi chức năng và đối tượng người dùng mà hệ thống cần phục vụ. 
Dựa trên nền tảng này, việc thiết kế và phát triển hệ thống quản lý dữ liệu điện tim sẽ được tiến hành, đảm bảo đáp ứng đầy đủ mọi yêu cầu của người dùng, đồng thời 
tối ưu hóa hiệu suất, bảo mật và tính khả dụng của hệ thống.
% \subsection{Phân tích tổng quan hệ thống}

\newpage