
\section*{CHƯƠNG 3. THIẾT KẾ HỆ THỐNG}
\setcounter{section}{3}
\setcounter{subsection}{0} %LƯU Ý MỖI LẦN THÊM CHƯƠNG MỚI CẦN THÊM CÂU NÀY ĐỂ RESET THỨ TỰ CỦA SUBSECTON VỀ 1
\setcounter{table}{0} % LƯU Ý SAU MỖI LẦN GỌI BẢNG HAY HÌNH ẢNH PHẢI THÊM CÂU NÀY ĐỂ RESET THỨ TỰ
\setcounter{figure}{0} %% LƯU Ý SAU MỖI LẦN GỌI BẢNG HAY HÌNH ẢNH PHẢI THÊM CÂU NÀY ĐỂ RESET THỨ TỰ
\addcontentsline{toc}{section}{\numberline{}CHƯƠNG 3. THIẾT KẾ HỆ THỐNG}

Chương này sẽ tập trung mô tả quá trình thiết kế hệ thống, từ cấu trúc tổng thể đến các thành phần chi tiết, dựa trên những phân tích đã thực hiện trong Chương 2.
Mở đầu là xây dựng sơ đồ kiến trúc hệ thống.
Tiếp theo, chương tập trung vào thiết kế giao diện người dùng và các chức năng chính cho website cùng server.
Nội dung chính được thể hiện qua hình ảnh và sơ đồ minh họa, không chỉ mô tả chi tiết luồng hoạt động mà còn làm rõ cách các thành phần trong hệ thống phối hợp, hỗ trợ lẫn nhau.
\subsection{Sơ đồ tổng quan kiến trúc của hệ thống}

\begin{figure}[H]
	\centering
	\includegraphics[width=12cm,height=16cm]{Images/System/fmECG_architecture-System_Architecture.drawio.png}
	\caption[Tổng quan kiến trúc hệ thống]{\bfseries \fontsize{12pt}{0pt}\selectfont Tổng quan kiến trúc hệ thống}
	\label{fmECG_architecture-System} %đặt tên cho ảnh
\end{figure}
Hệ thống bao gồm ba thành phần chính: Máy chủ (Server), Thiết bị (Device) và Ứng dụng (web và app).
Mỗi thành phần đảm nhiệm một vai trò quan trọng, cùng phối hợp để đảm bảo hoạt động của toàn bộ hệ thống như được minh họa trong hình vẽ.

Hình \ref{fmECG_architecture-System} thể hiện ba phần:

\begin{adjustwidth}{1.5em}{}
	\begin{itemize}
		\item Device (Thiết bị): Gồm các thiết bị phần cứng đo các dữ liệu sức khỏe tim mạch, có khả năng kết nối với ứng dụng di động của bệnh nhân thông qua Bluetooth.
		\item Application (Ứng dụng): Bao gồm ứng dụng di động và website, phục vụ nhu cầu sử dụng của bệnh nhân, bác sĩ và quản trị viên.
		\item Server (Máy chủ): Chứa các dịch vụ (Services) xử lý yêu cầu từ ứng dụng và quản lý cơ sở dữ liệu.
	\end{itemize}
\end{adjustwidth}

Trong sơ đồ kiến trúc hệ thống, bệnh nhân sử dụng ứng dụng di động (Mobile App) để kết nối trực tiếp với thiết bị (Device).
Ứng dụng di động sẽ thuộc khối Ứng dụng (Application). Nó chịu trách nhiệm cho việc giao tiếp với khối Máy chủ (hay Server) thông qua giao thức HTTP và các API.
Khi yêu cầu được nhận từ khối Ứng dụng, khối Máy chủ sẽ thực hiện xử lý dữ liệu thông qua các dịch vụ (Services) được thiết kế riêng biệt.
Tùy theo loại yêu cầu, các dịch vụ này sẽ truy xuất hoặc cập nhật dữ liệu trong cơ sở dữ liệu, sau đó gửi kết quả phản hồi cho người dùng,
hoàn thiện quá trình tương tác giữa người dùng và hệ thống.\begin{adjustwidth}{1.5em}{}
	\begin{itemize}
		\item Authen Service: Đảm nhận các nhiệm vụ liên quan đến bảo mật hệ thống, bao gồm mã hóa dữ liệu nhạy cảm, tạo và xác minh token để đảm bảo tính an toàn khi truy cập,
		      quản lý phân quyền người dùng đối với API,và thực hiện mã hóa thông tin trước khi lưu trữ nhằm ngăn chặn rò rỉ dữ liệu.
		\item User Service: Xử lý toàn bộ các thao tác liên quan đến người dùng, như tạo tài khoản mới, xác minh thông tin đăng nhập, lấy thông tin cá nhân của người dùng,
		      đồng thời hỗ trợ cập nhật và chỉnh sửa thông tin cá nhân khi cần thiết.
		\item Device Service: Chịu trách nhiệm quản lý thiết bị, bao gồm các chức năng như thêm mới, chỉnh sửa thông tin, xóa thiết bị,
		      và cập nhật tình trạng thiết bị hoặc thông số liên quan để đảm bảo thiết bị hoạt động đúng trong hệ thống.
		\item Storage Service: Quản lý và vận hành hệ thống lưu trữ dữ liệu, bao gồm lưu trữ file, tài liệu, và các thông tin quan trọng của hệ thống.
		      Đồng thời, đảm bảo tính nhất quán của dữ liệu thông qua cơ chế xử lý race condition, khóa truy cập và đồng bộ hóa dữ liệu trong các trường hợp truy cập đồng thời,
		      cũng như tối ưu hóa hiệu suất lưu trữ và truy xuất dữ liệu.
		\item Record Service: Xử lý các dữ liệu liên quan đến phiên đo, bao gồm thêm mới, cập nhật, xóa dữ liệu, và xử lý các file đo được từ thiết bị trước khi lưu trữ hoặc gửi đến người dùng.
		\item Message Service: Quản lý toàn bộ các yêu cầu liên quan đến nhắn tin, bao gồm gửi, nhận, lưu trữ tin nhắn, hỗ trợ các nhóm trò chuyện giữa những người dùng trong hệ thống.
		\item Schedule Service: Đảm nhiệm việc quản lý lịch khám, bao gồm đặt lịch và xử lý các phản hồi liên quan, nhằm đảm bảo quy trình đặt lịch diễn ra mượt mà giữa bệnh nhân, bác sĩ và hệ thống.
		\item Diagnosis Service: Quản lý các tác vụ liên quan đến chẩn đoán, bao gồm thêm mới và chỉnh sửa thông tin chẩn đoán cho bệnh nhân, đảm bảo dữ liệu chính xác và hỗ trợ quá trình điều trị hiệu quả.
		\item Notification Service: Đảm bảo quản lý hiệu quả các thông báo cho người dùng, từ việc gửi thông báo sự kiện, cảnh báo, đến nhắc nhở,
		      giúp người dùng cập nhật kịp thời các thông tin quan trọng từ hệ thống.
		\item Statistic Service: Chịu trách nhiệm tổng hợp số liệu thống kê trong hệ thống, bao gồm số lượng người dùng (bệnh nhân và bác sĩ),
		      số thiết bị, và dữ liệu phiên đo theo từng tháng, nhằm cung cấp thông tin phục vụ quản lý hiệu quả.
	\end{itemize}
\end{adjustwidth}

Phần dưới đây trình bày chi tiết về từng thành phần trong kiến trúc hệ thống, được hình thành dựa trên những nhóm người dùng đã được xác định trước đó.
\newpage
\subsection{Sơ đồ khối phần mềm}

\subsubsection{Website dành cho bệnh nhân}
\begin{figure}[H]
	\centering
	\includegraphics[width=12cm,height=15cm]{Images/System/fmECG_architecture-Patient.drawio.png}
	\caption[Sơ đồ khối Website dành cho bệnh nhân]{\bfseries \fontsize{12pt}{0pt}\selectfont Sơ đồ khối Website dành cho bệnh nhân}
	\label{fmECG_architecture-Patient} %đặt tên cho ảnh
\end{figure}
Tầng trên cùng trong sơ đồ hình \ref{fmECG_architecture-Patient} là User Interface (Giao diện người dùng), nơi bệnh nhân trực tiếp tương tác với hệ thống thông qua API Adapter để gửi yêu cầu và nhận phản hồi.
Các yêu cầu này được xử lý bởi các Services chính, bao gồm User Service, Device Service, Record Service, Schedule Service, Diagnosis Service, Notification Service và Storage Service.

Những Services này được thiết kế nhằm đáp ứng các nhu cầu của bệnh nhân, từ quản lý thông tin cá nhân, mượn và trả thiết bị, theo dõi lịch sử dữ liệu phiên đo, cho đến việc quản lý lịch khám,
tra cứu thông tin chẩn đoán cho từng lịch khám. Ngoài ra, hệ thống còn đảm bảo việc gửi thông báo nhắc nhở kịp thời và hỗ trợ bệnh nhân trao đổi thông tin với bác sĩ một cách liền mạch và hiệu quả.

\subsubsection{Website dành cho bác sĩ}
\begin{figure}[H]
	\centering
	\includegraphics[width=12cm,height=15cm]{Images/System/fmECG_architecture-Doctors.drawio.png}
	\caption[Sơ đồ khối Website dành cho bác sĩ]{\bfseries \fontsize{12pt}{0pt}\selectfont Sơ đồ khối Website dành cho bác sĩ}
	\label{fmECG_architecture-Doctor} %đặt tên cho ảnh
\end{figure}

Tương tự với bệnh nhân, sơ đồ hình \ref{fmECG_architecture-Doctor} được xây dựng để hỗ trợ bác sĩ thực hiện các nhiệm vụ chuyên môn thông qua giao diện người dùng và API Adapter, đảm bảo việc xử lý thông tin diễn ra nhanh chóng và hiệu quả.

Các Services chính, bao gồm User Service, Device Service, Record Service, Schedule Service, Diagnosis Service, Notification Service và Storage Service, đóng vai trò quan trọng trong việc hỗ trợ bác sĩ.
Các nhiệm vụ được tập trung vào việc theo dõi và phân tích dữ liệu phiên đo từ bệnh nhân, quản lý lịch khám, chấp nhận hoặc từ chối yêu cầu đặt lịch, ghi nhận thông tin chẩn đoán, và trao đổi trực tiếp với bệnh nhân.
Hơn nữa, hệ thống cung cấp khả năng tự động gửi thông báo về lịch khám và nhắc nhở các lịch khám sắp tới, giúp bác sĩ quản lý thời gian và công việc hiệu quả hơn.

Cách tiếp cận này không chỉ hỗ trợ bác sĩ tổ chức công việc thuận lợi mà còn góp phần nâng cao hiệu quả điều trị cho bệnh nhân.

\subsubsection{Website cho quản trị viên}
\begin{figure}[H]
	\centering
	\includegraphics[width=12cm,height=15cm]{Images/System/fmECG_architecture-Admin.drawio.png}
	\caption[Sơ đồ khối Website dành cho quản trị viên]{\bfseries \fontsize{12pt}{0pt}\selectfont Sơ đồ khối Website dành cho quản trị viên}
	\label{fmECG_architecture-Admin} %đặt tên cho ảnh
\end{figure}
Về cơ bản, website dành cho quản trị viên được thiết kế với cấu trúc tương tự như website dành cho bác sĩ và bệnh nhân nhưng với quyền hạn mở rộng hơn.
Quản trị viên có thể quản lý toàn bộ thông tin người dùng, dữ liệu phiên đo, theo dõi các thống kê tổng quan của hệ thống, kiểm soát thông tin và kết quả chẩn đoán từ các lịch khám,
đồng thời điều hành việc mượn trả thiết bị một cách linh hoạt và hiệu quả.

\subsection{Thiết kế cơ sở dữ liệu}

\subsubsection{Chuyển đổi từ mô hình thực thể liên kết sang mô hình quan hệ}
Dựa trên bảng mô tả các thực thể và thuộc tính, chúng em tiến hành chuyển đổi từ mô hình thực thể liên kết thành mô hình quan hệ như sau.

\begin{itemize}
	\item Tài khoản đăng nhập (\textbf{ID tài khoản đăng nhập}, Địa chỉ email đăng ký, Mật khẩu truy cập)
	\item Token đăng nhập (\textbf{ID token đăng nhập}, ID tài khoản đăng nhập, Token làm mới, Hạn sử dụng, Trạng thái token)
	\item Vai trò người dùng (\textbf{ID vai trò}, Tên vai trò)
	\item Trạng thái hoạt động (\textbf{ID trạng thái người dùng}, Mô tả trạng thái người dùng)
	\item Người dùng (\textbf{ID người dùng}, ID tài khoản đăng nhập, ID vai trò người dùng, ID trạng thái hoạt động, Tên đầy đủ, Ngày tháng năm sinh, Giới tính, Số liên lạc, Đường dẫn ảnh đại diện, Thông tin bổ sung)
	\item Loại thiết bị (\textbf{ID loại thiết bị}, Tên loại thiết bị)
	\item Trạng thái thiết bị (\textbf{ID trạng thái thiết bị}, Mô tả trạng thái thiết bị)
	\item Thiết bị (\textbf{ID thiết bị}, ID người dùng thiết bị, ID loại thiết bị, ID trạng thái thiết bị, Tên thiết bị, Thông tin chi tiết về thiết bị, Ngày bắt đầu thời gian mượn, Ngày kết thúc thời gian mượn)
	\item Thông số kỹ thuật (\textbf{ID thông số kỹ thuật}, ID thiết bị, Loại thông số, Tên thông số, Giá trị thông số, Mô tả chi tiết thông số)
	\item Dữ liệu phiên đo (\textbf{ID dữ liệu phiên đo}, ID bệnh nhân, ID thiết bị, Loại bản ghi, Đường dẫn lưu trữ dữ liệu phiên đo, Thời gian bắt đầu thu thập dữ liệu, Thời gian kết thúc thu thập dữ liệu)
	\item Trạng thái lịch khám (\textbf{ID trạng thái lịch khám}, Mô tả trạng thái lịch khám)
	\item Kết quả lịch khám (\textbf{ID kết quả lịch khám}, Mô tả kết quả lịch khám)
	\item Lịch khám (\textbf{ID lịch khám}, ID bệnh nhân, ID bác sĩ, ID trạng thái lịch khám, ID kết quả lịch khám, Thời gian bắt đầu lịch khám, Thời gian kết thúc lịch khám)
	\item Thông báo liên quan đến lịch khám (\textbf{ID thông báo}, ID lịch khám, Loại thông báo, Nội dung thông báo, Trạng thái thông báo, Trạng thái đã xem, Lý do từ chối lịch khám)
	\item Chẩn đoán (\textbf{ID chẩn đoán}, ID lịch khám, Thông tin chẩn đoán)
	\item Tin nhắn (\textbf{ID tin nhắn}, ID người gửi, ID nhóm trò chuyện nhận tin nhắn, Nội dung tin nhắn, Thời gian gửi tin nhắn)
	\item Nhóm trò chuyện (\textbf{ID nhóm trò chuyện}, Tên nhóm trò chuyện, Người tạo nhóm, Danh sách thành viên nhóm, Sự kiện gửi tin nhắn, Sự kiện nhận tin nhắn)

\end{itemize}

\subsubsection{Chuẩn hoá 3NF}
Các bảng đã được thiết kế theo nguyên tắc chuẩn hoá 3NF, vì không có thuộc tính lặp lại và các thuộc tính không phụ thuộc vào một tập hợp con của khóa chính.

\paragraph{Chuẩn hoá bảng Tài khoản}
\mbox{}
\begin{table}[H]
	\caption{\bfseries \fontsize{12pt}{0pt}\selectfont Bảng chuẩn hoá bảng Tài khoản đăng nhập}
	\centering
	\begin{tabularx}{0.9\textwidth}{|X|X|}
		\hline
		\textbf{Danh sách thuộc tính} & ID tài khoản đăng nhập, Địa chỉ email đăng ký, Mật khẩu truy cập                                   \\
		\hline
		\textbf{Quy tắc nghiệp vụ}    & \textbf{Phụ thuộc hàm}                                                                             \\
		\hline
		Mỗi tài khoản có một ID riêng, có duy nhất Địa chỉ email đăng ký, Mật khẩu truy cập
		                              & \parbox[t]{\linewidth}{$\text{ID tài khoản} \rightarrow$ Địa chỉ email đăng ký, Mật khẩu truy cập} \\
		\hline
		\multicolumn{2}{|X|}{$\Rightarrow \text{Khoá chính của bảng: ID tài khoản đăng nhập}$}                                             \\
		\multicolumn{2}{|X|}{$\Rightarrow \text{Bảng Tài khoản đăng nhập đã ở 3NF}$}                                                       \\
		\hline
	\end{tabularx}
\end{table}

\paragraph{Chuẩn hoá bảng Token đăng nhập}
\mbox{}
\begin{table}[H]
	\caption{\bfseries \fontsize{12pt}{0pt}\selectfont Bảng chuẩn hoá bảng Token đăng nhập}
	\centering
	\begin{tabularx}{0.9\textwidth}{|X|X|}
		\hline
		\textbf{Danh sách thuộc tính} & ID token đăng nhập, ID tài khoản đăng nhập, Token làm mới, Hạn sử dụng, Trạng thái token                         \\
		\hline
		\textbf{Quy tắc nghiệp vụ}    & \textbf{Phụ thuộc hàm}                                                                                           \\
		\hline
		Mỗi tài khoản có một ID token riêng, có duy nhất ID tài khoản, Token làm mới, Hạn sử dụng, Trạng thái token
		                              & \parbox[t]{\linewidth}{$\text{ID token} \rightarrow$ ID tài khoản, Token làm mới, Hạn sử dụng, Trạng thái token} \\
		\hline
		\multicolumn{2}{|X|}{$\Rightarrow \text{Khoá chính của bảng: ID token đăng nhập}$}                                                               \\
		\multicolumn{2}{|X|}{$\Rightarrow \text{Bảng Token đăng nhập đã ở 3NF}$}                                                                         \\
		\hline
	\end{tabularx}
\end{table}

\cleardoublepage

\paragraph{Chuẩn hoá bảng Vai trò người dùng}
\mbox{}
\begin{table}[H]
	\caption{\bfseries \fontsize{12pt}{0pt}\selectfont Bảng chuẩn hoá bảng Vai trò người dùng}
	\centering
	\begin{tabularx}{0.9\textwidth}{|X|X|}
		\hline
		\textbf{Danh sách thuộc tính} & ID vai trò, Tên vai trò                                             \\
		\hline
		\textbf{Quy tắc nghiệp vụ}    & \textbf{Phụ thuộc hàm}                                              \\
		\hline
		Mỗi vai trò có một ID riêng, có duy nhất Tên vai trò
		                              & \parbox[t]{\linewidth}{$\text{ID vai trò} \rightarrow$ Tên vai trò} \\
		\hline
		\multicolumn{2}{|X|}{$\Rightarrow \text{Khoá chính của bảng: ID vai trò}$}                          \\
		\multicolumn{2}{|X|}{$\Rightarrow \text{Bảng Vai trò người dùng đã ở 3NF}$}                         \\
		\hline
	\end{tabularx}
\end{table}

\paragraph{Chuẩn hoá bảng Trạng thái hoạt động}
\mbox{}
\begin{table}[H]
	\caption{\bfseries \fontsize{12pt}{0pt}\selectfont Bảng chuẩn hoá bảng Trạng thái hoạt động}
	\centering
	\begin{tabularx}{0.9\textwidth}{|X|X|}
		\hline
		\textbf{Danh sách thuộc tính} & ID trạng thái người dùng, Mô tả trạng thái người dùng                                             \\
		\hline
		\textbf{Quy tắc nghiệp vụ}    & \textbf{Phụ thuộc hàm}                                                                            \\
		\hline
		Mỗi trạng thái người dùng có một ID riêng, có duy nhất Mô tả trạng thái người dùng
		                              & \parbox[t]{\linewidth}{$\text{ID trạng thái người dùng} \rightarrow$ Mô tả trạng thái người dùng} \\
		\hline
		\multicolumn{2}{|X|}{$\Rightarrow \text{Khoá chính của bảng: ID trạng thái người dùng}$}                                          \\
		\multicolumn{2}{|X|}{$\Rightarrow \text{Bảng Trạng thái hoạt động đã ở 3NF}$}                                                     \\
		\hline
	\end{tabularx}
\end{table}

\paragraph{Chuẩn hoá bảng Người dùng}
\mbox{}
\begin{table}[H]
	\caption{\bfseries \fontsize{12pt}{0pt}\selectfont Bảng chuẩn hoá bảng Người dùng}
	\centering
	\begin{tabularx}{0.9\textwidth}{|X|X|}
		\hline
		\textbf{Danh sách thuộc tính}                                                                                                                                                                                          & ID người dùng, ID tài khoản đăng nhập, ID vai trò người dùng, ID trạng thái hoạt động, Tên đầy đủ, Ngày tháng năm sinh, Giới tính, Số liên lạc, Đường dẫn ảnh đại diện, Thông tin bổ sung                                             \\
		\hline
		\textbf{Quy tắc nghiệp vụ}                                                                                                                                                                                             & \textbf{Phụ thuộc hàm}                                                                                                                                                                                                                \\
		\hline
		Mỗi người dùng có một ID riêng, có duy nhất ID tài khoản đăng nhập, ID vai trò người dùng, ID trạng thái hoạt động, Tên đầy đủ, Ngày tháng năm sinh, Giới tính, Số liên lạc, Đường dẫn ảnh đại diện, Thông tin bổ sung & \parbox[t]{\linewidth}{$\text{ID người dùng} \rightarrow$ ID tài khoản đăng nhập, ID vai trò người dùng, ID trạng thái hoạt động, Tên đầy đủ, Ngày tháng năm sinh, Giới tính, Số liên lạc, Đường dẫn ảnh đại diện, Thông tin bổ sung} \\
		\hline
		\multicolumn{2}{|X|}{$\Rightarrow \text{Khoá chính của bảng: ID người dùng}$}                                                                                                                                                                                                                                                                                                                                                                                  \\
		\multicolumn{2}{|X|}{$\Rightarrow \text{Bảng Người dùng đã ở 3NF}$}                                                                                                                                                                                                                                                                                                                                                                                            \\
		\hline
	\end{tabularx}
\end{table}

\cleardoublepage

\paragraph{Chuẩn hoá bảng Loại thiết bị}
\mbox{}
\begin{table}[H]
	\caption{\bfseries \fontsize{12pt}{0pt}\selectfont Bảng chuẩn hoá bảng Loại thiết bị}
	\centering
	\begin{tabularx}{0.9\textwidth}{|X|X|}
		\hline
		\textbf{Danh sách thuộc tính} & ID loại thiết bị, Tên loại thiết bị                                             \\
		\hline
		\textbf{Quy tắc nghiệp vụ}    & \textbf{Phụ thuộc hàm}                                                          \\
		\hline
		Mỗi loại thiết bị có một ID riêng, có duy nhất Tên loại thiết bị
		                              & \parbox[t]{\linewidth}{$\text{ID loại thiết bị} \rightarrow$ Tên loại thiết bị} \\
		\hline
		\multicolumn{2}{|X|}{$\Rightarrow \text{Khoá chính của bảng: ID loại thiết bị}$}                                \\
		\multicolumn{2}{|X|}{$\Rightarrow \text{Bảng loại thiết bị đã ở 3NF}$}                                          \\
		\hline
	\end{tabularx}
\end{table}

\paragraph{Chuẩn hoá bảng Trạng thái thiết bị}
\mbox{}
\begin{table}[H]
	\caption{\bfseries \fontsize{12pt}{0pt}\selectfont Bảng chuẩn hoá bảng Trạng thái thiết bị}
	\centering
	\begin{tabularx}{0.9\textwidth}{|X|X|}
		\hline
		\textbf{Danh sách thuộc tính} & ID trạng thái thiết bị, Mô tả trạng thái thiết bị                                             \\
		\hline
		\textbf{Quy tắc nghiệp vụ}    & \textbf{Phụ thuộc hàm}                                                                        \\
		\hline
		Mỗi trạng thái thiết bị có một ID riêng, có duy nhất Mô tả trạng thái thiết bị
		                              & \parbox[t]{\linewidth}{$\text{ID trạng thái thiết bị} \rightarrow$ Mô tả trạng thái thiết bị} \\
		\hline
		\multicolumn{2}{|X|}{$\Rightarrow \text{Khoá chính của bảng: ID trạng thái thiết bị}$}                                        \\
		\multicolumn{2}{|X|}{$\Rightarrow \text{Bảng Trạng thái thiết bị đã ở 3NF}$}                                                  \\
		\hline
	\end{tabularx}
\end{table}

\paragraph{Chuẩn hoá bảng Thiết bị}
\mbox{}
\begin{table}[H]
	\caption{\bfseries \fontsize{12pt}{0pt}\selectfont Bảng chuẩn hoá bảng Thiết bị}
	\centering
	\begin{tabularx}{0.9\textwidth}{|X|X|}
		\hline
		\textbf{Danh sách thuộc tính} & ID thiết bị, ID người dùng thiết bị, Tên thiết bị, Thông tin chi tiết về thiết bị, Ngày bắt đầu thời gian mượn, Ngày kết thúc thời gian mượn \\
		\hline
		\textbf{Quy tắc nghiệp vụ}    & \textbf{Phụ thuộc hàm}                                                                                                                       \\
		\hline
		Mỗi thiết bị có một ID thiết bị riêng, có duy nhất tên thiết bị, loại thiết bị, thông tin thiết bị,
		ID người dùng thiết bị, trạng thái thiết bị, ngày bắt đầu sử dụng, ngày kết thúc sử dụng
		                              & \parbox[t]{\linewidth}{$\text{ID thiết bị} \rightarrow$ ID người dùng thiết bị, Tên thiết bị,
		Loại thiết bị, Thông tin thiết bị, Trạng thái thiết bị, Ngày bắt đầu sử dụng, Ngày kết thúc sử dụng}                                                                         \\
		\hline
		\multicolumn{2}{|X|}{$\Rightarrow \text{Khoá chính của bảng: ID thiết bị}$}                                                                                                  \\
		\multicolumn{2}{|X|}{$\Rightarrow \text{Bảng Thiết bị đã ở 3NF}$}                                                                                                            \\
		\hline
	\end{tabularx}
\end{table}

\cleardoublepage

\paragraph{Chuẩn hoá bảng Thông số kỹ thuật}
\mbox{}
\begin{table}[H]
	\caption{\bfseries \fontsize{12pt}{0pt}\selectfont Bảng chuẩn hoá bảng Thông số kỹ thuật}
	\centering
	\begin{tabularx}{0.9\textwidth}{|X|X|}
		\hline
		\textbf{Danh sách thuộc tính} & ID thông số kỹ thuật, ID thiết bị, Loại thông số, Tên thông số, Giá trị thông số, Mô tả chi tiết thông số                                             \\
		\hline
		\textbf{Quy tắc nghiệp vụ}    & \textbf{Phụ thuộc hàm}                                                                                                                                \\
		\hline
		Mỗi thông số kỹ thuật sẽ có một ID riêng, có duy nhất ID thiết bị, Loại thông số, Tên thông số, Giá trị thông số, Mô tả chi tiết thông số
		                              & \parbox[t]{\linewidth}{$\text{ID thông số kỹ thuật} \rightarrow$ ID thiết bị, Loại thông số, Tên thông số, Giá trị thông số, Mô tả chi tiết thông số} \\
		\hline
		\multicolumn{2}{|X|}{$\Rightarrow \text{Khoá chính của bảng: ID thông số kỹ thuật}$}                                                                                                  \\
		\multicolumn{2}{|X|}{$\Rightarrow \text{Bảng Thông số kỹ thuật đã ở 3NF}$}                                                                                                            \\
		\hline
	\end{tabularx}
\end{table}

\paragraph{Chuẩn hoá bảng Dữ liệu phiên đo}
\mbox{}
\begin{table}[H]
	\caption{\bfseries \fontsize{12pt}{0pt}\selectfont Bảng chuẩn hoá bảng Dữ liệu phiên đo}
	\centering
	\begin{tabularx}{0.9\textwidth}{|X|X|}
		\hline
		\textbf{Danh sách thuộc tính} & ID dữ liệu phiên đo, ID bệnh nhân, ID thiết bị, Loại bản ghi, Đường dẫn lưu trữ dữ liệu phiên đo, Thời gian bắt đầu thu thập dữ liệu, Thời gian kết thúc thu thập dữ liệu                                             \\
		\hline
		\textbf{Quy tắc nghiệp vụ}    & \textbf{Phụ thuộc hàm}                                                                                                                                                                                                \\
		\hline
		Mỗi dữ liệu phiên đo có một ID riêng, có duy nhất ID bệnh nhân, ID thiết bị, Loại bản ghi, Đường dẫn lưu trữ dữ liệu phiên đo, Thời gian bắt đầu thu thập dữ liệu, Thời gian kết thúc thu thập dữ liệu
		                              & \parbox[t]{\linewidth}{$\text{ID dữ liệu phiên đo} \rightarrow$ ID bệnh nhân, ID thiết bị, Loại bản ghi, Đường dẫn lưu trữ dữ liệu phiên đo, Thời gian bắt đầu thu thập dữ liệu, Thời gian kết thúc thu thập dữ liệu} \\
		\hline
		\multicolumn{2}{|X|}{$\Rightarrow \text{Khoá chính của bảng: ID dữ liệu phiên đo}$}                                                                                                                                                                   \\
		\multicolumn{2}{|X|}{$\Rightarrow \text{Bảng Dữ liệu phiên đo đã ở 3NF}$}                                                                                                                                                                             \\
		\hline
	\end{tabularx}
\end{table}

\paragraph{Chuẩn hoá bảng Trạng thái lịch khám}
\mbox{}
\begin{table}[H]
	\caption{\bfseries \fontsize{12pt}{0pt}\selectfont Bảng chuẩn hoá bảng Trạng thái lịch khám}
	\centering
	\begin{tabularx}{0.9\textwidth}{|X|X|}
		\hline
		\textbf{Danh sách thuộc tính} & ID trạng thái lịch khám, Mô tả trạng thái lịch khám                                             \\
		\hline
		\textbf{Quy tắc nghiệp vụ}    & \textbf{Phụ thuộc hàm}                                                                          \\
		\hline
		Mỗi trạng thái lịch khám có một ID riêng, có duy nhất Mô tả trạng thái lịch khám
		                              & \parbox[t]{\linewidth}{$\text{ID trạng thái lịch khám} \rightarrow$ Mô tả trạng thái lịch khám} \\
		\hline
		\multicolumn{2}{|X|}{$\Rightarrow \text{Khoá chính của bảng: ID trạng thái lịch khám}$}                                         \\
		\multicolumn{2}{|X|}{$\Rightarrow \text{Bảng Trạng thái lịch khám đã ở 3NF}$}                                                   \\
		\hline
	\end{tabularx}
\end{table}

\cleardoublepage

\paragraph{Chuẩn hoá bảng Kết quả lịch khám}
\mbox{}
\begin{table}[H]
	\caption{\bfseries \fontsize{12pt}{0pt}\selectfont Bảng chuẩn hoá bảng Kết quả lịch khám}
	\centering
	\begin{tabularx}{0.9\textwidth}{|X|X|}
		\hline
		\textbf{Danh sách thuộc tính} & ID kết quả lịch khám, Mô tả kết quả lịch khám                                             \\
		\hline
		\textbf{Quy tắc nghiệp vụ}    & \textbf{Phụ thuộc hàm}                                                                    \\
		\hline
		Mỗi Kết quả lịch khám có một ID riêng, có duy nhất Mô tả kết quả lịch khám
		                              & \parbox[t]{\linewidth}{$\text{ID kết quả lịch khám} \rightarrow$ Mô tả kết quả lịch khám} \\
		\hline
		\multicolumn{2}{|X|}{$\Rightarrow \text{Khoá chính của bảng: ID kết quả lịch khám}$}                                      \\
		\multicolumn{2}{|X|}{$\Rightarrow \text{Bảng Kết quả lịch khám đã ở 3NF}$}                                                \\
		\hline
	\end{tabularx}
\end{table}

\paragraph{Chuẩn hoá bảng Lịch khám}
\mbox{}
\begin{table}[H]
	\caption{\bfseries \fontsize{12pt}{0pt}\selectfont Bảng chuẩn hoá bảng Lịch khám}
	\centering
	\begin{tabularx}{0.9\textwidth}{|X|X|}
		\hline
		\textbf{Danh sách thuộc tính} & ID lịch khám, ID bệnh nhân, ID bác sĩ, ID trạng thái lịch khám, ID kết quả lịch khám, Thời gian bắt đầu lịch khám, Thời gian kết thúc lịch khám                                             \\
		\hline
		\textbf{Quy tắc nghiệp vụ}    & \textbf{Phụ thuộc hàm}                                                                                                                                                                      \\
		\hline
		Mỗi Lịch khám có một ID riêng, có duy nhất ID bệnh nhân, ID bác sĩ, ID trạng thái lịch khám, ID kết quả lịch khám, Thời gian bắt đầu lịch khám, Thời gian kết thúc lịch khám
		                              & \parbox[t]{\linewidth}{$\text{ID lịch khám} \rightarrow$ ID bệnh nhân, ID bác sĩ, ID trạng thái lịch khám, ID kết quả lịch khám, Thời gian bắt đầu lịch khám, Thời gian kết thúc lịch khám} \\
		\hline
		\multicolumn{2}{|X|}{$\Rightarrow \text{Khoá chính của bảng: ID  lịch khám}$}                                                                                                                                               \\
		\multicolumn{2}{|X|}{$\Rightarrow \text{Bảng Lịch khám đã ở 3NF}$}                                                                                                                                                          \\
		\hline
	\end{tabularx}
\end{table}

\paragraph{Chuẩn hoá bảng Thông báo liên quan đến lịch khám}
\mbox{}
\begin{table}[H]
	\caption{\bfseries \fontsize{12pt}{0pt}\selectfont Bảng chuẩn hoá bảng Thông báo liên quan đến lịch khám}
	\centering
	\begin{tabularx}{0.9\textwidth}{|X|X|}
		\hline
		\textbf{Danh sách thuộc tính} & ID thông báo, ID lịch khám, Loại thông báo, Nội dung thông báo, Trạng thái thông báo, Trạng thái đã xem, Lý do từ chối lịch khám                                             \\
		\hline
		\textbf{Quy tắc nghiệp vụ}    & \textbf{Phụ thuộc hàm}                                                                                                                                                       \\
		\hline
		Mỗi Thông báo liên quan đến lịch khám có một ID riêng, có duy nhất ID lịch khám, Loại thông báo, Nội dung thông báo, Trạng thái thông báo, Trạng thái đã xem, Lý do từ chối lịch khám
		                              & \parbox[t]{\linewidth}{$\text{ID thông báo} \rightarrow$ ID lịch khám, Loại thông báo, Nội dung thông báo, Trạng thái thông báo, Trạng thái đã xem, Lý do từ chối lịch khám} \\
		\hline
		\multicolumn{2}{|X|}{$\Rightarrow \text{Khoá chính của bảng: ID  thông báo}$}                                                                                                                                \\
		\multicolumn{2}{|X|}{$\Rightarrow \text{Bảng Thông báo liên quan đến lịch khám đã ở 3NF}$}                                                                                                                   \\
		\hline
	\end{tabularx}
\end{table}

\paragraph{Chuẩn hoá bảng Chẩn đoán}
\mbox{}
\begin{table}[H]
	\caption{\bfseries \fontsize{12pt}{0pt}\selectfont Bảng chuẩn hoá bảng Chẩn đoán}
	\centering
	\begin{tabularx}{0.9\textwidth}{|X|X|}
		\hline
		\textbf{Danh sách thuộc tính} & ID chẩn đoán, ID lịch khám, Thông tin chẩn đoán                                             \\
		\hline
		\textbf{Quy tắc nghiệp vụ}    & \textbf{Phụ thuộc hàm}                                                                      \\
		\hline
		Mỗi Chẩn đoán có một ID riêng, có duy nhất ID lịch khám, Thông tin chẩn đoán
		                              & \parbox[t]{\linewidth}{$\text{ID chẩn đoán} \rightarrow$ ID lịch khám, Thông tin chẩn đoán} \\
		\hline
		\multicolumn{2}{|X|}{$\Rightarrow \text{Khoá chính của bảng: ID  chẩn đoán}$}                                               \\
		\multicolumn{2}{|X|}{$\Rightarrow \text{Bảng Chẩn đoán đã ở 3NF}$}                                                          \\
		\hline
	\end{tabularx}
\end{table}

\paragraph{Chuẩn hoá bảng Tin nhắn}
\mbox{}
\begin{table}[H]
	\caption{\bfseries \fontsize{12pt}{0pt}\selectfont Bảng chuẩn hoá bảng Tin nhắn}
	\centering
	\begin{tabularx}{0.9\textwidth}{|X|X|}
		\hline
		\textbf{Danh sách thuộc tính} & ID tin nhắn, ID người gửi, ID nhóm trò chuyện nhận tin nhắn, Nội dung tin nhắn, Thời gian gửi tin nhắn                                             \\
		\hline
		\textbf{Quy tắc nghiệp vụ}    & \textbf{Phụ thuộc hàm}                                                                                                                             \\
		\hline
		Mỗi Tin nhắn có một ID riêng, có duy nhất ID người gửi, ID nhóm trò chuyện nhận tin nhắn, Nội dung tin nhắn, Thời gian gửi tin nhắn
		                              & \parbox[t]{\linewidth}{$\text{ID tin nhắn} \rightarrow$ ID người gửi, ID nhóm trò chuyện nhận tin nhắn, Nội dung tin nhắn, Thời gian gửi tin nhắn} \\
		\hline
		\multicolumn{2}{|X|}{$\Rightarrow \text{Khoá chính của bảng: ID  tin nhắn}$}                                                                                                       \\
		\multicolumn{2}{|X|}{$\Rightarrow \text{Bảng Tin nhắn đã ở 3NF}$}                                                                                                                  \\
		\hline
	\end{tabularx}
\end{table}

\paragraph{Chuẩn hoá bảng Nhóm trò chuyện}
\mbox{}
\begin{table}[H]
	\caption{\bfseries \fontsize{12pt}{0pt}\selectfont Bảng chuẩn hoá bảng Nhóm trò chuyện}
	\centering
	\begin{tabularx}{0.9\textwidth}{|X|X|}
		\hline
		\textbf{Danh sách thuộc tính} & ID nhóm trò chuyện, Tên nhóm trò chuyện, Người tạo nhóm, Danh sách thành viên nhóm, Sự kiện gửi tin nhắn, Sự kiện nhận tin nhắn                                             \\
		\hline
		\textbf{Quy tắc nghiệp vụ}    & \textbf{Phụ thuộc hàm}                                                                                                                                                      \\
		\hline
		Mỗi Nhóm trò chuyện có một ID riêng, có duy nhất Tên nhóm trò chuyện, Người tạo nhóm, Danh sách thành viên nhóm, Sự kiện gửi tin nhắn, Sự kiện nhận tin nhắn
		                              & \parbox[t]{\linewidth}{$\text{ID nhóm trò chuyện} \rightarrow$ Tên nhóm trò chuyện, Người tạo nhóm, Danh sách thành viên nhóm, Sự kiện gửi tin nhắn, Sự kiện nhận tin nhắn} \\
		\hline
		\multicolumn{2}{|X|}{$\Rightarrow \text{Khoá chính của bảng: ID  nhóm trò chuyện}$}                                                                                                                         \\
		\multicolumn{2}{|X|}{$\Rightarrow \text{Bảng Nhóm trò chuyện đã ở 3NF}$}                                                                                                                                    \\
		\hline
	\end{tabularx}
\end{table}

\subsubsection{Sơ đồ ERD}

\begin{figure}[H]
	\centering
	\includegraphics[width=16cm]{Images/System/ERD-db-Page-2.drawio.png}
	\caption[Sơ đồ ERD]{\bfseries \fontsize{12pt}{0pt}\selectfont Sơ đồ ERD}
	\label{fmECG_architecture-Database}
\end{figure}

\subsection{Thiết kế giao diện}
Sau khi thiết kế cơ sở dữ liệu của hệ thống, chúng em tiến đến bước tiếp theo của việc thiết kế hệ thống là thiết kế giao diện người dùng. Giao diện người dùng của hệ thống sẽ được thiết kế dựa trên các yêu cầu của người dùng, đảm bảo giao diện người dùng thân thiện, dễ sử dụng và đáp ứng được nhu cầu của người dùng.
\subsubsection{Giao diện trang giới thiệu về chúng tôi}
\begin{figure}[H]
	\centering
	\includegraphics[width=15cm,height=8cm]{Images/about_us/about_us_1.png}
	\caption[Giao diện trang giới thiệu về chúng tôi]{\bfseries \fontsize{12pt}{0pt}\selectfont Giao diện trang giới thiệu về chúng tôi}
	\label{}
\end{figure}
\begin{figure}[H]
	\centering
	\includegraphics[width=15cm,height=8cm]{Images/about_us/about_us_2.png}
	\caption[Giao diện trang giới thiệu về chúng tôi]{\bfseries \fontsize{12pt}{0pt}\selectfont Giao diện trang giới thiệu về chúng tôi}
	\label{}
\end{figure}
\begin{figure}[H]
	\centering
	\includegraphics[width=15cm,height=8cm]{Images/about_us/about_us_3.png}
	\caption[Giao diện trang giới thiệu về chúng tôi]{\bfseries \fontsize{12pt}{0pt}\selectfont Giao diện trang giới thiệu về chúng tôi}
	\label{}
\end{figure}
\begin{figure}[H]
	\centering
	\includegraphics[width=15cm,height=8cm]{Images/about_us/about_us_4.png}
	\caption[Giao diện trang giới thiệu về chúng tôi]{\bfseries \fontsize{12pt}{0pt}\selectfont Giao diện trang giới thiệu về chúng tôi}
	\label{}
\end{figure}
\begin{figure}[H]
	\centering
	\includegraphics[width=15cm,height=8cm]{Images/about_us/about_us_5.png}
	\caption[Giao diện trang giới thiệu về chúng tôi]{\bfseries \fontsize{12pt}{0pt}\selectfont Giao diện trang giới thiệu về chúng tôi}
	\label{about-us}
\end{figure}

Hình \ref{about-us} minh họa giao diện trang giới thiệu về chúng tôi, nơi cung cấp thông tin về nhóm phát triển, bao gồm mục tiêu,
định hướng hoạt động, thông tin liên hệ, và các công nghệ chủ chốt được sử dụng để tạo ra sản phẩm này.

\subsubsection{Giao diện góc nhìn của quản trị viên}
\begin{figure}[H]
	\centering
	\includegraphics[width=15cm,height=8cm]{Images/admin_ui/home.png}
	\caption[Giao diện trang chủ quản trị viên]{\bfseries \fontsize{12pt}{0pt}\selectfont Giao diện trang chủ quản trị viên}
	\label{home-admin}
\end{figure}

Hình \ref{home-admin} mô tả giao diện trang chủ dành cho quản trị viên. Sau khi đăng nhập với quyền quản trị, người dùng sẽ được đưa đến màn hình tổng quan,
nơi tích hợp các thống kê quan trọng, biểu đồ trực quan, và báo cáo chi tiết. Giao diện này được tối ưu hóa để giúp quản trị viên quản lý thông tin hiệu quả,
theo dõi tiến độ công việc, và truy cập nhanh chóng vào các tính năng chính, đảm bảo việc điều hành hệ thống diễn ra mượt mà và chính xác.

\begin{figure}[H]
	\centering
	\includegraphics[width=15cm,height=8cm]{Images/admin_ui/users.png}
	\caption[Giao diện trang quản lý người dùng]{\bfseries \fontsize{12pt}{0pt}\selectfont Giao diện trang quản lý người dùng}
	\label{users-admin}
\end{figure}

Hình \ref{users-admin} là giao diện trang quản lý người dùng. Giao diện này cho phép quản trị viên thực hiện các thao tác như thêm mới,
chỉnh sửa, hoặc xóa thông tin người dùng, đồng thời cung cấp khả năng xem chi tiết hồ sơ của từng người dùng một cách dễ dàng và nhanh chóng.

\begin{figure}[H]
	\centering
	\includegraphics[width=15cm,height=8cm]{Images/admin_ui/user-info.png}
	\caption[Giao diện cửa sổ ngăn kéo hiển thị thông tin chi tiết của người dùng]{\bfseries \fontsize{12pt}{0pt}\selectfont Giao diện cửa sổ ngăn kéo hiển thị thông tin chi tiết của người dùng}
	\label{user-info-admin}
\end{figure}

Hình \ref{user-info-admin} minh họa giao diện cửa sổ ngăn kéo hiển thị thông tin chi tiết của người dùng. Giao diện này
cho phép quản trị viên xem đầy đủ thông tin cá nhân và các chi tiết liên quan của người dùng một cách rõ ràng và tiện lợi.

\begin{figure}[H]
	\centering
	\includegraphics[width=15cm,height=8cm]{Images/admin_ui/user-detail.png}
	\caption[Giao diện cửa sổ pop up chỉnh sửa thông tin của người dùng]{\bfseries \fontsize{12pt}{0pt}\selectfont Giao diện cửa sổ pop up chỉnh sửa thông tin của người dùng}
	\label{user-detail-admin}
\end{figure}

Hình \ref{user-detail-admin} minh họa giao diện cửa sổ pop up cho phép người dùng có thể cập nhân thông tin một cách đơn giản và nhanh chóng.

\begin{figure}[H]
	\centering
	\includegraphics[width=15cm,height=8cm]{Images/admin_ui/devices.png}
	\caption[Giao diện trang quản lý thiết bị]{\bfseries \fontsize{12pt}{0pt}\selectfont Giao diện trang quản lý thiết bị}
	\label{devices-admin}
\end{figure}

Hình \ref{devices-admin} là giao diện trang quản lý thiết bị. Quản trị viên có thể thêm, sửa, xóa thông tin thiết bị, cũng như xem thông tin chi tiết của thiết bị khi click vào 1 thiết bị bất kỳ.

\begin{figure}[H]
	\centering
	\includegraphics[width=15cm,height=8cm]{Images/admin_ui/device-info.png}
	\caption[Giao diện trang thông tin thiết bị]{\bfseries \fontsize{12pt}{0pt}\selectfont Giao diện trang thông tin thiết bị}
	\label{device-info-admin}
\end{figure}

Hình \ref{device-info-admin} là giao diện cửa sổ trượt khi click vào 1 thiết bị bất kỳ nói trên. Nội dung chi tiết hơn về thông số thiết bị được hiển thị ở giao diện này. Quản trị viên có thể xem nhanh các số liệu cần thiết.

\begin{figure}[H]
	\centering
	\includegraphics[width=15cm,height=8cm]{Images/admin_ui/device-detail.png}
	\caption[Giao diện trang thông số kỹ thuật]{\bfseries \fontsize{12pt}{0pt}\selectfont Giao diện trang thông số kỹ thuật}
	\label{device-detail-admin}
\end{figure}

Hình \ref{device-detail-admin} là giao diện trang sửa thông tin, thông số kỹ thuật của thiết bị. Nhập các thông tin cần thiết, chọn các giá trị từ các danh sách giá trị phù hợp để chỉnh sửa thông tin.

\begin{figure}[H]
	\centering
	\includegraphics[width=10cm]{Images/admin_ui/assign.png}
	\caption[Giao diện trang quản lý cho mượn thiết bị]{\bfseries \fontsize{12pt}{0pt}\selectfont Giao diện trang quản lý cho mượn thiết bị}
	\label{assign-admin}
\end{figure}

Hình \ref{assign-admin} là giao diện minh họa giao diện phân công thiết bị, hỗ trợ quản trị viên trong việc quản lý người dùng nào đang phụ trách thiết bị.

\begin{figure}[H]
	\centering
	\includegraphics[width=15cm,height=8cm]{Images/admin_ui/records.png}
	\caption[Giao diện trang dữ liệu phiên đo]{\bfseries \fontsize{12pt}{0pt}\selectfont Giao diện trang dữ liệu phiên đo}
	\label{records-admin}
\end{figure}

Hình \ref{records-admin} là giao diện quản lý dữ liệu phiên đo tại các buổi khám. Liên kết tới dữ liệu các phiên đo được lưu tại đây, quản trị viên và bác sĩ có thể sử dụng cho việc phân tích sau này.

\begin{figure}[H]
	\centering
	\includegraphics[width=15cm]{Images/admin_ui/graph.jpg}
	\caption[Giao diện đồ thị chỉ số tim mạch]{\bfseries \fontsize{12pt}{0pt}\selectfont Giao diện đồ thị chỉ số tim mạch}
	\label{graph}
\end{figure}

Hình \ref{graph} là giao diện thể hiện đồ thị các tín hiệu tim mạch thu thập được từ dữ liệu phiên đo.
Trong đó ECG, PPG, và PCG là các thuật ngữ liên quan đến các phương pháp đo tín hiệu sinh lý của cơ thể, thường được sử dụng trong y học và khoa học để giám sát sức khỏe.
ECG (Electrocardiogram) - Điện tâm đồ là đường màu xanh lá, mô tả tín hiệu đo hoạt động điện của tim theo thời gian.
PPG (Photoplethysmography) - Quang phổ thể tích là đường màu xanh đỏ, mô tả sự thay đổi thể tích máu trong mạch máu bằng cách sử dụng ánh sáng.
Và PCG (Phonocardiography) - Âm thanh tim, đường màu xanh dương, mô tả tín hiệu ghi lại âm thanh được tạo ra bởi hoạt động của tim, thường gọi là âm thanh tim \cite{ecg_pcg_ppg}.
\begin{figure}[H]
	\centering
	\includegraphics[width=15cm,height=8cm]{Images/admin_ui/schedule.png}
	\caption[Giao diện trang quản lý đặt lịch]{\bfseries \fontsize{12pt}{0pt}\selectfont Giao diện trang quản lý đặt lịch (góc nhìn của quản trị viên)}
	\label{schedules-admin}
\end{figure}

Hình \ref{schedules-admin} là giao diện giúp quản trị viên có thể tra cứu thông tin đặt lịch, từ đó có thể lấy dữ liệu để thống kê cho các mục đích sau này.
\begin{figure}[H]
	\centering
	\includegraphics[width=12cm]{Images/admin_ui/schedule-info.png}
	\caption[Giao diện trang thông tin đặt lịch]{\bfseries \fontsize{12pt}{0pt}\selectfont Giao diện trang thông tin đặt lịch}
	\label{schedule-info-admin}
\end{figure}

Hình \ref{schedule-info-admin} là giao diện trang thông tin đặt lịch, hỗ trợ người dùng có thể xem chi tiết tên bác sĩ, bệnh nhân, thời gian khám. Giao diện này cũng sẽ dùng cho bệnh nhân và bác sĩ.

\begin{figure}[H]
	\centering
	\includegraphics[width=12cm]{Images/admin_ui/diagnosis.png}
	\caption[Giao diện chẩn đoán]{\bfseries \fontsize{12pt}{0pt}\selectfont Giao diện chẩn đoán}
	\label{diag-admin}
\end{figure}

Hình \ref{diag-admin} minh họa giao diện chẩn đoán, nơi quản trị viên hoặc bệnh nhân có thể xem thông tin chi tiết về chẩn đoán, nhưng không có quyền thêm mới hay chỉnh sửa thông tin chẩn đoán.

\begin{figure}[H]
	\centering
	\includegraphics[width=15cm,height=8cm]{Images/admin_ui/chat.png}
	\caption[Giao diện trang nhắn tin]{\bfseries \fontsize{12pt}{0pt}\selectfont Giao diện trang nhắn tin}
	\label{chat-admin}
\end{figure}

Hình \ref{chat-admin} minh họa giao diện trang nhắn tin, hỗ trợ quản trị viên nhắn tin và tạo nhóm với người dùng khác để thuận tiện trong việc trao đổi, chia sẻ thông tin và hướng dẫn sử dụng hệ thống.
Ngoài ra, bác sĩ cũng có quyền tạo nhóm và nhắn tin với các bệnh nhân đã được chấp nhận lịch khám.
Tuy nhiên, bệnh nhân chỉ có thể nhắn tin với quản trị viên và các bác sĩ đã chấp nhận lịch khám của mình, mà không có quyền tạo nhóm.

\begin{figure}[H]
	\centering
	\includegraphics[width=15cm,height=8cm]{Images/admin_ui/account.png}
	\caption[Giao diện trang quản lý tài khoản]{\bfseries \fontsize{12pt}{0pt}\selectfont Giao diện trang quản lý tài khoản}
	\label{account-admin}
\end{figure}

Hình \ref{account-admin} là giao diện trang quản lý tài khoản, thông tin tài khoản. Quản trị viên nói riêng hay người dùng nói chung có thể xem hoặc thay đổi thông tin về tài khoản của mình.

\subsubsection{Giao diện góc nhìn của bệnh nhân}
\begin{figure}[H]
	\centering
	\includegraphics[width=15cm,height=8cm]{Images/patient_ui/doctor-info.png}
	\caption[Giao diện trang thông tin bác sĩ]{\bfseries \fontsize{12pt}{0pt}\selectfont Giao diện trang thông tin bác sĩ}
	\label{doctor-info-patient}
\end{figure}

Hình \ref{doctor-info-patient} là giao diện trang thông tin bác sĩ giúp cho bệnh nhân có thể xem thông tin chi tiết của bác sĩ mà bệnh nhân cần tìm.

\begin{figure}[H]
	\centering
	\includegraphics[width=12cm]{Images/patient_ui/schedule-with-doctor.png}
	\caption[Giao diện đặt lịch với bác sĩ đã chọn]{\bfseries \fontsize{12pt}{0pt}\selectfont Giao diện đặt lịch với bác sĩ đã chọn}
	\label{schedule-with-doctor}
\end{figure}

Hình \ref{schedule-with-doctor} minh họa giao diện cho phép bệnh nhân lựa chọn bác sĩ mong muốn và sau đó chọn ngày giờ phù hợp để đặt lịch khám.
Từ đó, bệnh nhân có thể dễ dàng lựa chọn thời gian khám thuận tiện nhất theo bác sĩ tương ứng.

\begin{figure}[H]
	\centering
	\includegraphics[width=12cm]{Images/patient_ui/schedule-with-time.png}
	\caption[Giao diện đặt lịch với thời gian đã chọn]{\bfseries \fontsize{12pt}{0pt}\selectfont Giao diện đặt lịch với thời gian đã chọn}
	\label{schedule-with-time}
\end{figure}

Hình \ref{schedule-with-time} cũng giống như chức năng trên, tuy nhiên bệnh nhân có thể chọn bác sĩ phù hợp với thời gian mong muốn của mình.

\begin{figure}[H]
	\centering
	\includegraphics[width=8cm]{Images/patient_ui/noti.png}
	\caption[Giao diện thông báo]{\bfseries \fontsize{12pt}{0pt}\selectfont Giao diện thông báo}
	\label{noti-patient}
\end{figure}

Hình \ref{noti-patient} minh họa giao diện thông báo, nơi bệnh nhân có thể theo dõi lịch sử đặt lịch và các thông báo từ hệ thống để biết liệu lịch khám của mình đã được bác sĩ chấp nhận hay chưa.
Giao diện này cũng được áp dụng tương tự đối với bác sĩ, giúp họ cập nhật trạng thái các lịch khám của mình.

\subsubsection{Giao diện góc nhìn của bác sĩ}
\begin{figure}[H]
	\centering
	\includegraphics[width=15cm,height=8cm]{Images/doctor_ui/home.png}
	\caption[Giao diện trang chủ và thông báo của bác sĩ]{\bfseries \fontsize{12pt}{0pt}\selectfont Giao diện trang chủ và thông báo của bác sĩ}
	\label{home-doctor}
\end{figure}

Hình \ref{home-doctor} thể hiện giao diện trang chủ và thông báo của bác sĩ. Tại đây, bác sĩ có thể theo dõi các thông báo từ hệ thống để nắm bắt thông tin về ngày giờ của các lịch khám đã hoặc đang chờ phê duyệt.

\begin{figure}[H]
	\centering
	\includegraphics[width=15cm,height=8cm]{Images/doctor_ui/patient-list.png}
	\caption[Giao diện trang danh sách bệnh nhân]{\bfseries \fontsize{12pt}{0pt}\selectfont Giao diện trang danh sách bệnh nhân}
	\label{patient-list}
\end{figure}

Hình \ref{patient-list} là giao diện trang danh sách bệnh nhân, hỗ trợ bác sĩ có thể xem danh sách bệnh nhân của mình và thông tin chi tiết của từng bệnh nhân.

\begin{figure}[H]
	\centering
	\includegraphics[width=15cm,height=8cm]{Images/doctor_ui/schedule.png}
	\caption[Giao diện trang quản lý lịch khám của bác sĩ]{\bfseries \fontsize{12pt}{0pt}\selectfont Giao diện trang quản lý lịch khám của bác sĩ}
	\label{schdule-doctor}
\end{figure}

Hình \ref{schdule-doctor} minh họa giao diện trang quản lý lịch khám của bác sĩ, giúp bác sĩ xem chi tiết các lịch khám của mình trong ngày.

\begin{figure}[H]
	\centering
	\includegraphics[width=15cm,height=8cm]{Images/doctor_ui/diag.png}
	\caption[Giao diện chẩn đoán]{\bfseries \fontsize{12pt}{0pt}\selectfont Giao diện chẩn đoán}
	\label{diag}
\end{figure}

Hình \ref{diag} thể hiện giao diện chẩn đoán, nơi bác sĩ có thể xem chi tiết thông tin và thêm hoặc chỉnh sửa kết quả chẩn đoán của bệnh nhân.

\begin{figure}[H]
	\centering
	\includegraphics[width=12cm]{Images/doctor_ui/diag_reschedule.png}
	\caption[Giao diện chẩn đoán và đặt lịch tái khám]{\bfseries \fontsize{12pt}{0pt}\selectfont Giao diện chẩn đoán và đặt lịch tái khám}
	\label{diag_reschedule}
\end{figure}

Bên cạnh đó, bác sĩ cũng có thể chủ động đặt lịch tái khám cho bệnh nhân ngay trong buổi khám. Hình \ref{diag_reschedule} là giao diện chẩn đoán và đặt lịch tái khám.

\begin{figure}[H]
	\centering
	\includegraphics[width=15cm,height=8cm]{Images/doctor_ui/schedule-detail.png}
	\caption[Giao diện thông tin lịch khám]{\bfseries \fontsize{12pt}{0pt}\selectfont Giao diện thông tin lịch khám}
	\label{schedule-detail-doctor}
\end{figure}

Hình \ref{schedule-detail-doctor} là giao diện thông tin chỉ tiết của lịch khám.

\begin{figure}[H]
	\centering
	\includegraphics[width=15cm]{Images/doctor_ui/review.png}
	\caption[Giao diện duyệt lịch khám]{\bfseries \fontsize{12pt}{0pt}\selectfont Giao diện duyệt lịch khám}
	\label{review}
\end{figure}

Hình \ref{review} là giao diện duyệt lịch khám. Các bác sĩ có thể chấp nhận hoặc từ chối lịch khám của bệnh nhân, sau đó thông báo sẽ được gửi đến bệnh nhân để họ có thể sắp xếp công việc của mình cho phù hợp.

\begin{figure}[H]
	\centering
	\includegraphics[width=15cm,height=8cm]{Images/doctor_ui/accepted.png}
	\caption[Giao diện chấp nhận lịch khám]{\bfseries \fontsize{12pt}{0pt}\selectfont Giao diện chấp nhận lịch khám}
	\label{accepted}
\end{figure}

Hình \ref{accepted} là giao diện sau khi bác sĩ chấp nhận lịch khám của bệnh nhân. Một pop up sẽ hiện lên để thông báo việc xác nhận lịch khám đã thành công.


% =====================================================================
\subsection{Thiết kế các chức năng cho website và server}

\subsubsection{Thiết kế các API cần thiết}


\begin{enumerate}[a)]
	\item API xác minh tài khoản

	\begin{xltabular}{\textwidth}{
	  | >{\raggedright\arraybackslash}m{4.6cm}
	  | >{\centering\arraybackslash}m{2.8cm}
	  | >{\raggedright\arraybackslash}X |
	  }
	  \caption{\bfseries \fontsize{12pt}{0pt}\selectfont Bảng API xác minh tài khoản}
	  \label{table_api_auth}
	  \\
	  \hline
	  \bfseries Đường dẫn    &\bfseries Phương thức    &\bfseries Mô tả\\ \hline
	  api/ecg/auth/create-account   &   POST  & Tạo mới tài khoản người dùng \\ \hline
	  api/ecg/auth/login   &    POST    & Đăng nhập vào hệ thống \\ \hline
	  api/ecg/auth/logout   &    POST    & Đăng xuất khỏi hệ thống \\ \hline
	\end{xltabular}
  
	\item API quản lý người dùng trong hệ thống
	\begin{xltabular}{\textwidth}{
	  | >{\raggedright\arraybackslash}m{4.6cm}
	  | >{\centering\arraybackslash}m{2.8cm}
	  | >{\raggedright\arraybackslash}X |
	  }
	  \caption{\bfseries \fontsize{12pt}{0pt}\selectfont Bảng API quản lý người dùng trong hệ thống}
	  \label{table_api_user}
	  \\
	  \hline
	  \bfseries Đường dẫn    &\bfseries Phương thức    &\bfseries Mô tả\\ \hline
	  api/ecg/users   &   GET  &  Tra cứu danh sách tất cả người dùng trong hệ thống\\  \hline
	  api/ecg/users/doctors   &   GET  &  Tra cứu danh sách toàn bộ bác sĩ trong hệ thống \\  \hline
	  api/ecg/users/:id   &   GET  &  Tra cứu dữ liệu người dùng cụ thể dựa trên id tương ứng \\  \hline
	  api/ecg/users/data/patient-data   &   GET  &  Tra cứu danh sách các bệnh nhân đang được theo dõi của bác sĩ cụ thể \\  \hline
	  api/ecg/users/data/doctor-id   &   GET  &  Tra cứu thông tin của bác sĩ phụ trách bệnh nhân \\  \hline
	  api/ecg/users/   &   PUT  &  Chỉnh sửa thông tin người dùng \\  \hline
	  api/ecg/users/:userId  &   DELETE  &  Xóa người dùng cụ thể dựa trên id tương ứng \\  \hline
	\end{xltabular}
  
  \item API quản lý thiết bị y tế
  \begin{xltabular}{\textwidth}{
	| >{\raggedright\arraybackslash}m{4.8cm}
	| >{\centering\arraybackslash}m{2.8cm}
	| >{\raggedright\arraybackslash}X |
	}
	\caption{\bfseries \fontsize{12pt}{0pt}\selectfont Bảng API quản lý thiết bị y tế}
	\label{table_api_device}
	\\
	\hline
	\bfseries Đường dẫn    &\bfseries Phương thức    &\bfseries Mô tả\\ \hline
	api/ecg/device   &   GET  & Tra cứu danh sách toàn bộ thiết bị y tế trong hệ thống\\ \hline
	api/ecg/devide/:id   &    GET    & Tra cứu thông tin chi tiết của một thiết bị y tế dựa trên id tương ứng \\ \hline
	api/ecg/device/add &   POST     & Thêm mới thiết bị y tế \\ \hline
	api/ecg/device/update  &     PUT   & Cập nhật thông tin chi tiết cho thiết bị y tế cụ thể \\ \hline
	api/ecg/device/:deviceId  &     DELETE   & Xóa thiết bị y tế dựa trên id tương ứng \\ \hline
  \end{xltabular}
  
  \item API quản lý dữ liệu phiên đo
  \begin{xltabular}{\textwidth}{
	| >{\raggedright\arraybackslash}p{5cm}
	| >{\centering\arraybackslash}m{2.8cm}
	| >{\raggedright\arraybackslash}X |
	}
	\caption{\bfseries \fontsize{12pt}{0pt}\selectfont Bảng API quản lý dữ liệu phiên đo}
	\label{table_api_record}
	\\
	\hline
	\bfseries Đường dẫn    &\bfseries Phương thức    &\bfseries Mô tả \\ \hline
	 api/ecg/records   &   GET  & Tra cứu danh sách các dữ liệu phiên đo \\ \hline
	 api/ecg/records/:id   &    GET    & Tra cứu dữ liệu phiên đo cụ thể dựa theo id tương ứng \\ \hline
	 api/ecg/records/data/ doctor &   GET     & Tra cứu danh sách các dữ liệu phiên đo của bệnh nhân mà bác sĩ phụ trách \\ \hline
	 api/ecg/records/data/ patient &   GET     & Tra cứu danh sách dữ liệu các phiên đo cá nhân \\ \hline
	 api/ecg/records/   &    POST    & Tạo dữ liệu phiên đo mới \\ \hline
	 api/ecg/records/   &    PUT    & Cập nhật dữ liệu phiên đo \\ \hline
	 api/ecg/records/:recordId  &    DELETE    & Xóa dữ liệu phiên đo dựa theo id tương ứng \\ \hline
	\end{xltabular}
  
  
  \item API quản lý dịch vụ lịch khám
  \begin{xltabular}{\textwidth}{
	| >{\raggedright\arraybackslash}m{4.5cm}
	| >{\centering\arraybackslash}m{2.8cm}
	| >{\raggedright\arraybackslash}X |
	}
	\caption{\bfseries \fontsize{12pt}{0pt}\selectfont Bảng API quản lý dịch vụ lịch khám}
	\label{table_api_schedule}
	\\
	\hline
	\bfseries Đường dẫn    &\bfseries Phương thức    &\bfseries Mô tả\\ \hline
	api/ecg/schedules   &   GET  & Tra cứu danh sách tất cả lịch khám trong hệ thống \\ \hline
	api/ecg/schedules/ doctor-id  &    GET    & Tra cứu danh sách lịch khám của bác sĩ cụ thể \\ \hline
	api/ecg/schedules/ patient-id  &    GET    & Tra cứu danh sách lịch khám của bệnh nhân cụ thể \\ \hline
	api/ecg/schedules/create-by-doctor  &    POST    & Cho phép bác sĩ đặt lịch tái khám cho bệnh nhân \\ \hline
	api/ecg/schedules/create-by-patient  &    POST    & Cho phép bệnh nhân đặt lịch khám với bác sĩ \\ \hline
	api/ecg/schedules/time/ available-doctor/: schedule-time  &    GET    & Tra cứu danh sách các bác sĩ khả dụng theo thời gian đã chọn \\ \hline
	api/ecg/schedules/ available-schedule/:id  &    GET    & Tra cứu các khung giờ trống có thể đặt lịch với bác sĩ cụ thể. \\ \hline
	api/ecg/schedules/accept-schedule  &    PUT    & Chấp nhận lịch khám cụ thể \\ \hline
	api/ecg/schedules/reject-schedule/:id  &    DELETE    & Từ chối lịch khám cụ thể \\ \hline
	\end{xltabular}
  
  \item API liên quan đến chẩn đoán cho bệnh nhân
  \begin{xltabular}{\textwidth}{
	| >{\raggedright\arraybackslash}m{4.5cm}
	| >{\centering\arraybackslash}m{2.8cm}
	| >{\raggedright\arraybackslash}X |
	}
	\caption{\bfseries \fontsize{12pt}{0pt}\selectfont Bảng API liên quan đến chẩn đoán cho bệnh nhân}
	\label{table_api_diagnosis}
	\\
	\hline
	\bfseries Đường dẫn    &\bfseries Phương thức    &\bfseries Mô tả\\ \hline
	api/ecg/diagnosis   &   POST  & Tạo chẩn đoán mới cho bệnh nhân \\ \hline
	api/ecg/diagnosis/ schedule/:scheduleId   &   POST  & Tra cứu thông tin chẩn đoán dựa trên id lịch khám tương ứng\\ \hline
	api/ecg/diagnosis/update   &   POST  & Cập nhật thông tin chẩn đoán \\ \hline
	\end{xltabular}
  
  
  \item API liên quan đến thông báo về lịch khám
  \begin{xltabular}{\textwidth}{
	| >{\raggedright\arraybackslash}m{4.5cm}
	| >{\centering\arraybackslash}m{2.8cm}
	| >{\raggedright\arraybackslash}X |
	}
	\caption{\bfseries \fontsize{12pt}{0pt}\selectfont Bảng API liên quan đến thông báo về lịch khám}
	\label{table_api_notification}
	\\
	\hline
	\bfseries Đường dẫn    &\bfseries Phương thức    &\bfseries Mô tả\\ \hline
	api/ecg/notification/get   &   GET  & Tra cứu tất cả các thông báo của người dùng cụ thể  \\ \hline
	api/ecg/notification   &   POST  & Tạo thông báo mới liên quan đến lịch khám \\ \hline
	api/ecg/notification/ update-seen   &   POST  & Cập nhật trạng thái thông báo đã được xem\\ \hline
	api/ecg/notification/:id   &   DELETE  & Xóa thông báo dựa trên id tương ứng\\ \hline
	\end{xltabular}
  
  \item API liên quan đến tin nhắn
  \begin{xltabular}{\textwidth}{
	| >{\raggedright\arraybackslash}m{4.5cm}
	| >{\centering\arraybackslash}m{2.8cm}
	| >{\raggedright\arraybackslash}X |
	}
	\caption{\bfseries \fontsize{12pt}{0pt}\selectfont Bảng API liên quan đến tin nhắn}
	\label{table_api_chat}
	\\
	\hline
	\bfseries Đường dẫn    &\bfseries Phương thức    &\bfseries Mô tả\\ \hline
	api/ecg/groupChat   &   POST  & Tạo nhóm trò chuyện mới  \\ \hline
	api/ecg/groupChat   &   GET  & Tra cứu danh sách các nhóm trò chuyện của người dùng  \\ \hline
	api/ecg/chat/messages/: groupChatId   &   GET  & Tra cứu lịch sử trò chuyện của các đoạn hội thoại đã thực hiện \\ \hline
	api/ecg/chat/send   &   POST  & Cho phép người dùng gửi tin nhắn đến các đối tượng liên quan \\ \hline
	\end{xltabular}
\end{enumerate}


\subsubsection{Sơ đồ tuần tự API}
Phần này cung cấp các sơ đồ tuần tự minh họa chi tiết cách thức hoạt động của các API trong hệ thống.
Dựa trên bảng API đã thiết kế, các sơ đồ tuần tự sẽ mô phỏng luồng xử lý từ khi nhận yêu cầu đến khi trả về kết quả cho người dùng,
cho thấy rõ sự tương tác giữa các thành phần và lớp chức năng trong hệ thống.
% ------------------------Auth----------------------

\paragraph{Các API phục vụ mục đích xác minh tài khoản}
\mbox{}

\begin{figure}[H]
	\centering
	\includegraphics[width=16cm]{Images/api_sequence/authen/authentication-register.drawio.png}
	\caption[Sơ đồ tuần tự API tạo mới tài khoản người dùng]{\bfseries \fontsize{12pt}{0pt}\selectfont Sơ đồ tuần tự API tạo mới tài khoản người dùng}
	\label{sequence_diagram_create_account}
\end{figure}

\begin{figure}[H]
	\centering
	\includegraphics[width=16cm]{Images/api_sequence/authen/authentication-login.drawio.png}
	\caption[Sơ đồ tuần tự API đăng nhập vào hệ thống]{\bfseries \fontsize{12pt}{0pt}\selectfont Sơ đồ tuần tự API đăng nhập vào hệ thống}
	\label{sequence_diagram_login}
\end{figure}

% % ------------------------User----------------------
\clearpage
\paragraph{Các API phục vụ mục đích quản lý người dùng trong hệ thống}
\mbox{}
\begin{figure}[H]
	\centering
	\includegraphics[width=16cm]{Images/api_sequence/user/getAllUsers.drawio.png}
	\caption[Sơ đồ tuần tự API tra cứu danh sách tất cả người dùng trong hệ thống]{\bfseries \fontsize{12pt}{0pt}\selectfont Sơ đồ tuần tự API tra cứu danh sách tất cả người dùng trong hệ thống}
	\label{sequence_diagram_get_all_users}
\end{figure}

\begin{figure}[H]
	\centering
	\includegraphics[width=16cm]{Images/api_sequence/user/getAllDoctors.drawio.png}
	\caption[Sơ đồ tuần tự API tra cứu danh sách toàn bộ bác sĩ trong hệ thống]{\bfseries \fontsize{12pt}{0pt}\selectfont Sơ đồ tuần tự API tra cứu danh sách toàn bộ bác sĩ trong hệ thống}
	\label{sequence_diagram_get_all_doctors}
\end{figure}

\begin{figure}[H]
	\centering
	\includegraphics[width=16cm]{Images/api_sequence/user/getUserById.drawio.png}
	\caption[Sơ đồ tuần tự API tra cứu dữ liệu người dùng cụ thể dựa trên id tương ứng]{\bfseries \fontsize{12pt}{0pt}\selectfont Sơ đồ tuần tự API tra cứu dữ liệu người dùng cụ thể dựa trên id tương ứng}
	\label{sequence_diagram_get_user_by_id}
\end{figure}

\begin{figure}[H]
	\centering
	\includegraphics[width=16cm, height=11cm]{Images/api_sequence/user/getPatientByDoctorId.drawio.png}
	\caption[Sơ đồ tuần tự API tra cứu danh sách bệnh nhân đang được theo dõi của bác sĩ cụ thể]{\bfseries \fontsize{12pt}{0pt}\selectfont Sơ đồ tuần tự API tra cứu danh sách bệnh nhân đang được theo dõi của bác sĩ cụ thể}
	\label{sequence_diagram_get_patient_data}
\end{figure}

\begin{figure}[H]
	\centering
	\includegraphics[width=16cm]{Images/api_sequence/user/getDoctorByPatientId.drawio.png}
	\caption[Sơ đồ tuần tự API tra cứu thông tin của bác sĩ phụ trách bệnh nhân]{\bfseries \fontsize{12pt}{0pt}\selectfont Sơ đồ tuần tự API tra cứu thông tin của bác sĩ phụ trách bệnh nhân}
	\label{sequence_diagram_get_doctor_data}
\end{figure}

\begin{figure}[H]
	\centering
	\includegraphics[width=16cm]{Images/api_sequence/user/updateUserById.drawio.png}
	\caption[Sơ đồ tuần tự API chỉnh sửa thông tin người dùng]{\bfseries \fontsize{12pt}{0pt}\selectfont Sơ đồ tuần tự API chỉnh sửa thông tin người dùng}
	\label{sequence_diagram_update_user}
\end{figure}

\begin{figure}[H]
	\centering
	\includegraphics[width=16cm]{Images/api_sequence/user/deleteUserById.drawio.png}
	\caption[Sơ đồ tuần tự API xóa người dùng cụ thể dựa trên id tương ứng]{\bfseries \fontsize{12pt}{0pt}\selectfont Sơ đồ tuần tự API xóa người dùng cụ thể dựa trên id tương ứng}
	\label{sequence_diagram_delete_user}
\end{figure}

% % ------------------------Device----------------------
\paragraph{Các API phục vụ mục đích quản lý thiết bị y tế}
\mbox{}
\begin{figure}[H]
	\centering
	\includegraphics[width=16cm]{Images/api_sequence/device/device-GetAllDevice.drawio.png}
	\caption[Sơ đồ tuần tự API tra cứu danh sách toàn bộ thiết bị y tế trong hệ thống]{\bfseries \fontsize{12pt}{0pt}\selectfont Sơ đồ tuần tự API tra cứu danh sách toàn bộ thiết bị y tế trong hệ thống}
	\label{sequence_diagram_get_all_devices}
\end{figure}

\begin{figure}[H]
	\centering
	\includegraphics[width=16cm]{Images/api_sequence/device/device-Add.drawio.png}
	\caption[Sơ đồ tuần tự API thêm mới thiết bị y tế]{\bfseries \fontsize{12pt}{0pt}\selectfont Sơ đồ tuần tự API thêm mới thiết bị y tế}
	\label{sequence_diagram_add_device}
\end{figure}

\begin{figure}[H]
	\centering
	\includegraphics[width=16cm]{Images/api_sequence/device/device-GetDeviceById.drawio.png}
	\caption[Sơ đồ tuần tự API tra cứu thông tin chi tiết một thiết bị y tế dựa trên id tương ứng]{\bfseries \fontsize{12pt}{0pt}\selectfont Sơ đồ tuần tự API tra cứu thông tin chi tiết một thiết bị y tế dựa trên id tương ứng}
	\label{sequence_diagram_add_device}
\end{figure}

\begin{figure}[H]
	\centering
	\includegraphics[width=16cm]{Images/api_sequence/device/device-updateDevice.drawio.png}
	\caption[Sơ đồ tuần tự API chỉnh sửa thông tin chi tiết cho thiết bị y tế cụ thể]{\bfseries \fontsize{12pt}{0pt}\selectfont Sơ đồ tuần tự API chỉnh sửa thông tin chi tiết cho thiết bị y tế cụ thể}
	\label{sequence_diagram_update_device}
\end{figure}

\begin{figure}[H]
	\centering
	\includegraphics[width=16cm]{Images/api_sequence/device/device-Add.drawio.png}
	\caption[Sơ đồ tuần tự API xóa thiết bị y tế dựa trên id tương ứng]{\bfseries \fontsize{12pt}{0pt}\selectfont Sơ đồ tuần tự API xóa thiết bị y tế dựa trên id tương ứng}
	\label{sequence_diagram_get_device_by_id}
\end{figure}
% % ------------------------Record----------------------
\paragraph{Các API phục vụ mục đích liên quan đến dữ liệu phiên đo}
\mbox{}
\begin{figure}[H]
	\centering
	\includegraphics[width=16cm]{Images/api_sequence/record/getAllRecord.drawio.png}
	\caption[Sơ đồ tuần tự API tra cứu danh sách các dữ liệu phiên đo]{\bfseries \fontsize{12pt}{0pt}\selectfont Sơ đồ tuần tự API tra cứu danh sách các dữ liệu phiên đo}
	\label{sequence_diagram_get_all_records}
\end{figure}

\begin{figure}[H]
	\centering
	\includegraphics[width=16cm]{Images/api_sequence/record/getRecordById.drawio.png}
	\caption[Sơ đồ tuần tự API tra cứu dữ liệu phiên đo cụ thể theo id tương ứng]{\bfseries \fontsize{12pt}{0pt}\selectfont Sơ đồ tuần tự API tra cứu dữ liệu phiên đo cụ thể theo id tương ứng}
	\label{sequence_diagram_get_record_by_id}
\end{figure}

\begin{figure}[H]
	\centering
	\includegraphics[width=16cm]{Images/api_sequence/record/getRecordByDoctorId.drawio.png}
	\caption[Sơ đồ tuần tự API tra cứu danh sách các dữ liệu phiên đo của bệnh nhân mà bác sĩ phụ trách]{\bfseries \fontsize{12pt}{0pt}\selectfont Sơ đồ tuần tự API tra cứu danh sách các dữ liệu phiên đo của bệnh nhân mà bác sĩ phụ trách}
	\label{sequence_diagram_get_record_by_doctor_id}
\end{figure}

\begin{figure}[H]
	\centering
	\includegraphics[width=16cm]{Images/api_sequence/record/getRecordByPatientId.drawio.png}
	\caption[Sơ đồ tuần tự API tra cứu danh sách dữ liệu các phiên đo cá nhân]{\bfseries \fontsize{12pt}{0pt}\selectfont Sơ đồ tuần tự API tra cứu danh sách dữ liệu các phiên đo cá nhân}
	\label{sequence_diagram_get_record_by_patient_id}
\end{figure}

\begin{figure}[H]
	\centering
	\includegraphics[width=16cm]{Images/api_sequence/record/createRecord.drawio.png}
	\caption[Sơ đồ tuần tự API tạo dữ liệu phiên đo mới]{\bfseries \fontsize{12pt}{0pt}\selectfont Sơ đồ tuần tự API tạo dữ liệu phiên đo mới}
	\label{sequence_diagram_create_record}
\end{figure}

\begin{figure}[H]
	\centering
	\includegraphics[width=16cm]{Images/api_sequence/record/updateRecordById.drawio.png}
	\caption[Sơ đồ tuần tự API cập nhật dữ liệu phiên đo]{\bfseries \fontsize{12pt}{0pt}\selectfont Sơ đồ tuần tự API cập nhật dữ liệu phiên đo}
	\label{sequence_diagram_update_record}
\end{figure}

\begin{figure}[H]
	\centering
	\includegraphics[width=16cm]{Images/api_sequence/record/deleteRecordByID.drawio.png}
	\caption[Sơ đồ tuần tự API xóa dữ liệu phiên đo theo id tương ứng]{\bfseries \fontsize{12pt}{0pt}\selectfont Sơ đồ tuần tự API xóa dữ liệu phiên đo theo id tương ứng}
	\label{sequence_diagram_delete_record}
\end{figure}

% % ------------------------Schedule----------------------
\paragraph{Các API phục vụ mục đích quản lý dịch vụ lịch khám}
\mbox{}
\begin{figure}[H]
	\centering
	\includegraphics[width=16cm]{Images/api_sequence/schedule/getAllSchedules.drawio.png}
	\caption[Sơ đồ tuần tự API tra cứu danh sách tất cả lịch khám trong hệ thống]{\bfseries \fontsize{12pt}{0pt}\selectfont Sơ đồ tuần tự API tra cứu danh sách tất cả lịch khám trong hệ thống}
	\label{sequence_diagram_get_schedule}
\end{figure}

\begin{figure}[H]
	\centering
	\includegraphics[width=16cm]{Images/api_sequence/schedule/getScheduleByDoctorId.drawio.png}
	\caption[Sơ đồ tuần tự API tra cứu danh sách lịch khám của bác sĩ cụ thể]{\bfseries \fontsize{12pt}{0pt}\selectfont Sơ đồ tuần tự API tra cứu danh sách lịch khám của bác sĩ cụ thể}
	\label{sequence_diagram_get_schedule_by_doctor}
\end{figure}

\begin{figure}[H]
	\centering
	\includegraphics[width=16cm]{Images/api_sequence/schedule/getScheduleByPatientId.drawio.png}
	\caption[Sơ đồ tuần tự API tra cứu danh sách lịch khám của bệnh nhân cụ thể]{\bfseries \fontsize{12pt}{0pt}\selectfont Sơ đồ tuần tự API tra cứu danh sách lịch khám của bệnh nhân cụ thể}
	\label{sequence_diagram_get_schedule_by_patient}
\end{figure}

\begin{figure}[H]
	\centering
	\includegraphics[width=16cm]{Images/api_sequence/schedule/createScheduleByDoctor.drawio.png}
	\caption[Sơ đồ tuần tự API cho phép bác sĩ đặt lịch tái khám cho bệnh nhân]{\bfseries \fontsize{12pt}{0pt}\selectfont Sơ đồ tuần tự API cho phép bác sĩ đặt lịch tái khám cho bệnh nhân}
	\label{sequence_diagram_create_schedule_by_doctor}
\end{figure}

\begin{figure}[H]
	\centering
	\includegraphics[width=16cm]{Images/api_sequence/schedule/createScheduleByPatient.drawio.png}
	\caption[Sơ đồ tuần tự API cho phép bệnh nhân đặt lịch khám với bác sĩ]{\bfseries \fontsize{12pt}{0pt}\selectfont Sơ đồ tuần tự API cho phép bệnh nhân đặt lịch khám với bác sĩ}
	\label{sequence_diagram_create_schedule_by_patient}
\end{figure}

\begin{figure}[H]
	\centering
	\includegraphics[width=16cm]{Images/api_sequence/schedule/getAvailableScheduleByDoctorId.drawio.png}
	\caption[Sơ đồ tuần tự API tra cứu các khung giờ trống có thể đặt lịch với bác sĩ cụ thể]{\bfseries \fontsize{12pt}{0pt}\selectfont Sơ đồ tuần tự API tra cứu các khung giờ trống có thể đặt lịch với bác sĩ cụ thể}
	\label{sequence_diagram_get_available_by_doctor}
\end{figure}

\begin{figure}[H]
	\centering
	\includegraphics[width=16cm]{Images/api_sequence/schedule/getAvailableDoctorByScheduleTime.drawio.png}
	\caption[Sơ đồ tuần tự API tra cứu các bác sĩ khả dụng theo thời gian đã chọn]{\bfseries \fontsize{12pt}{0pt}\selectfont Sơ đồ tuần tự API tra cứu các bác sĩ khả dụng theo thời gian đã chọn}
	\label{sequence_diagram_get_available_with_time}
\end{figure}

\begin{figure}[H]
	\centering
	\includegraphics[width=16cm]{Images/api_sequence/schedule/acceptSchedule.drawio.png}
	\caption[Sơ đồ tuần tự API chấp nhận lịch khám cụ thể]{\bfseries \fontsize{12pt}{0pt}\selectfont Sơ đồ tuần tự API chấp nhận lịch khám cụ thể}
	\label{sequence_diagram_accept_schedule}
\end{figure}

\begin{figure}[H]
	\centering
	\includegraphics[width=16cm]{Images/api_sequence/schedule/deleteScheduleById.drawio.png}
	\caption[Sơ đồ tuần tự API từ chối lịch khám cụ thể]{\bfseries \fontsize{12pt}{0pt}\selectfont Sơ đồ tuần tự API bác sĩ từ chối lịch khám cụ thể}
	\label{sequence_diagram_reject_schedule}
\end{figure}

% % ------------------------Diag----------------------
\paragraph{Các API phục vụ mục đích liên quan đến chẩn đoán}
\mbox{}

\begin{figure}[H]
	\centering
	\includegraphics[width=16cm]{Images/api_sequence/diag/getByScheduleId.drawio.png}
	\caption[Sơ đồ tuần tự API tra cứu thông tin chẩn đoán dựa trên id lịch khám tương ứng]{\bfseries \fontsize{12pt}{0pt}\selectfont Sơ đồ tuần tự API tra cứu thông tin chẩn đoán dựa trên id lịch khám tương ứng}
	\label{sequence_diagram_get_diagnosis_by_schedule}
\end{figure}

\begin{figure}[H]
	\centering
	\includegraphics[width=16cm]{Images/api_sequence/diag/create.drawio.png}
	\caption[Sơ đồ tuần tự API tạo chẩn đoán mới cho bệnh nhân]{\bfseries \fontsize{12pt}{0pt}\selectfont Sơ đồ tuần tự API tạo chẩn đoán mới cho bệnh nhân}
	\label{sequence_diagram_create_diagnosis}
\end{figure}

\begin{figure}[H]
	\centering
	\includegraphics[width=16cm]{Images/api_sequence/diag/update.drawio.png}
	\caption[Sơ đồ tuần tự API cập nhật thông tin chẩn đoán]{\bfseries \fontsize{12pt}{0pt}\selectfont Sơ đồ tuần tự API cập nhật thông tin chẩn đoán}
	\label{sequence_diagram_update_diagnosis}
\end{figure}

% % ------------------------Noti----------------------
\cleardoublepage
\paragraph{Các API phục vụ mục đích liên quan đến thông báo}
\mbox{}

\begin{figure}[H]
	\centering
	\includegraphics[width=16cm]{Images/api_sequence/noti/getNotificationByUserId.drawio.png}
	\caption[Sơ đồ tuần tự API tra cứu tất cả thông báo của người dùng cụ thể]{\bfseries \fontsize{12pt}{0pt}\selectfont Sơ đồ tuần tự API tra cứu tất cả thông báo của người dùng cụ thể}
	\label{sequence_diagram_get_notification_by_user}
\end{figure}
\begin{figure}[H]
	\centering
	\includegraphics[width=16cm]{Images/api_sequence/noti/createNotification.drawio.png}
	\caption[Sơ đồ tuần tự API tạo thông báo mới liên quan đến lịch khám]{\bfseries \fontsize{12pt}{0pt}\selectfont Sơ đồ tuần tự API tạo thông báo mới liên quan đến lịch khám}
	\label{sequence_diagram_create_notification}
\end{figure}

\begin{figure}[H]
	\centering
	\includegraphics[width=16cm]{Images/api_sequence/noti/updateSeenStatus.drawio.png}
	\caption[Sơ đồ tuần tự API cập nhật trạng thái thông báo đã được xem]{\bfseries \fontsize{12pt}{0pt}\selectfont Sơ đồ tuần tự API cập nhật trạng thái thông báo đã được xem}
	\label{sequence_diagram_update_seen}
\end{figure}

\begin{figure}[H]
	\centering
	\includegraphics[width=16cm]{Images/api_sequence/noti/deleteNotification.drawio.png}
	\caption[Sơ đồ tuần tự API xóa thông báo dựa trên id tương ứng]{\bfseries \fontsize{12pt}{0pt}\selectfont Sơ đồ tuần tự API xóa thông báo dựa trên id tương ứng}
	\label{sequence_diagram_delete_notification}
\end{figure}
% % ------------------------Chat----------------------
\cleardoublepage
\paragraph{Các API phục vụ mục đích liên quan đến tin nhắn}
\mbox{}
\begin{figure}[H]
	\centering
	\includegraphics[width=16cm]{Images/api_sequence/chat/loadMessages.drawio.png}
	\caption[Sơ đồ tuần tự API tra cứu lịch sử trò chuyện của các đoạn hội thoại đã thực hiện]{\bfseries \fontsize{12pt}{0pt}\selectfont Sơ đồ tuần tự API tra cứu lịch sử trò chuyện của các đoạn hội thoại đã thực hiện}
	\label{sequence_diagram_load_chat}
\end{figure}

\begin{figure}[H]
	\centering
	\includegraphics[width=16cm]{Images/api_sequence/chat/sendMessage.drawio.png}
	\caption[Sơ đồ tuần tự API giúp người dùng gửi tin nhắn đến các đối tượng liên quan]{\bfseries \fontsize{12pt}{0pt}\selectfont Sơ đồ tuần tự API giúp người dùng gửi tin nhắn đến các đối tượng liên quan}
	\label{sequence_diagram_send_chat}
\end{figure}

\begin{figure}[H]
	\centering
	\includegraphics[width=16cm]{Images/api_sequence/chat/createGroupChat.drawio.png}
	\caption[Sơ đồ tuần tự API tạo nhóm trò chuyện mới]{\bfseries \fontsize{12pt}{0pt}\selectfont Sơ đồ tuần tự API tạo nhóm trò chuyện mới}
	\label{sequence_diagram_create_group_chat}
\end{figure}

\begin{figure}[H]
	\centering
	\includegraphics[width=16cm]{Images/api_sequence/chat/getGroupChat.drawio.png}
	\caption[Sơ đồ tuần tự API lấy danh sách nhóm trò chuyện của người dùng]{\bfseries \fontsize{12pt}{0pt}\selectfont Sơ đồ tuần tự API lấy danh sách nhóm trò chuyện của người dùng}
	\label{sequence_diagram_get_group_chat}
\end{figure}

Những sơ đồ tuần tự này giúp đảm bảo các API được thiết kế và triển khai đúng với mục tiêu đề ra.
Nhờ đó, hệ thống không chỉ hoạt động ổn định và hiệu quả mà còn đáp ứng linh hoạt các yêu cầu thực tiễn của người dùng,
đồng thời tạo tiền đề vững chắc cho việc mở rộng và cải tiến trong tương lai.

\subsection{Kết luận}

Chương này tập trung mô tả chi tiết quá trình thiết kế hệ thống, từ kiến trúc tổng quan đến các thành phần cụ thể.
Quá trình thiết kế hướng đến việc xây dựng một kiến trúc hoạt động hiệu quả, đảm bảo sự mượt mà trong vận hành,
đồng thời ưu tiên các yếu tố như bảo mật, hiệu suất cao và khả năng mở rộng linh hoạt.
\newpage