\section*{LỜI NÓI ĐẦU} % dấu * để không đánh sô thứ tự vào lời nói đầu
\thispagestyle{empty}
Trong thời đại công nghệ phát triển vượt bậc, Internet vạn vật (IoT) đã trở thành một phần không thể thiếu trong nhiều lĩnh vực, đặc biệt là y tế. Việc ứng dụng IoT vào quản lý thiết bị y tế đã mở ra những cơ hội mới, góp phần nâng cao chất lượng chăm sóc sức khỏe và tối ưu hóa các quy trình y khoa. Tại Việt Nam, trong bối cảnh công nghiệp hóa, hiện đại hóa đang diễn ra mạnh mẽ, việc áp dụng các công nghệ tiên tiến vào y tế không chỉ giúp cải thiện hệ thống chăm sóc sức khỏe mà còn xây dựng nền tảng vững chắc để phát triển nguồn nhân lực khỏe mạnh, sẵn sàng đóng góp cho sự phát triển của đất nước.

Đồ án này trình bày một hệ thống ứng dụng di động tích hợp trong quản lý dữ liệu tim mạch, hỗ trợ kết nối trực tiếp giữa bệnh nhân và bác sĩ. Thông qua hệ thống này, bệnh nhân có thể dễ dàng theo dõi tình trạng sức khỏe tại nhà, kết nối với các bác sĩ để nhận được sự tư vấn kịp thời và chuyên nghiệp. Giao diện thân thiện của ứng dụng cho phép người dùng đặt lịch hẹn, xem lịch sử bệnh án, theo dõi dữ liệu điện tim, và tương tác trực tiếp với đội ngũ y bác sĩ một cách hiệu quả.

Trong quá trình thực hiện đồ án, em đã có cơ hội làm việc cùng team web của SPARC Laboratory và nhận được sự hướng dẫn tận tình từ các thầy cô và anh/chị/bạn trong các phòng thí nghiệm thuộc khoa Điện - Điện tử. Đặc biệt, em xin gửi lời cảm ơn chân thành đến TS. Hàn Huy Dũng và SPARC Laboratory – những người đã trực tiếp hướng dẫn, chỉ ra các điểm cần cải thiện trong quá trình thực hiện đồ án cũng như thiết kế hệ thống. Đồng thời, em cũng rất trân trọng sự hỗ trợ và hợp tác từ nhóm firmware SPARC Lab, qua đó em đã học hỏi và tích lũy được nhiều kiến thức quý báu.

Mặc dù đã nỗ lực hết mình trong quá trình hoàn thiện đồ án, nhưng khó có thể tránh khỏi những thiếu sót. Em rất mong nhận được ý kiến đóng góp từ quý thầy cô và bạn đọc để có thể cải thiện và phát triển đề tài này tốt hơn trong tương lai.

Em xin chân thành cảm ơn!


\cleardoublepage
