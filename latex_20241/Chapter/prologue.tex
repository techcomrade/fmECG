\section*{LỜI NÓI ĐẦU} % dấu * để không đánh sô thứ tự vào lời nói đầu
\thispagestyle{empty}
Trong thời đại công nghệ phát triển vượt bậc, Internet vạn vật (IoT) ngày càng khẳng định vai trò quan trọng, đặc biệt trong lĩnh vực y tế. Việc áp dụng IoT vào quản lý thiết bị y tế không chỉ giúp tối ưu hóa quy trình chăm sóc sức khỏe mà còn mở ra những cơ hội đổi mới, nâng cao chất lượng dịch vụ y tế.
Tại Việt Nam, ứng dụng các công nghệ tiên tiến vào lĩnh vực y tế đóng vai trò quan trọng trong việc cải thiện sức khỏe cộng đồng, từ đó góp phần xây dựng nguồn nhân lực khỏe mạnh, đáp ứng yêu cầu phát triển đất nước.

Đồ án này của chúng em xây dựng một hệ thống trang web quản lý các thiết bị y tế và dữ liệu điện tim, với mục tiêu tăng cường kết nối và hỗ trợ tương tác giữa bệnh nhân và bác sĩ.
Hệ thống cho phép bệnh nhân dễ dàng theo dõi tình trạng sức khỏe tại nhà, đồng thời nhận được sự hỗ trợ chuyên môn nhanh chóng từ đội ngũ y tế. Với giao diện thân thiện và dễ sử dụng, hệ thống tích hợp các chức năng hữu ích như đặt lịch hẹn, theo dõi dữ liệu điện tim, và trao đổi thông tin trực tiếp giữa bệnh nhân và bác sĩ thông qua tính năng nhắn tin.

Trong quá trình thực hiện, chúng em đã nhận được sự hỗ trợ và hướng dẫn tận tình từ các thầy cô cùng sự giúp đỡ của các anh/chị/bạn tại các phòng thí nghiệm thuộc khoa Điện - Điện tử. Đặc biệt, chúng em xin bày tỏ lòng biết ơn sâu sắc đến TS. Hàn Huy Dũng, người đã luôn đồng hành và đưa ra những góp ý quý báu trong việc hoàn thiện đồ án.
Đồng thời, sự phối hợp cùng nhóm firmware của SPARC Lab đã giúp chúng em tích lũy được nhiều kiến thức và kinh nghiệm thực tế.

Dẫu đã nỗ lực hoàn thiện, đồ án của chúng em chắc chắn vẫn còn một số hạn chế. Chúng em rất mong nhận được những ý kiến đóng góp từ thầy cô và bạn đọc để tiếp tục phát triển và nâng cao chất lượng đề tài trong tương lai.

Chúng em xin chân thành cảm ơn!



\cleardoublepage
