
\section*{CHƯƠNG 1. THU THẬP YÊU CẦU}
\setcounter{section}{1}
\setcounter{subsection}{0} %LƯU Ý MỖI LẦN THÊM CHƯƠNG MỚI CẦN THÊM CÂU NÀY ĐỂ RESET THỨ TỰ CỦA SUBSECTON VỀ 1
\setcounter{table}{0} % LƯU Ý SAU MỖI LẦN GỌI BẢNG HAY HÌNH ẢNH PHẢI THÊM CÂU NÀY ĐỂ RESET THỨ TỰ
\setcounter{figure}{0} %% LƯU Ý SAU MỖI LẦN GỌI BẢNG HAY HÌNH ẢNH PHẢI THÊM CÂU NÀY ĐỂ RESET THỨ TỰ
\addcontentsline{toc}{section}{\numberline{}CHƯƠNG 1. THU THẬP YÊU CẦU}
Trong chương này, chúng em sẽ tiến hành thu thập yêu cầu cho dự án đề tài "Hệ thống theo dõi và quản lý dữ liệu điện tim" dựa trên các mục tiêu
đã nêu ra trong Mục Đề xuất hệ thống ở Phần mở đầu.

\subsection{Các kĩ thuật được sử dụng}
Kĩ thuật phỏng vấn: Đây là 1 kĩ thuật quan trọng để thu thập dữ liệu về các yêu cầu của hệ thống cần thiết kế. Sau đây là bảng danh sách câu hỏi và câu trả lời mà chúng em đã tổng hợp được:
\begin{table}[H]
    \centering
    \caption{Bảng câu hỏi phỏng vấn và câu trả lời}
    \begin{tabularx}{0.9\textwidth}{
    | >{\raggedright\arraybackslash}X
    | >{\raggedright\arraybackslash}X|
    }
    \hline
    \bfseries Câu hỏi                              &   \bfseries Câu trả lời \\\hline
    Đối tượng của hệ thống bao gồm những ai?       &   Bao gồm người dùng (bệnh nhân), bác sĩ, quản trị viên \\\hline
    Hệ thống có thể triển khai trên nền tảng nào?  &   Hệ thống có thể triển khai trên cả máy tính và điện thoại \\\hline
    Hệ thống được hiển thị bằng ngôn ngữ nào?      &   Hệ thống hỗ trợ cả tiếng Việt và tiếng Anh\\\hline
    Hệ thống của bạn quản lí những gì?             &   Hệ thống quản lí thông tin cá nhân của các đối tượng sử dụng, quản lí các thiết bị, bản ghi \\\hline
    Website của hệ thống có những chức năng gì?    &   Website có chức năng cung cấp giao diện quản lí cho 2 đối tượng là quản trị viên và bác sĩ (quản lí người dùng, các thiết bị, bản ghi), chức năng xem đồ thị để bác sĩ dễ dàng chẩn đoán bệnh và có thể trao đổi trực tiếp với bệnh nhân về tình hình sức khỏe cũng như các kết quả đo được từ thiết bị \\\hline
    Hệ thống có giới hạn quyền truy cập với những đối tượng khác nhau không?   &  Có, hệ thống sẽ cho mỗi người dùng những quyền hạn riêng tương ứng với chức vụ của mình. Ví dụ như bác sĩ sẽ không thể truy cập api quản lí toàn bộ người dùng (đây là quyền của quản trị viên) \\\hline
    Hệ thống có giới hạn lượt gọi api của 1 người dùng không?                  &  Có, người dùng không được gọi 1 api quá 5 lần trong 1 giây \\\hline
    Yêu cầu về ngôn ngữ lập trình và cơ sở dữ liệu &   Hệ thống sử dụng ngôn ngữ javascript và quản trị CSDL bằng mySQL \\\hline
    Là 1 người dùng, anh/chị nghĩ hệ thống cần có những yêu cầu gì?            &  Hệ thống cần hoạt động tốt, có giao diện đơn giản, dễ sử dụng. \\\hline
    \end{tabularx}
\end{table}

\subsection{Yêu cầu hệ thống}
\subsubsection{Yêu cầu về người dùng hệ thống}
Hệ thống được thiết kế để phục vụ các đối tượng sau:
\begin{adjustwidth}{1.5em}{}
\begin{itemize}
    \item Người dùng: Người sử dụng hệ thống để thực hiện theo dõi dữ liệu điện tim thông qua Bluetooth. Bệnh nhân có quyền truy cập vào kết quả ECG của mình, có thể được một bác sĩ phụ trách và theo dõi các thông tin liên quan đến điện tim và sức khoẻ
    \item Quản trị viên: Người sử dụng hệ thống để quản lý các tài khoản người dùng, thiết bị, phân công bệnh nhân cho bác sĩ và quản lý mối quan hệ giữa bác sĩ và bệnh nhân, quản lý các bản dữ liệu đo
\end{itemize}
\end{adjustwidth}

\subsubsection{Yêu cầu chức năng}
Các chức năng chính của hệ thống bao gồm:
\begin{adjustwidth}{1.5em}{}
  \begin{itemize}
      \item Ghi lại dữ liệu thiết bị: Hệ thống cho phép ghi lại dữ liệu từ thiết bị y tế được truyền qua Bluetooth. Dữ liệu được chuyển tới ứng dụng của người dùng thông qua Bluetooth để lưu trữ, phân tích và có thể xem lại sau này
      \item Hiển thị và phân tích dữ liệu: Hệ thống hiển thị dữ liệu y tế theo dạng bảng biểu và đồ thị. Hệ thống cũng hỗ trợ xuất ra các tệp đã được chuẩn hoá cho các dữ liệu chuỗi thời gian (time-series database) để phục vụ mục đích phân tích và nghiên cứu sâu hơn
      \item Lưu trữ: Hệ thống hỗ trợ lưu dữ liệu mà người dùng đo được từ thiết bị trên cả ứng dụng và trên server của hệ thống. Dữ liệu điện tim cũng được đồng bộ hóa và lưu trữ trên máy chủ của hệ thống. Qua quá trình đồng bộ hóa, dữ liệu từ ứng dụng được truyền đến máy chủ và được lưu trữ an toàn và bảo mật trên hệ thống. Việc lưu trữ dữ liệu trên cả ứng dụng và máy chủ giúp đảm bảo rằng dữ liệu quan trọng này được lưu trữ một cách đáng tin cậy và có sẵn cho phân tích hoặc sử dụng tương lai
      \item Trao đổi và chia sẻ thông tin về dữ liệu y tế: Hệ thống giúp người dùng có thể trao đổi trực tiếp với nhau, chia sẻ kết quả đo điện tim, hỏi đáp về các vấn đề sức khỏe hoặc thảo luận về các quyết định. Điều này mang lại sự tiện lợi và hỗ trợ đáng kể cho người dùng trong việc xác định về tình trạng sức khoẻ hiện tại của bản thân
  \end{itemize}
\end{adjustwidth}
% Hệ thống hỗ trợ các chức năng cơ bản sau đối với người dùng:
\textbf{Đối với người dùng:}
\begin{adjustwidth}{1.5em}{}
\begin{itemize}
    \item Đăng nhập và đăng ký tài khoản bằng thông tin cá nhân, bao gồm tên, địa chỉ email, ngày sinh, số điện thoại và mật khẩu
    \item Cập nhật các thông tin cá nhân
    \item Được theo dõi điện tim trực tiếp khi kết nối ứng dụng di động với thiết bị đo điện tim thông qua Bluetooth
    \item Xem kết quả ECG của mình, bao gồm biểu đồ và các thông số liên quan
    \item Theo dõi các tin tức liên quan đến sức khoẻ và tim mạch
    \item Nhận thông báo và có thể trao đổi trực tiếp với bác sĩ về tình hình sức khoẻ và các kết quả đo được từ thiết bị
\end{itemize}
\end{adjustwidth}
\textbf{Đối với quản trị viên:}
\begin{adjustwidth}{1.5em}{}
\begin{itemize}
    \item Đăng nhập và đăng ký tài khoản bằng thông tin cá nhân, bao gồm tên, địa chỉ email, số điện thoại và mật khẩu
    \item Cập nhật thông tin cá nhân
    \item Quản lý danh sách người dùng trong hệ thống, bao gồm bệnh nhân và bác sĩ
    \item Quản lý danh sách thiết bị
    \item Quản lý các bản ghi dữ liệu của các thiết bị trong hệ thống
    \item Quản lý các tin tức được đăng trên ứng dụng của người dùng
\end{itemize}
\end{adjustwidth}

\textbf{Đối với bác sĩ:}
\begin{adjustwidth}{1.5em}{}
\begin{itemize}
    \item Đăng nhập và đăng ký tài khoản bằng thông tin cá nhân, bao gồm tên, địa chỉ email, ngày sinh, số điện thoại và mật khẩu
    \item Cập nhật thông tin cá nhân
    \item Quản lý danh sách bệnh nhân, có thể thêm, xóa bệnh nhân
    \item Xem kết quả ECG của các bệnh nhân trong danh sách bệnh nhân của mình, bao gồm biểu đồ và các thông số liên quan
    \item Quản lý danh sách các bản ghi record của các bệnh nhân trong list quản lý của mình
\end{itemize}
\end{adjustwidth}

\subsubsection{Yêu cầu phi chức năng}
\begin{itemize}
    \item Hệ thống hỗ trợ ngôn ngữ Tiếng Việt và Tiếng Anh
    \item Hệ thống có thể tương thích với các loại thiết bị phổ biến hiện nay (với Android: Android 10+, IOS: IOS > 12.1)
    \item Hệ thống đảm bảo tính bảo mật và quyền riêng tư thông tin của người dùng
    \item Hệ thống phải có giao diện người dùng thân thiện, dễ sử dụng để có thể tương tác mà không gặp quá nhiều khó khăn
    \item Thời gian phản hồi của hệ thống phải nhanh chóng và ổn định
    % \item Hệ thống cần sao lưu dữ liệu định kỳ để đảm bảo tính an toàn và khả năng khôi phục dữ liệu khi cần thiết.
\end{itemize}

Thông qua việc phân tích yêu cầu hệ thống, chúng ta có cái nhìn tổng quan về các chức năng, yêu cầu phi chức năng và 
các đối tượng người dùng mà hệ thống phải hỗ trợ. Phần phân tích này sẽ cung cấp cơ sở cho việc thiết kế và phát triển hệ thống quản lý ECG, 
đáp ứng đầy đủ các yêu cầu của người dùng và đảm bảo hiệu suất, bảo mật và tính khả dụng của hệ thống
% \subsection{Phân tích tổng quan hệ thống}

\subsection{Kết luận chương}

Trong chương này, chúng em đã thực hiện thu thập yêu cầu về
 hệ thống cho đề tài , nhằm tìm hiểu
  các mục tiêu và yêu cầu đã được đề xuất.

Chúng em đã xác định rõ ràng các khía cạnh quan trọng của hệ thống,
 tập trung vào việc thiết kế một hệ thống quản lý ECG hiệu quả,
  trực quan và có khả năng theo dõi sức khỏe tim mạch một cách
   chính xác. 


\newpage
