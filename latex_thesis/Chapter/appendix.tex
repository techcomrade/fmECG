
\section*{PHỤ LỤC}
\phantomsection\addcontentsline{toc}{section}{\numberline{} PHỤ LỤC}


\subsection*{Công cụ hỗ trợ trong đồ án}

\subsubsection*{Github}

\begin{enumerate}[a)]
  \item Giới thiệu chung
  
  GitHub là một nền tảng lưu trữ mã nguồn và quản lý phiên bản sử dụng Git, được ra mắt vào năm 2008. Được thiết kế để hỗ trợ cộng đồng phát triển phần mềm, GitHub cho phép các nhà phát triển lưu trữ, quản lý, và theo dõi các dự án phần mềm một cách hiệu quả. Hiện tại, GitHub là một trong những nền tảng phổ biến nhất cho phát triển phần mềm mã nguồn mở và dự án doanh nghiệp.
  

Hỗ trợ cho các công cụ phát triển phần mềm khác: GitHub tích hợp với các công cụ phát triển phần mềm phổ biến như Visual Studio Code, Eclipse, Sublime Text, và nhiều công cụ khác để tạo ra môi trường phát triển toàn diện cho các lập trình viên.



  \item Cách sử dụng github
  
  Tạo tài khoản GitHub: Truy cập GitHub theo đường dẫn https://github.com/ và đăng ký một tài khoản miễn phí, 

Tạo một kho lưu trữ: Người dùng cần phải tạo các kho lưu trữ để lưu trữ code trong dự án của mình.

Làm việc với dữ liệu trong kho lưu trữ: Người dùng có thể thêm, sửa đổi hoặc xoá các tệp tin trong kho lưu trữ của mình.

Tạo nhánh và quản lý nhánh: Khi có nhiều người dùng sử dụng một kho lưu trữ có thể tạo các nhánh riêng biệt để thay đổi code và ghép các nhánh đó vào nhánh chính sau khi công việc đã hoàn thành

Chia sẻ kiến thức tài liệu về dự án qua wiki của github: Để cho dự án chuyên nghiệp hơn người dùng có thể viết tài liệu giới thiệu cho dự án ở mục wiki của github. Việc này sẽ giúp những thành viên cùng phát triển hiểu dự án và hoàn thành tốt hơn công việc của mình.

\end{enumerate}

\subsection*{Đường dẫn mã nguồn}

Link mã nguồn: https://github.com/techcomrade/fmECG

\clearpage