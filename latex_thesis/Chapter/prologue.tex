\section*{LỜI NÓI ĐẦU} % dấu * để không đánh sô thứ tự vào lời nói đầu
\thispagestyle{empty}
Trong bối cảnh phát triển không ngừng của công nghệ, Internet vạn vật (IoT) đã và đang trở thành một yếu tố không thể thiếu trong nhiều lĩnh vực, đặc biệt là y tế. Việc ứng dụng IoT trong quản lý thiết bị y tế đã mở ra những hướng đi mới, giúp cải thiện chất lượng chăm sóc sức khỏe và tối ưu hóa quy trình điều trị. Đối với Việt Nam, trong quá trình công nghiệp hóa và hiện đại hóa, việc áp dụng những công nghệ tiên tiến vào các lĩnh vực thiết yếu, đặc biệt là y tế, trở thành nền tảng quan trọng để chăm sóc sức khỏe con người, tạo ra những công dân khỏe mạnh, sẵn sàng đóng góp vào sự phát triển của đất nước.

Nhận thức sâu sắc về tầm quan trọng của vấn đề này, chúng em đã thực hiện đồ án với đề tài "Xây dựng hệ thống theo dõi và quản lý dữ liệu điện tim" nhằm áp dụng những kiến thức đã học, kết nối các thiết bị IoT chăm sóc sức khỏe đến người dùng một cách hiệu quả nhất, đồng thời thiết kế ứng dụng hỗ trợ các bạn phát triển phần cứng và firmware có thể kiểm thử thiết bị dễ dàng hơn.

Trong quá trình thực hiện đồ án, chúng em đã nhận được sự hướng dẫn và hỗ trợ tận tình từ các thầy cô và các anh/chị/bạn trong các phòng thí nghiệm thuộc trường Điện - Điện tử. Trước tiên, chúng em xin gửi lời cảm ơn chân thành đến TS. Nguyễn Thị Kim Thoa và TS. Hàn Huy Dũng, những người đã trực tiếp hướng dẫn và chỉ ra những điểm cần khắc phục trong quy trình thực hiện đồ án cũng như thiết kế hệ thống của chúng em. Ngoài ra, chúng em cũng rất may mắn khi được kết hợp và làm việc cùng các anh/chị/bạn ở nhóm firmware SPARC Lab, nhờ đó mà chúng em đã học hỏi được rất nhiều.

Mặc dù đã cố gắng kiểm tra kỹ lưỡng, nhưng đồ án của chúng em chắc chắn không thể tránh khỏi những thiếu sót và hạn chế. Chúng em rất mong nhận được ý kiến đóng góp từ các thầy cô và bạn đọc để hoàn thiện và phát triển đề tài hơn nữa.



Chúng em xin chân thành cảm ơn! 



\cleardoublepage
