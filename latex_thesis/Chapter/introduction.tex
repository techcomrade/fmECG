
\section*{PHẦN MỞ ĐẦU}
\phantomsection\addcontentsline{toc}{section}{\numberline{} PHẦN MỞ ĐẦU}
\subsection*{Đặt vấn đề}


Lĩnh vực y tế đang trải qua những thay đổi đột phá trong cuộc cách mạng công nghiệp 4.0, mang đến nhiều cải tiến và ứng dụng công nghệ tiên tiến vào chăm sóc sức khỏe.
Tầm quan trọng của việc ứng dụng công nghệ vào y tế càng được nâng cao trong đại dịch COVID-19,
có thể kể đến như máy rửa tay tự động do TS.Hàn Huy Dũng (đang công tác tại Trường Điện - Điện tử, thuộc Đại học Bách khoa Hà Nội) 
cùng các cộng sự sáng chế \cite{ref_thay_dzung} và rất nhiều ứng dụng di động nổi bật được đông đảo người dân Việt Nam sử dụng như PC-Covid - ứng dụng theo dõi quản lý thông tin tiêm chủng cá nhân. Đó là những tín hiệu tích cực cho thấy nước ta đang tích cực ứng
dụng công nghệ 4.0 vào trong ngành y tế một cách chủ động. Hiện nay, việc chăm sóc sức khỏe ngày càng được coi trọng và trở thành ưu tiên hàng đầu. Đối
 với những người có nhu cầu tự theo dõi sức khỏe, các thiết bị đo lường nhỏ gọn và ứng dụng di động hỗ trợ đã trở thành công cụ thiết yếu. Đặc biệt,
  đối với những người đang gặp vấn đề về sức khỏe, việc phải đến các bệnh viện đông đúc để kiểm tra và khám chữa bệnh là một thách thức không nhỏ. Do đó,
   câu hỏi đặt ra là liệu có phương án khả thi nào giúp người dùng có thể theo dõi tình trạng sức khỏe ngay tại nhà, nhưng vẫn nhận được sự tư vấn và hỗ trợ từ các chuyên gia y tế hay không? 

\subsection*{Đề xuất hệ thống}
% \phantomsection\addcontentsline{toc}{section}{\numberline{} Đề xuất hệ thống}

Xuất phát từ nhu cầu chăm sóc sức khoẻ tại nhà của người bệnh, đồ án đề xuất một hệ thống IOT theo dõi và quản lý dữ liệu điện tim bao gồm: Website, Server, Mobile App, Thiết bị đo. Trong đồ án này chúng em tập trung vào Server và website.

Với nền tảng được tiếp cận thiết bị đo điện tim bằng điện cực không tiếp xúc trong thời gian gần đây, 
cùng với đó là việc các thiết bị IOT đang rất phát triển và là ưu tiên trong thời điểm hiện nay, chúng em mong muốn xây dựng được một hệ thống
có thể kết nối được các thiết bị đo điện tim, thu thập dữ liệu điện tim thời gian thực, đồng thời có thể lưu trữ và phục vụ cho mục đích
phân tích dữ liệu về sau này. Cụ thể, hệ thống sẽ bao gồm:

\begin{adjustwidth}{1.5em}{}
  \begin{itemize}
     
      \item Một website cho bác sĩ để có thể xem được kết quả đo của các bệnh nhân được quản lý, trao đổi được với bệnh nhân
      \item Một website cho admin để quản lý hệ thống, đặc biệt là có phần phân công bác sĩ quản lý cho bệnh nhân, xem được thông tin thiết bị đang có cũng như theo dõi kết quả đo
      \item Một server để lưu cơ sở dữ liệu liên quan đến người dùng và dữ liệu đo của bệnh nhân, có thể phục vụ cho công tác nghiên cứu và
      phân tich dữ liệu sau này
      \item Có tích hợp trợ lý ảo giúp người dùng dễ dàng tiếp cận và sử dụng hệ thống
  \end{itemize}
  \end{adjustwidth}


\subsection*{Mục tiêu của đề tài}
Sau khi đã trình bày đề xuất về một hệ thống theo dõi và quản lý dữ liệu điện tim, mục tiêu đặt ra khi thực hiện
đề tài này đó là:

\begin{adjustwidth}{1.5em}{}
  \begin{itemize}
      \item Nắm được cơ sở lý thuyết và cách thiết kế một hệ thống phần mềm.
      \item Thực hiện hoàn chỉnh Website và Server được đề ra trong mục Đề xuất hệ thống, các ứng dụng hoạt động ổn định
      \item Có thể kết hợp tốt với các thiết bị phần cứng đang được hợp tác nghiên cứu
      \item Cung cấp tài liệu tham khảo một cách đầy đủ, trung thực

  \end{itemize}
  \end{adjustwidth}





\subsection*{Phương pháp nghiên cứu}
Trong đồ án lần này, chúng em đã thực hiện kết hợp các phương pháp nghiên cứu, đầu tiên là tham khảo thông tin các bài
báo, sản phẩm về thiết bị điện tim trong các phòng nghiên cứu tại trường, cùng với đó, tìm hiểu cách các hệ thống phần cứng và phần mềm hoạt động,
kết nối với nhau. Sau khi đã nắm được cơ sở lý thuyết, chúng em tiến hành các bài thực nghiệm, lưu trữ dữ liệu các bản ghi đo đạc, thử nghiệm vẽ các biểu đồ thể hiện chỉ số được lấy từ các bản ghi
kết hợp với nhóm firmware để phân tích số liệu nhằm chắc chắn dữ liệu được truyền đúng, kết nối với những chuyên gia có chuyên môn về lĩnh vực y tế, đặc biệt là lĩnh vực tim mạch để chắc chắn đồ thị đã biểu diễn được đúng dữ liệu đo được.

\subsection*{Kết quả đạt được}

Trong suốt quá trình thực hiện đồ án, hai chúng em Trần Minh Tuấn, Phạm Quang Huy đã được tìm hiểu và nghiên cứu sâu hơn về cả phần cứng,
hệ thống IOT, cách kết nối các hệ thống với nhau. Các kết quả đạt được cho đến thời điểm hoàn thiện quyển đồ án bao gồm:

\begin{adjustwidth}{1.5em}{}
  \begin{itemize}
      \item Hoàn thành quyển đồ án với nội dung chi tiết về quá trình xây dựng và phát triển hệ thống
      \item Hoàn thành các sản phẩm ứng dụng đã đề ra trong mục Đề xuất hệ thống, các sản phẩm đã có sự kết nối, dữ liệu điện tim được theo dõi
      và lưu trên server, dữ liệu có thể được phân tích và nghiên cứu sau này
      \item Được phát triển các kỹ năng làm việc nhóm, viết đồ án, kết hợp với nhóm phần cứng, nhóm firmware, các chuyên gia trong lĩnh vực y tế để các sản phẩm được hoàn thiện hơn
    \end{itemize}
  \end{adjustwidth}
\subsection*{Cấu trúc đồ án}

% \phantomsection\addcontentsline{toc}{section}{\numberline{} Cấu trúc đồ án}
\begin{adjustwidth}{1.5em}{}
\begin{itemize}
  \item Phần mở đầu: Trình bày về mục đích của đồ án, đề xuất hệ thống, phần tích tính khả thi và bố cục đồ án
  \item Chương 1: Trình bày chi tiết các khâu thu thập yêu cầu. Bao gồm kĩ thuật thu thập, xác định yêu cầu hệ thống, thiết kế sơ đồ use case
  \item Chương 2: Trình bày chi tiết các khâu trong phân tích hệ thống. Bao gồm mô tả thẻ CRC, thiết kế sơ đồ lớp, sơ đồ tuần tự
  \item Chương 3: Trình bày chi tiết khâu thiết kế cho hệ thống. Bao gồm thiết kế sơ đồ kiến trúc hệ thống, sơ đồ khối
  phần mềm, thiết kế cơ sở dữ liệu, thiết kế giao diện, sơ đồ lớp và thiết kế chức năng cho hệ thống
  \item Chương 4: Trình bày khâu triền khai và kiểm thử
  \item Phần kết luận: Kết luận và đưa ra hướng phát triển 
\end{itemize}
\end{adjustwidth}

\cleardoublepage

% \pagenumbering{arabic}