
\section*{PHẦN MỞ ĐẦU}
\phantomsection\addcontentsline{toc}{section}{\numberline{} PHẦN MỞ ĐẦU}
\subsection*{Đặt vấn đề}
\addcontentsline{toc}{section}{\numberline{} Đặt vấn đề}
% Cuộc cách mạng công nghiệp lần thứ tư (hay còn được gọi là cuộc cách mạng công nghiệp 4.0) đã và đang phát triển với tốc
% độ rất nhanh, ảnh hưởng đến mọi mặt đời sống xã hội. Nội dung cốt lõi của cuộc cách mạng chính là sự kết hợp giữa 
% khoa học công nghệ, trí tuệ nhân tạo và sự sáng tạo của con người. Đối với Việt Nam đang trong quá trình công nghiệp hoá, 
% hiện đại hoá, việc áp dụng được những công nghệ mới trong một số lĩnh vực thiết yếu của xã hội, đặc biệt trong ngành y tế, 
% chính là nền tảng quan trọng để chăm sóc sức khoẻ con người, từ đó tạo nên những con người với sức khoẻ tốt nhất, sẵn sàng 
% đóng góp cho sự phát triển của đất nước. 
Lĩnh vực y tế đang có những bước chuyển mình lớn trong cuộc cách mạng công nghiệp lần thứ tư 
(hay còn được gọi là cuộc cách mạng công nghiệp 4.0). Đại dịch COVID-19 đã chứng minh được tầm quan trọng của việc áp dụng
khoa học kỹ thuật vào những sản phẩm y tế giúp đẩy lùi dịch bệnh,
có thể kể đến như máy rửa tay tự động do TS.Hàn Huy Dũng (đang công tác tại Trường Điện - Điện tử, thuộc Đại học Bách khoa Hà Nội) 
(thêm reference) cùng các cộng sự sáng chế, và một số ứng dụng di động
nổi bật được hầu hết người dân Việt Nam sử dụng trong đại dịch COVID-19 như Bluezone - ứng dụng cảnh báo tiếp xúc gần với
những người nhiễm COVID qua Bluetooth low energy, ứng dụng NCOVI, theo dõi các ca nhiễm và thực hiện khai báo y tế, ứng
dụng PC-Covid để cập nhật các thông tin tiêm vắc xin, thông tin xét nghiệm. Đây là một tín hiệu cho thấy nước ta đang áp
dụng công nghệ 4.0 vào trong ngành y tế một cách chủ động. Hiện nay, việc chăm sóc sức khoẻ đang được chú trọng, đặc biệt
là đối với những mẹ bầu, những người cần theo dõi sức khoẻ định kỳ liên tục, việc di chuyển đến bệnh viện đông đúc để
thăm khám rất khó khăn, cộng với chi phí không hề rẻ, và tỉ lệ sẩy thai, thai lưu khi phát hiện không kịp thời là khá cao.
câu hỏi đặt ra là có cách nào có thể giúp các mẹ bầu không cần di chuyển đến bệnh viện mà vẫn có thể theo dõi được sức khoẻ của mình và
em bé, vẫn được sự chăm sóc, tư vấn của bác sĩ và chi phí ở mức có thể tiếp cận với nhiều gia đình. 

\subsection*{Đề xuất hệ thống}
\phantomsection\addcontentsline{toc}{section}{\numberline{} Đề xuất hệ thống}
Với nền tảng đã có thiết bị đo điện tim của mẹ và thai nhi bằng điện cực không tiếp xúc thì việc phát triển một hệ thống
phần mềm để có thể tiếp cận gần hơn với mục tiêu là giúp các mẹ bầu không cần di chuyển, vẫn theo dõi được tim thai của
cả mẹ và bé, và chi phí tốt với nhiều gia đình là điều mà đồ án chúng em hướng tới. Hệ thống của chúng em sẽ gồm:

- Một ứng dụng di động cho riêng những mẹ bầu để có thể theo dõi trực tiếp tình trạng sức khoẻ của mẹ và thai nhi, đồng
thời có thể trao đổi với bác sĩ về tình trạng các lần đo, xem tin tức về các thông tin liên quan tới sức khoẻ của mẹ bầu.

- Một ứng dụng di động cho bác sĩ để có thể xem được kết quả đo của các bệnh nhân được quản lý, trao đổi được với bệnh nhân.

- Một ứng dụng web cho admin để quản lý hệ thống, đặc biệt là có phần phân công bác sĩ quản lý cho bệnh nhân.

- Một server để lưu cơ sở dữ liệu liên quan đến người dùng và dữ liệu đo của người dùng, có thể phục vụ cho công tác nghiên cứu
sau này.
(Nếu có) \cite{nhu2019effects}

\subsection*{Cấu trúc đồ án}
\phantomsection\addcontentsline{toc}{section}{\numberline{} Cấu trúc đồ án}
- Phần mở đầu: Trình bày về mục đích của đồ án, thu thập yêu cầu, đề xuất hệ thống, phần tích tính khả thi và bố cục đồ án.

- Chương 1: Trình bày chi tiết các khâu trong phân tích hệ thống. 
Bao gồm xác định yêu cầu, thiết kế sơ đồ use case, biểu đồ hoạt động, biểu đồ tuần tự.

- Chương 2: Trình bày chi tiết khâu thiết kế cho hệ thống. Bao gồm thiết kế giao diện phần mềm, thiết kết API, thiết kế cơ sở dữ liệu
và giải pháp tối ưu hiệu nặng.

- Chương 3: Trình bày khâu triền khai và kiểm thử.

- Chương 4: Kết luận và nêu ra hướng phát triển.
\cleardoublepage

\pagenumbering{arabic} %đánh số thứ tự 1,2,3....