
\section*{PHẦN MỞ ĐẦU}
\phantomsection\addcontentsline{toc}{section}{\numberline{} PHẦN MỞ ĐẦU}
\subsection*{Đặt vấn đề}
\addcontentsline{toc}{section}{\numberline{} Đặt vấn đề}
% Cuộc cách mạng công nghiệp lần thứ tư (hay còn được gọi là cuộc cách mạng công nghiệp 4.0) đã và đang phát triển với tốc
% độ rất nhanh, ảnh hưởng đến mọi mặt đời sống xã hội. Nội dung cốt lõi của cuộc cách mạng chính là sự kết hợp giữa 
% khoa học công nghệ, trí tuệ nhân tạo và sự sáng tạo của con người. Đối với Việt Nam đang trong quá trình công nghiệp hoá, 
% hiện đại hoá, việc áp dụng được những công nghệ mới trong một số lĩnh vực thiết yếu của xã hội, đặc biệt trong ngành y tế, 
% chính là nền tảng quan trọng để chăm sóc sức khoẻ con người, từ đó tạo nên những con người với sức khoẻ tốt nhất, sẵn sàng 
% đóng góp cho sự phát triển của đất nước. 
Lĩnh vực y tế đang có những bước chuyển mình lớn trong cuộc cách mạng công nghiệp lần thứ tư 
(hay còn được gọi là cuộc cách mạng công nghiệp 4.0). Đại dịch COVID-19 đã chứng minh được tầm quan trọng của việc áp dụng
khoa học kỹ thuật vào những sản phẩm y tế giúp đẩy lùi dịch bệnh,
có thể kể đến như máy rửa tay tự động do TS.Hàn Huy Dũng (đang công tác tại Trường Điện - Điện tử, thuộc Đại học Bách khoa Hà Nội) 
(thêm reference) cùng các cộng sự sáng chế, và một số ứng dụng di động
nổi bật được hầu hết người dân Việt Nam sử dụng trong đại dịch COVID-19 như Bluezone - ứng dụng cảnh báo tiếp xúc gần với
những người nhiễm COVID qua Bluetooth low energy, ứng dụng NCOVI, theo dõi các ca nhiễm và thực hiện khai báo y tế, ứng
dụng PC-Covid để cập nhật các thông tin tiêm vắc xin, thông tin xét nghiệm. Đây là một tín hiệu cho thấy nước ta đang áp
dụng công nghệ 4.0 vào trong ngành y tế một cách chủ động. Hiện nay, việc chăm sóc sức khoẻ đang được chú trọng, đặc biệt
là đối với những mẹ bầu, những người cần theo dõi sức khoẻ định kỳ liên tục, việc di chuyển đến bệnh viện đông đúc để
thăm khám rất khó khăn, cộng với chi phí không hề rẻ, và tỉ lệ sẩy thai, thai lưu khi phát hiện không kịp thời là khá cao.
câu hỏi đặt ra là có cách nào có thể giúp các mẹ bầu không cần di
chuyển

\subsection*{Đề xuất hệ thống}
\phantomsection\addcontentsline{toc}{section}{\numberline{} Đề xuất hệ thống}
(Nếu có) \cite{nhu2019effects}

\subsection*{Cấu trúc đồ án}
\phantomsection\addcontentsline{toc}{section}{\numberline{} Cấu trúc đồ án}
(nếu có)
\cleardoublepage

\pagenumbering{arabic} %đánh số thứ tự 1,2,3....