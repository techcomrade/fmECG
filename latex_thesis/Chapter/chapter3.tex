
\section*{CHƯƠNG 3. NHẬN XÉT VÀ ĐỀ XUẤT}
\setcounter{section}{3}
\setcounter{subsection}{0} %LƯU Ý MỖI LẦN THÊM CHƯƠNG MỚI CẦN THÊM CÂU NÀY ĐỂ RESET THỨ TỰ CỦA SUBSECTON VỀ 1
\setcounter{table}{0} % LƯU Ý SAU MỖI LẦN GỌI BẢNG HAY HÌNH ẢNH PHẢI THÊM CÂU NÀY ĐỂ RESET THỨ TỰ
\setcounter{figure}{0} %% LƯU Ý SAU MỖI LẦN GỌI BẢNG HAY HÌNH ẢNH PHẢI THÊM CÂU NÀY ĐỂ RESET THỨ TỰ
\addcontentsline{toc}{section}{\numberline{}CHƯƠNG 3. NHẬN XÉT VÀ ĐỀ XUẤT}

\subsection{Ưu điểm}
Trong môi trường làm việc chuyên nghiệp của công ty, em đã được phát triển thêm các kỹ năng không chỉ liên quan tới
công việc mà còn rất nhiều kỹ năng mềm khác. Những điểm tích cực em đã hoàn thành khi làm việc ở công ty:
\begin{itemize}
  \item Có tinh thần trách nhiệm, hoàn thành tốt các nhiệm vụ được giao
  \item Năng động, hoà đồng với mọi người trong nhóm
  \item Có khả năng phát triển và ham học hỏi
\end{itemize}

\subsection{Nhược điểm}
Ngoài những điểm tích cực vẫn có một số điểm em cần khắc phục:
\begin{itemize}
  \item Chưa thực sự cân bằng được với việc làm và học ở trường
  \item Đôi lúc chưa thực sự tập trung khi làm việc
\end{itemize}
\subsection{Đề xuất}
Nhà trường đã rất tạo điều kiện cho sinh viên khi không những có thể tổ chức các buổi hội thảo hướng nghiệp, tìm kiếm
thông tin việc làm mà còn kết hợp với các công ty tuyển thực tập trực tiếp cho sinh viên. Đây là một điều em thấy rất tự
hào. Cho đến thời điểm hiện tại về vấn đề thực tập em hoàn toàn hài lòng về sự hỗ trợ như các email nhắc nhở, đốc thúc
các sinh viên và đảm bảo các sinh viên có vị trí thực tập. Em chưa có thêm đề xuất nào về vấn đề thực tập.
\newpage
