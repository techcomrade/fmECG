
\section*{CHƯƠNG 3. TRIỂN KHAI VÀ KIỂM THỬ}
\setcounter{section}{3}
\setcounter{subsection}{0} %LƯU Ý MỖI LẦN THÊM CHƯƠNG MỚI CẦN THÊM CÂU NÀY ĐỂ RESET THỨ TỰ CỦA SUBSECTON VỀ 1
\setcounter{table}{0} % LƯU Ý SAU MỖI LẦN GỌI BẢNG HAY HÌNH ẢNH PHẢI THÊM CÂU NÀY ĐỂ RESET THỨ TỰ
\setcounter{figure}{0} %% LƯU Ý SAU MỖI LẦN GỌI BẢNG HAY HÌNH ẢNH PHẢI THÊM CÂU NÀY ĐỂ RESET THỨ TỰ
\addcontentsline{toc}{section}{\numberline{}CHƯƠNG 3. TRIỂN KHAI VÀ KIỂM THỬ}

\subsection{Triển khai ứng dụng}
Trong quá trình triển khai ứng dụng, chúng em sử dụng dịch vụ Elastic Compute
 Cloud (EC2) của AWS để chạy ứng dụng và sử dụng dịch vụ Relational Database Service
  (RDS) để lưu trữ cơ sở dữ liệu của ứng dụng. Việc sử dụng EC2 và RDS giúp
   chúng em tối ưu hóa việc quản lý hệ thống, đảm bảo tính sẵn sàng và
    mở rộng khả năng chịu tải cho ứng dụng.

\subsubsection{Triển khai ứng dụng Web trên AWS EC2}
\begin{itemize}
  \item Tạo máy ảo EC2: Chúng em đã tạo một EC2 instance với loại instance t2.micro. Đây là loại instance nhỏ, phù hợp với các ứng dụng có lưu lượng truy cập thấp hoặc giai đoạn phát triển. Instance này sử dụng kiến trúc 64-bit, cho phép chạy các ứng dụng trên cả hệ thống 32-bit và 64-bit. Bộ nhớ của instance là 1 GiB, đủ để chạy ứng dụng Node.js cùng với các dependencies. Dung lượng lưu trữ 8GB, chúng em đã cài đặt hệ điều hành Linux/UNIX để chạy mã nguồn của ứng dụng.
  \item Cài đặt các phần mềm và dependencies: Sau khi triển khai máy ảo EC2, chúng em đã cài đặt các phần mềm và dependencies cần thiết để chạy ứng dụng, bao gồm Node.js và các thư viện hỗ trợ và các gói npm cần thiết.
  \item Tạo và cấu hình môi trường ứng dụng:Chúng em đã tạo môi trường ứng dụng, bao gồm việc cấu hình các biến môi trường, thiết lập các file cấu hình, và chạy các lệnh khởi tạo ban đầu cho ứng dụng.
  \item Triển khai ứng dụng: Tiếp theo, chúng em tải lên mã nguồn của ứng dụng lên máy ảo EC2 thông qua git
  \item Mở cổng cho ứng dụng: Chúng em đã mở cổng mạng trên máy ảo EC2 để cho phép ứng dụng lắng nghe các yêu cầu từ internet.
  \item Khởi động ứng dụng:Để khởi động ứng dụng, chúng em chạy các lệnh quản lý quá trình như npm start
\end{itemize}

\subsubsection{Triển khai cơ sở dữ liệu trên RDS}
\begin{itemize}
  \item Tạo cơ sở dữ liệu RDS: Chúng em đã triển khai một cơ sở dữ liệu MySQL trên dịch vụ RDS của AWS. Cơ sở dữ liệu này được đặt tên là "fm\_ecg" và có mật khẩu "fmecgdatabase" để bảo mật. Địa chỉ host của cơ sở dữ liệu là "fm-ecg-database.cx3akkmg3hid.ap-southeast-1.rds.amazonaws.com" cho phép EC2 instance kết nối và truy vấn dữ liệu từ cơ sở dữ liệu. Để truy cập và quản lý cơ sở dữ liệu, chúng em sử dụng tài khoản "admin" với các quyền tương ứng.
  \item Kết nối giữa EC2 và RDS: Chúng em đã cấu hình ứng dụng Node.js chạy trên EC2 instance để kết nối đến cơ sở dữ liệu RDS thông qua thông tin host, tên người dùng và mật khẩu đã cung cấp. Khi ứng dụng gửi các truy vấn SQL đến cơ sở dữ liệu, RDS sẽ xử lý các truy vấn này và trả về kết quả cho ứng dụng.
  \item Backup và giám sát cơ sở dữ liệu: Chúng em đã thiết lập các chính sách sao lưu định kỳ cho cơ sở dữ liệu RDS để đảm bảo an toàn cho dữ liệu và khả năng khôi phục trong trường hợp xảy ra sự cố.
\end{itemize}

\subsubsection{Triển khai ứng dụng di động trên nền tảng Android}
Trong quá trình phát triển ứng dụng di động chúng em có sử dụng máy ảo Android (phiên bản Android API 33) và máy thật
(Android 11 - 13), đồng thời xuất file APK để các thiết bị muốn test đều có thể dùng thử.

\subsection{Kiểm thử}

\subsubsection{Kiểm thử hoạt động của các API}


Môi trường: 

\begin{adjustwidth}{2em}{}
\begin{itemize}
  \item Base URL: http://13.250.106.115/
\end{itemize}
\end{adjustwidth}

Công cụ: Postman - Để xây dựng và thực hiện các yêu cầu API.



\paragraph{API liên quan đến thông tin người dùng}
\mbox{}

Tham khảo bảng \ref{table_api_user} để xem thông tin của các api liên quan

\begin{enumerate}[a)]
  \item URL: GET api/users/profile 
  
  \begin{table}[H]
    \centering
    \caption{\bfseries \fontsize{12pt}{0pt}\selectfont Bảng API liên quan đến tin tức}
    \begin{tabularx}{\textwidth}{
    | >{\raggedright\arraybackslash}m{1cm}
    | >{\raggedright\arraybackslash}X
    | >{\raggedright\arraybackslash}X
    | >{\raggedright\arraybackslash}X
    | >{\raggedright\arraybackslash}m{1cm}|
    }
    \hline
    \bfseries Test case    &\bfseries Điều kiện   &\bfseries Đầu vào 
    &\bfseries Đầu ra mong muốn &\bfseries Kết quả\\ \hline


    TC-1
    & User đã đăng nhập vào hệ thống thành công
    & JWT Cookie có tồn tại
    & 

    Status code: 200 OK

      Response content:

      \{

    "status": "success",

    "data": Thông tin của user

    \}
    
    & OK

    \\ \hline
  
    TC-2
    & User chưa đăng nhập vào hệ thống
    & JWT Cookie không tồn tại
    & 

    Status code: 500 Internal Server Error

      Response content:

      \{

    "status": "error",

    "msg": "An error occurred while retrieving the user profile"

    \}
    
    & OK

    \\ \hline
  
    \end{tabularx}
    \label{table_api_news}
  \end{table}
  

  \item URL: PUT api/users/profile
  
  \begin{table}[H]
    \centering
    \caption{\bfseries \fontsize{12pt}{0pt}\selectfont Bảng API liên quan đến tin tức}
    \begin{tabularx}{\textwidth}{
    | >{\raggedright\arraybackslash}m{1cm}
    | >{\raggedright\arraybackslash}X
    | >{\raggedright\arraybackslash}X
    | >{\raggedright\arraybackslash}X
    | >{\raggedright\arraybackslash}m{1cm}|
    }
    \hline
    \bfseries Test case    &\bfseries Điều kiện   &\bfseries Đầu vào 
    &\bfseries Đầu ra mong muốn &\bfseries Kết quả\\ \hline


    TC-1
    & User đã đăng nhập vào hệ thống thành công
    & Thông tin user muốn cập nhật

    \{

    "name": họ tên,
    "doB": ngày sinh,
    "phone\_number": số điện thoại

\}

    & 

    Status code: 200 OK

      Response content:

      \{

    "status": "success",

    "data": Thông tin của user sau khi cập nhật lại

    \}
    
    & OK

    \\ \hline
  
    TC-2
    & User chưa đăng nhập vào hệ thống
    & Thông tin user muốn cập nhật

    \{

    "name": họ tên,
    "doB": ngày sinh,
    "phone\_number": số điện thoại

\}
    & 

    Status code: 500 Internal Server Error

      Response content:

      \{

    "status": "error",

    "msg": "An error occurred while retrieving the user profile"

    \}
    
    & OK

    \\ \hline
  
    \end{tabularx}
    \label{table_api_news}
  \end{table}
  


  \item URL: PUT api/users/change-password
  
\break

  \begin{xltabular}{\textwidth}{
    | >{\raggedright\arraybackslash}m{1cm}
    | >{\raggedright\arraybackslash}X
    | >{\raggedright\arraybackslash}X
    | >{\raggedright\arraybackslash}X
    | >{\raggedright\arraybackslash}m{1cm}|
    }
    \caption{\bfseries \fontsize{12pt}{0pt}\selectfont Bảng API liên quan đến tin tức}
    % \label{table_api_news}
    \\
    \hline
    \bfseries Test case    &\bfseries Điều kiện   &\bfseries Đầu vào 
    &\bfseries Đầu ra mong muốn &\bfseries Kết quả\\ \hline
  
  
    TC-1
    & User đã đăng nhập vào hệ thống thành công và mật khẩu hiện tại trùng với req.currentPassword, req.newPassword và req.confirmPassword trùng nhau 
    & Thông tin thay đổi mật khẩu
  
    \{
  
      "currentPassword": "123456",
      "newPassword": "1234567",
      "confirmPassword": "1234567"
  
  \}
  
    & 
  
    Status code: 200 OK
  
      Response content:
  
      \{
  
    "status": "success",
  
    msg: "Password changed successfully"
  
    \}
    
    & OK
  
    \\ \hline
  
    TC-2
    & User chưa đăng nhập vào hệ thống
    & Thông tin thay đổi mật khẩu
  
    \{
  
      "currentPassword": mật khẩu hiện tại,
      "newPassword": mật khẩu mới,
      "confirmPassword": xác nhận lại mật khẩu
  
  \}
    & 
  
    Status code: 500 Internal Server Error
  
      Response content:
  
      \{
  
    "status": "error",
  
    "msg": "An error occurred while retrieving the user profile"
  
    \}
    
    & OK
  
    \\ \hline
  
    TC-3
    & User đã đăng nhập vào hệ thống thành công và mật khẩu hiện tại không
    trùng với
    req.currentPassword 
    & Thông tin thay đổi mật khẩu
  
    \{
  
      "currentPassword": "1234568",
      "newPassword": "1234567",
      "confirmPassword": "1234567"
  
  \}
  
    & 
  
    Status code: 401 Unauthorized
  
      Response content:
  
      \{
  
    "status": "error",
  
    msg: "Current password is incorrect"
  
    \}
    
    & OK
  
    \\ \hline
  
    TC-4
    & User đã đăng nhập vào hệ thống thành công và và mật khẩu hiện tại trùng với req.currentPassword, req.newPassword và req.confirmPassword không trùng nhau 
    & Thông tin thay đổi mật khẩu
  
    \{
  
      "currentPassword": "123456",
      "newPassword": "1234567324",
      "confirmPassword": "127"
  
  \}
  
    & 
  
    Status code: 401 Unauthorized
  
      Response content:
  
      \{
  
    "status": "error",
  
    msg: "New password and confirm password do not match"
  
    \}
    
    & OK
  
    \\ \hline
  
    
  
    \end{xltabular}



  \item URL: GET api/users/{:userId}
  
  
  \begin{xltabular}{\textwidth}{
    | >{\raggedright\arraybackslash}m{1cm}
    | >{\raggedright\arraybackslash}X
    | >{\raggedright\arraybackslash}X
    | >{\raggedright\arraybackslash}X
    | >{\raggedright\arraybackslash}m{1cm}|
    }
    \caption{\bfseries \fontsize{12pt}{0pt}\selectfont Bảng API liên quan đến tin tức}
    % \label{table_api_news}
    \\
    \hline
    \bfseries Test case    &\bfseries Điều kiện   &\bfseries Đầu vào 
    &\bfseries Đầu ra mong muốn &\bfseries Kết quả\\ \hline
  
  
    TC-1
    & User ID có tồn tại trên database
    & User ID
  
    & 
  
    Status code: 200 OK
  
      Response content:
  
      \{
  
    "status": "success",
  
    data: Thông tin user
  
    \}
    
    & OK
  
    \\ \hline
  
    TC-2
    & User ID không tồn tại trên database
    & User ID
  
    & 
  
    Status code: 404 Not Found
  
      Response content:
  
      \{
  
    "status": "error",
  
    "msg": "User not found"
  
    \}
    
    & OK
  
    \\ \hline
    
  
    \end{xltabular}

  
\end{enumerate}


\paragraph{API liên quan đến việc xác thực người dùng}
\mbox{}

Tham khảo bảng \ref{table_api_auth} để xem thông tin của các api liên quan



\begin{enumerate}[a)]
  \item URL: POST api/register
  
  \begin{xltabular}{\textwidth}{
    | >{\raggedright\arraybackslash}m{1cm}
    | >{\raggedright\arraybackslash}X
    | >{\raggedright\arraybackslash}X
    | >{\raggedright\arraybackslash}X
    | >{\raggedright\arraybackslash}m{1cm}|
    }
    \caption{\bfseries \fontsize{12pt}{0pt}\selectfont Bảng API liên quan đến tin tức}
    % \label{table_api_news}
    \\
    \hline
    \bfseries Test case    &\bfseries Điều kiện   &\bfseries Đầu vào 
    &\bfseries Đầu ra mong muốn &\bfseries Kết quả\\ \hline
  
  
    TC-1
    & User chưa có tài khoản trên hệ thống
    & Thông tin đăng ký tài khoản

    \{

    "password": "123456789",
    "confirm\_password": "123456789",
    "name": "Anh Tuan",
    "doB": "20-10-2001",
    "email": "test@gmail.com",
    "phone\_number": "0123344562",
    "role": 0

   \}
  
    & 
  
    Status code: 200 OK
  
      Response content:
  
      \{
  
    "status": "success",
  
    data: Thông tin user sau khi đăng ký thành công
  
    \}
    
    & OK
  
    \\ \hline
  
    TC-2
    & User đã có tài khoản trên hệ thống
    & Thông tin đăng ký tài khoản

    \{

    "password": "123456789",
    "confirm\_password": "123456789",
    "name": "Anh Tuan",
    "doB": "20-10-2001",
    "email": "test@gmail.com",
    "phone\_number": "0123344562",
    "role": 0

   \}
  
    & 
  
    Status code: 400 Bad Request
  
      Response content:
  
      \{
  
    "status": "error",
  
    "msg": "Email is already in use"
  
    \}
    
    & OK
  
    \\ \hline


    TC-3
    & User chưa có tài khoản trên hệ thống, req.password và req.confirm\_password không trùng nhau
    & Thông tin đăng ký tài khoản

    \{

    "password": "123456789",

    "confirm\_password": "123456789",

    "name": "Anh Tuan",

    "doB": "20-10-2001",

    "email": "test@gmail.com",

    "phone\_number": "0123344562",

    "role": 0

   \}
  
    & 
  
    Status code: 400 Bad Request
  
      Response content:
  
      \{
  
    "status": "error",
  
    "msg": "Passwords do not match"
  
    \}
    
    & OK
  
    \\ \hline
    
  
    \end{xltabular}

  


  \item URL: POST api/login
  

  \begin{xltabular}{\textwidth}{
    | >{\raggedright\arraybackslash}m{1cm}
    | >{\raggedright\arraybackslash}X
    | >{\raggedright\arraybackslash}X
    | >{\raggedright\arraybackslash}X
    | >{\raggedright\arraybackslash}m{1cm}|
    }
    \caption{\bfseries \fontsize{12pt}{0pt}\selectfont Bảng API liên quan đến tin tức}
    % \label{table_api_news}
    \\
    \hline
    \bfseries Test case    &\bfseries Điều kiện   &\bfseries Đầu vào 
    &\bfseries Đầu ra mong muốn &\bfseries Kết quả\\ \hline
  
  
    TC-1
    & Thông tin tài khoản và mật khẩu hợp lệ
    & Thông tin đăng nhập

    \{

    "email": email của user,
    "password": mật khẩu của user

   \}
  
    & 
  
    Status code: 200 OK
  
      Response content:
  
      \{
  
    "status": "success",
  
    data: Thông tin user sau khi đăng ký thành công
  
    \}
    
    & OK
  
    \\ \hline
  
    TC-2
    & Thông tin tài khoản và mật khẩu không hợp lệ
    & Thông tin đăng nhập

    \{

    "email": email của user,
    "password": mật khẩu của user

   \}
  
   &
  
    Status code: 401 Unauthorized
  
      Response content:
  
      \{
  
    "status": "error",
  
    "msg": "Invalid email or password"
  
    \}
    
    & OK
  
    \\ \hline

  
    \end{xltabular}



  \item URL: GET api/logout
  

  \begin{xltabular}{\textwidth}{
    | >{\raggedright\arraybackslash}m{1cm}
    | >{\raggedright\arraybackslash}X
    | >{\raggedright\arraybackslash}X
    | >{\raggedright\arraybackslash}X
    | >{\raggedright\arraybackslash}m{1cm}|
    }
    \caption{\bfseries \fontsize{12pt}{0pt}\selectfont Bảng API liên quan đến tin tức}
    % \label{table_api_news}
    \\
    \hline
    \bfseries Test case    &\bfseries Điều kiện   &\bfseries Đầu vào 
    &\bfseries Đầu ra mong muốn &\bfseries Kết quả\\ \hline
  
  
    TC-1
    & User đã đăng nhập vào hệ thống
    & JWT Token tồn tại
  
    & 
  
    Status code: 200 OK
  
      Response content:
  
      \{
  
    "status": "success",
  
    "msg": "Logged out successfully"
  
    \}
    
    & OK
  
    \\ \hline
  
    TC-2
    & User chưa đăng nhập vào hệ thống
    & JWT Token không tồn tại
  
   &
  
    Status code: 401 Unauthorized
  
      Response content:
  
      \{
  
    "status": "error",
  
    "msg": "No token found"
  
    \}
    
    & OK
  
    \\ \hline

  
    \end{xltabular}



  \item URL: POST api/reset-password
  


  \begin{xltabular}{\textwidth}{
    | >{\raggedright\arraybackslash}m{1cm}
    | >{\raggedright\arraybackslash}X
    | >{\raggedright\arraybackslash}X
    | >{\raggedright\arraybackslash}X
    | >{\raggedright\arraybackslash}m{1cm}|
    }
    \caption{\bfseries \fontsize{12pt}{0pt}\selectfont Bảng API liên quan đến tin tức}
    % \label{table_api_news}
    \\
    \hline
    \bfseries Test case    &\bfseries Điều kiện   &\bfseries Đầu vào 
    &\bfseries Đầu ra mong muốn &\bfseries Kết quả\\ \hline
  
  
    TC-1
    & User đã đăng ký tài khoản
    & Email user

    \{

    "email": email user
\}
  
    & 
  
    Status code: 200 OK
  
      Response content:
  
      \{
  
    "status": "success",
  
    "msg": "Reset token sent to email"

    "resetToken": token
  
    \}
    
    & OK
  
    \\ \hline
  
    TC-2
    & User chưa đăng ký tài khoản
    & Email user

    \{

    "email": email user
\}
   &
  
    Status code: 404 Not Found
  
      Response content:
  
      \{
  
    "status": "error",
  
    "msg": "User not found"
  
    \}
    
    & OK
  
    \\ \hline

  
    \end{xltabular}



  \item URL: POST api/reset-password/reset 
  


  \begin{xltabular}{\textwidth}{
    | >{\raggedright\arraybackslash}m{1cm}
    | >{\raggedright\arraybackslash}X
    | >{\raggedright\arraybackslash}X
    | >{\raggedright\arraybackslash}X
    | >{\raggedright\arraybackslash}m{1cm}|
    }
    \caption{\bfseries \fontsize{12pt}{0pt}\selectfont Bảng API liên quan đến tin tức}
    % \label{table_api_news}
    \\
    \hline
    \bfseries Test case    &\bfseries Điều kiện   &\bfseries Đầu vào 
    &\bfseries Đầu ra mong muốn &\bfseries Kết quả\\ \hline
  
  
    TC-1
    & User đã đăng ký tài khoản, reset token và mật khẩu hợp lệ
    & Thông tin reset mật khẩu

    \{

      "resetToken": "816e8d",

      "password": "123456",

      "confirm\_password": "123456",

      "email": "test@gmail.com"

  \}
  
    & 
  
    Status code: 200 OK
  
      Response content:
  
      \{
  
    "status": "success",
  
    "msg": "Password reset successful"
  
    \}
    
    & OK
  
    \\ \hline
  
    TC-2
    & Reset token không hợp lệ
    & Thông tin reset mật khẩu

    \{

      "resetToken": "816e8d",

      "password": "123456",

      "confirm\_password": "123456",

      "email": "test@gmail.com"

  \}
   &
  
    Status code: 400 Bad Request
  
      Response content:
  
      \{
  
    "status": "error",
  
    "msg": "Invalid reset token"
  
    \}
    
    & OK
  
    \\ \hline

    TC-3
    & Reset token hết hạn
    & Thông tin reset mật khẩu

    \{

      "resetToken": "816e8d",

      "password": "123456",

      "confirm\_password": "123456",

      "email": "test@gmail.com"

  \}
   &
  
    Status code: 400 Bad Request
  
      Response content:
  
      \{
  
    "status": "error",
  
    "msg": "Reset token has expired"
  
    \}
    
    & OK
  
    \\ \hline


    TC-4
    & req.password và req.confirm\_password không trùng
    & Thông tin reset mật khẩu

    \{

      "resetToken": "816e8d",

      "password": "123456",

      "confirm\_password": "123456",

      "email": "test@gmail.com"

  \}
   &
  
    Status code: 400 Bad Request
  
      Response content:
  
      \{
  
    "status": "error",
  
    "msg": "Password confirmation does not match"
  
    \}
    
    & OK
  
    \\ \hline

  
    \end{xltabular}



\end{enumerate}


\paragraph{API liên quan đến tin tức}
\mbox{}

Tham khảo bảng \ref{table_api_news} để xem thông tin của các api liên quan

\begin{enumerate}[a)]
  \item URL: GET api/news/{:newsId}
  


  \begin{xltabular}{\textwidth}{
    | >{\raggedright\arraybackslash}m{1cm}
    | >{\raggedright\arraybackslash}X
    | >{\raggedright\arraybackslash}X
    | >{\raggedright\arraybackslash}X
    | >{\raggedright\arraybackslash}m{1cm}|
    }
    \caption{\bfseries \fontsize{12pt}{0pt}\selectfont Bảng API liên quan đến tin tức}
    % \label{table_api_news}
    \\
    \hline
    \bfseries Test case    &\bfseries Điều kiện   &\bfseries Đầu vào 
    &\bfseries Đầu ra mong muốn &\bfseries Kết quả\\ \hline
  
  
    TC-1
    & Tin tức tồn tại với ID tương ứng
    & ID tin tức
    & 
  
    Status code: 200 OK
  
      Response content:
  
      \{
  
    "status": "success",

    data: Nội dung của tin tức
  
    \}
    
    & OK
  
    \\ \hline
  
    TC-2
    & Tin tức không tồn tại với ID tương ứng
    & ID danh mục tin tức
   &
  
    Status code: 404 Not Found
  
      Response content:
  
      \{
  
    "status": "error",
  
    "msg": "News not found"
  
    \}
    
    & OK
  
    \\ \hline

  
    \end{xltabular}



  \item URL: GET api/news
  

  \begin{xltabular}{\textwidth}{
    | >{\raggedright\arraybackslash}m{1cm}
    | >{\raggedright\arraybackslash}X
    | >{\raggedright\arraybackslash}X
    | >{\raggedright\arraybackslash}X
    | >{\raggedright\arraybackslash}m{1cm}|
    }
    \caption{\bfseries \fontsize{12pt}{0pt}\selectfont Bảng API liên quan đến tin tức}
    % \label{table_api_news}
    \\
    \hline
    \bfseries Test case    &\bfseries Điều kiện   &\bfseries Đầu vào 
    &\bfseries Đầu ra mong muốn &\bfseries Kết quả\\ \hline
  
  
    TC-1
    & NULL
    & NULL
    & 
  
    Status code: 200 OK
  
      Response content:
  
      \{
  
    "status": "success",
  
    "count": Số lượng tin tức

    data: Danh sách các tin tức
  
    \}
    
    & OK
  
    \\ \hline
  
    TC-2
    & NULL
    & Lỗi đường truyền server
   &
  
    Status code: 500 Internal Server Error
  
      Response content:
  
      \{
  
    "status": "error",
  
    "msg": "An error occurred while retrieving the news"
  
    \}
    
    & OK
  
    \\ \hline

  
    \end{xltabular}



  \item URL: GET api/categories
  

  \begin{xltabular}{\textwidth}{
    | >{\raggedright\arraybackslash}m{1cm}
    | >{\raggedright\arraybackslash}X
    | >{\raggedright\arraybackslash}X
    | >{\raggedright\arraybackslash}X
    | >{\raggedright\arraybackslash}m{1cm}|
    }
    \caption{\bfseries \fontsize{12pt}{0pt}\selectfont Bảng API liên quan đến tin tức}
    % \label{table_api_news}
    \\
    \hline
    \bfseries Test case    &\bfseries Điều kiện   &\bfseries Đầu vào 
    &\bfseries Đầu ra mong muốn &\bfseries Kết quả\\ \hline
  
  
    TC-1
    & NULL
    & NULL
    & 
  
    Status code: 200 OK
  
      Response content:
  
      \{
  
    "status": "success",
  
    "count": Số lượng danh mục tin tức

    data: Danh sách các danh mục tin tức
  
    \}
    
    & OK
  
    \\ \hline
  
    TC-2
    & Lỗi đường truyền server
    & NULL
   &
  
    Status code: 500 Internal Server Error
  
      Response content:
  
      \{
  
    "status": "error",
  
    "msg": "An error occurred while retrieving the news categories"
  
    \}
    
    & OK
  
    \\ \hline

  
    \end{xltabular}



  \item URL: GET api/category/{:categoryId}
  

  \begin{xltabular}{\textwidth}{
    | >{\raggedright\arraybackslash}m{1cm}
    | >{\raggedright\arraybackslash}X
    | >{\raggedright\arraybackslash}X
    | >{\raggedright\arraybackslash}X
    | >{\raggedright\arraybackslash}m{1cm}|
    }
    \caption{\bfseries \fontsize{12pt}{0pt}\selectfont Bảng API liên quan đến tin tức}
    % \label{table_api_news}
    \\
    \hline
    \bfseries Test case    &\bfseries Điều kiện   &\bfseries Đầu vào 
    &\bfseries Đầu ra mong muốn &\bfseries Kết quả\\ \hline
  
  
    TC-1
    & Danh mục tin tức tồn tại với ID tương ứng
    & ID danh mục tin tức
    & 
  
    Status code: 200 OK
  
      Response content:
  
      \{
  
    "status": "success",

    data: Thông tin danh mục tin tức
  
    \}
    
    & OK
  
    \\ \hline
  
    TC-2
    & Danh mục tin tức không tồn tại với ID tương ứng
    & ID danh mục tin tức
   &
  
    Status code: 404 Not Found
  
      Response content:
  
      \{
  
    "status": "error",
  
    "msg": "News category not found"
  
    \}
    
    & OK
  
    \\ \hline

  
    \end{xltabular}



  \item URL: GET api/news/category/{:categoryId}
  
  
  \begin{xltabular}{\textwidth}{
    | >{\raggedright\arraybackslash}m{1cm}
    | >{\raggedright\arraybackslash}X
    | >{\raggedright\arraybackslash}X
    | >{\raggedright\arraybackslash}X
    | >{\raggedright\arraybackslash}m{1cm}|
    }
    \caption{\bfseries \fontsize{12pt}{0pt}\selectfont Bảng API liên quan đến tin tức}
    % \label{table_api_news}
    \\
    \hline
    \bfseries Test case    &\bfseries Điều kiện   &\bfseries Đầu vào 
    &\bfseries Đầu ra mong muốn &\bfseries Kết quả\\ \hline
  
  
    TC-1
    & Danh mục tin tức tồn tại với ID tương ứng
    & ID danh mục tin tức
    & 
  
    Status code: 200 OK
  
      Response content:
  
      \{
  
    "status": "success",
    "count": Số lượng tin tức

    data: Thông tin tin tức
  
    \}
    
    & OK
  
    \\ \hline
  
    TC-2
    & Danh mục tin tức không tồn tại với ID tương ứng
    & ID danh mục tin tức
   &
  
    Status code: 404 Not Found
  
      Response content:
  
      \{
  
    "status": "error",
  
    "msg": "News category not found"
  
    \}
    
    & OK
  
    \\ \hline

  
    \end{xltabular}




\end{enumerate}



\paragraph{API liên quan đến bản ghi ECG}
\mbox{}

Tham khảo bảng \ref{table_api_ecg} để xem thông tin của các api liên quan

\begin{enumerate}[a)]
  \item URL: POST api/ecg-records/upload 
  

  \begin{xltabular}{\textwidth}{
    | >{\raggedright\arraybackslash}m{1cm}
    | >{\raggedright\arraybackslash}X
    | >{\raggedright\arraybackslash}X
    | >{\raggedright\arraybackslash}X
    | >{\raggedright\arraybackslash}m{1cm}|
    }
    \caption{\bfseries \fontsize{12pt}{0pt}\selectfont Bảng API liên quan đến tin tức}
    % \label{table_api_news}
    \\
    \hline
    \bfseries Test case    &\bfseries Điều kiện   &\bfseries Đầu vào 
    &\bfseries Đầu ra mong muốn &\bfseries Kết quả\\ \hline
  
  
    TC-1
    & NULL
    & Thông tin của ECG record dưới dạng form-data

\{

file: File dữ liệu

user\_id: ID user

device\_id: ID thiết bị

stop\_time: Thời gian dừng đo

start\_time: Thời gian bắt đầu đo

sensor\_type: Loại cảm biến


\}

    & 
  
    Status code: 200 OK
  
      Response content:
  
      \{
  
    "status": "success",

    data: Thông tin của ECG record
  
    \}
    
    & OK
  
    \\ \hline
  
    TC-2
    & Lỗi đường truyền server
    & Thông tin của ECG record dưới dạng form-data

\{

file: File dữ liệu

user\_id: ID user

device\_id: ID thiết bị

stop\_time: Thời gian dừng đo

start\_time: Thời gian bắt đầu đo

sensor\_type: Loại cảm biến


\}
   &
  
    Status code: 500 Internal Server Error
  
      Response content:
  
      \{
  
    "status": "error",
  
    "msg": "An error occurred while retrieving the news categories"
  
    \}
    
    & OK
  
    \\ \hline

  
    \end{xltabular}


  \item URL: GET api/ecg-records/patient/{:patientId}
  

  
  \begin{xltabular}{\textwidth}{
    | >{\raggedright\arraybackslash}m{1cm}
    | >{\raggedright\arraybackslash}X
    | >{\raggedright\arraybackslash}X
    | >{\raggedright\arraybackslash}X
    | >{\raggedright\arraybackslash}m{1cm}|
    }
    \caption{\bfseries \fontsize{12pt}{0pt}\selectfont Bảng API liên quan đến tin tức}
    % \label{table_api_news}
    \\
    \hline
    \bfseries Test case    &\bfseries Điều kiện   &\bfseries Đầu vào 
    &\bfseries Đầu ra mong muốn &\bfseries Kết quả\\ \hline
  
  
    TC-1
    & NULL
    & ID bệnh nhân

    & 
  
    Status code: 200 OK
  
      Response content:
  
      \{
  
    "status": "success",
    "count": Số lượng ECG record của bệnh nhân

    data: Thông tin của tất cả ECG record của bệnh nhân
  
    \}
    
    & OK
  
    \\ \hline
  
    TC-2
    & Lỗi đường truyền server
    & ID bệnh nhân

   &
  
    Status code: 500 Internal Server Error
  
      Response content:
  
      \{
  
    "status": "error",
  
    "msg": "An error occurred while retrieving the news categories"
  
    \}
    
    & OK
  
    \\ \hline

  
    \end{xltabular}




  \item URL: GET api/ecg-records/doctor/{:doctorId}
  

  \begin{xltabular}{\textwidth}{
    | >{\raggedright\arraybackslash}m{1cm}
    | >{\raggedright\arraybackslash}X
    | >{\raggedright\arraybackslash}X
    | >{\raggedright\arraybackslash}X
    | >{\raggedright\arraybackslash}m{1cm}|
    }
    \caption{\bfseries \fontsize{12pt}{0pt}\selectfont Bảng API liên quan đến tin tức}
    % \label{table_api_news}
    \\
    \hline
    \bfseries Test case    &\bfseries Điều kiện   &\bfseries Đầu vào 
    &\bfseries Đầu ra mong muốn &\bfseries Kết quả\\ \hline
  
  
    TC-1
    & NULL
    & ID bác sỹ

    & 
  
    Status code: 200 OK
  
      Response content:
  
      \{
  
    "status": "success",
    data: Thông tin của tất cả ECG record của từng bệnh nhân mà bác sỹ quản lý
  
    \}
    
    & OK
  
    \\ \hline
  
    TC-2
    & Lỗi đường truyền server
    & ID bác sĩ

   &
  
    Status code: 500 Internal Server Error
  
      Response content:
  
      \{
  
    "status": "error",
  
    "msg": "An error occurred while retrieving the news categories"
  
    \}
    
    & OK
  
    \\ \hline

  
    \end{xltabular}



  \item URL: GET api/ecg-records/record-data/{:recordId}
  

  \begin{xltabular}{\textwidth}{
    | >{\raggedright\arraybackslash}m{1cm}
    | >{\raggedright\arraybackslash}X
    | >{\raggedright\arraybackslash}X
    | >{\raggedright\arraybackslash}X
    | >{\raggedright\arraybackslash}m{1cm}|
    }
    \caption{\bfseries \fontsize{12pt}{0pt}\selectfont Bảng API liên quan đến tin tức}
    % \label{table_api_news}
    \\
    \hline
    \bfseries Test case    &\bfseries Điều kiện   &\bfseries Đầu vào 
    &\bfseries Đầu ra mong muốn &\bfseries Kết quả\\ \hline
  
  
    TC-1
    & Có tồn tại ECG record
    & ID ECG record 

    & 
  
    Status code: 200 OK
  
      Response content:
  
      \{
  
    "status": "success",
    count: Tổng số lượng dữ liệu của ECG record
    data: Dữ liệu của ECG record
  
    \}
    
    & OK
  
    \\ \hline
  
    TC-2
    & Không tồn tại ECG record
    & ID ECG record 

   &
  
    Status code: 404 Not Found
  
      Response content:
  
      \{
  
    "status": "error",
  
    "msg": "EcgRecord not found"
  
    \}
    
    & OK
  
    \\ \hline

  
    \end{xltabular}


\end{enumerate}


\paragraph{API liên quan liên quan đến việc phân công bệnh nhân cho bác sỹ
64
}
\mbox{}

Tham khảo bảng \ref{table_api_pat_doc} để xem thông tin của các api liên quan

\begin{enumerate}[a)]
  \item URL: GET api/doctor/{:doctorId}/patients
  

  \begin{xltabular}{\textwidth}{
    | >{\raggedright\arraybackslash}m{1cm}
    | >{\raggedright\arraybackslash}X
    | >{\raggedright\arraybackslash}X
    | >{\raggedright\arraybackslash}X
    | >{\raggedright\arraybackslash}m{1cm}|
    }
    \caption{\bfseries \fontsize{12pt}{0pt}\selectfont Bảng API liên quan đến tin tức}
    % \label{table_api_news}
    \\
    \hline
    \bfseries Test case    &\bfseries Điều kiện   &\bfseries Đầu vào 
    &\bfseries Đầu ra mong muốn &\bfseries Kết quả\\ \hline
  
  
    TC-1
    & NULL
    & ID Doctor

    & 
  
    Status code: 200 OK
  
      Response content:
  
      \{
  
    "status": "success",
    count: Tổng số lượng bệnh nhân mà bác sỹ đang được phân công,
    data: Thông tin của các bệnh nhân mà bác sỹ đang được phân công
  
    \}
    
    & OK
  
    \\ \hline
  
    TC-2
    & Lỗi đường truyền server
    & ID Doctor

   &
  
    Status code: 500 Internal Server Error
  
      Response content:
  
      \{
  
    "status": "error",
  
    "msg": "An error occurred while retrieving the patients"
  
    \}
    
    & OK
  
    \\ \hline

  
    \end{xltabular}


  \item URL: GET api/patient/{:patientId}/doctor
  


  \begin{xltabular}{\textwidth}{
    | >{\raggedright\arraybackslash}m{1cm}
    | >{\raggedright\arraybackslash}X
    | >{\raggedright\arraybackslash}X
    | >{\raggedright\arraybackslash}X
    | >{\raggedright\arraybackslash}m{1cm}|
    }
    \caption{\bfseries \fontsize{12pt}{0pt}\selectfont Bảng API liên quan đến tin tức}
    % \label{table_api_news}
    \\
    \hline
    \bfseries Test case    &\bfseries Điều kiện   &\bfseries Đầu vào 
    &\bfseries Đầu ra mong muốn &\bfseries Kết quả\\ \hline
  
  
    TC-1
    & NULL
    & ID Patient

    & 
  
    Status code: 200 OK
  
      Response content:
  
      \{
  
    "status": "success",
    data: Thông tin của bác sỹ được phân công cho bệnh nhân
  
    \}
    
    & OK
  
    \\ \hline
  
    TC-2
    & Lỗi đường truyền server
    & ID Patient

   &
  
    Status code: 500 Internal Server Error
  
      Response content:
  
      \{
  
    "status": "error",
  
    "msg": "An error occurred while retrieving the patients"
  
    \}
    
    & OK
  
    \\ \hline

  
    \end{xltabular}



\end{enumerate}


\subsubsection{Kiểm thử ứng dụng web}



\newpage
